% arara: pdflatex
%% arara: biber
%% arara: pdflatex
%%%%%%%%%%%%%%%%%%%%%%%%%%%%%%%%%%%%%%%%%
% Wenneker Article
% LaTeX Template
% Version 2.0 (28/2/17)
%
% This template was downloaded from:
% http://www.LaTeXTemplates.com
%
% Authors:
% Vel (vel@LaTeXTemplates.com)
% Frits Wenneker
%
% License:
% CC BY-NC-SA 3.0 (http://creativecommons.org/licenses/by-nc-sa/3.0/)
%
%%%%%%%%%%%%%%%%%%%%%%%%%%%%%%%%%%%%%%%%%

%----------------------------------------------------------------------------------------
%	PACKAGES AND OTHER DOCUMENT CONFIGURATIONS
%----------------------------------------------------------------------------------------

\documentclass[11pt, a4paper, twocolumn]{article} % 10pt font size (11 and 12 also possible), A4 paper (letterpaper for US letter) and two column layout (remove for one column)
%%%%%%%%%%%%%%%%%%%%%%%%%%%%%%%%%%%%%%%%%
% Wenneker Article
% Structure Specification File
% Version 1.0 (28/2/17)
%
% This file originates from:
% http://www.LaTeXTemplates.com
%
% Authors:
% Frits Wenneker
% Vel (vel@LaTeXTemplates.com)
%
% License:
% CC BY-NC-SA 3.0 (http://creativecommons.org/licenses/by-nc-sa/3.0/)
%
%%%%%%%%%%%%%%%%%%%%%%%%%%%%%%%%%%%%%%%%%

%----------------------------------------------------------------------------------------
%	PACKAGES AND OTHER DOCUMENT CONFIGURATIONS
%----------------------------------------------------------------------------------------

\usepackage[english]{babel} % English language hyphenation

\usepackage{microtype} % Better typography

\usepackage{amsmath,amsfonts,amsthm} % Math packages for equations

\usepackage[svgnames]{xcolor} % Enabling colors by their 'svgnames'

\usepackage[hang, small, labelfont=bf, up, textfont=it]{caption} % Custom captions under/above tables and figures

\usepackage{booktabs} % Horizontal rules in tables

\usepackage{lastpage} % Used to determine the number of pages in the document (for "Page X of Total")

\usepackage{graphicx} % Required for adding images
\usepackage{svg}
\usepackage{pstricks}
\def\svgwidth{0.5\textwidth}

\usepackage{enumitem} % Required for customising lists
\setlist{noitemsep} % Remove spacing between bullet/numbered list elements

\usepackage{xparse}
%\usepackage{todonotes}

\usepackage{sectsty} % Enables custom section titles
\allsectionsfont{\usefont{OT1}{phv}{b}{n}} % Change the font of all section commands (Helvetica)

%----------------------------------------------------------------------------------------
%	MARGINS AND SPACING
%----------------------------------------------------------------------------------------

\usepackage{geometry} % Required for adjusting page dimensions

\geometry{
	top=1cm, % Top margin
	bottom=1.5cm, % Bottom margin
	left=2cm, % Left margin
	right=2cm, % Right margin
	includehead, % Include space for a header
	includefoot, % Include space for a footer
	%showframe, % Uncomment to show how the type block is set on the page
}

\setlength{\columnsep}{7mm} % Column separation width

%----------------------------------------------------------------------------------------
%	FONTS
%----------------------------------------------------------------------------------------

\usepackage[T1]{fontenc} % Output font encoding for international characters
\usepackage[utf8]{inputenc} % Required for inputting international characters

\usepackage{XCharter} % Use the XCharter font

%----------------------------------------------------------------------------------------
%	HEADERS AND FOOTERS
%----------------------------------------------------------------------------------------

\setlength{\headheight}{52pt}% 
\usepackage{fancyhdr} % Needed to define custom headers/footers
\pagestyle{fancy} % Enables the custom headers/footers

\renewcommand{\headrulewidth}{0.0pt} % No header rule
\renewcommand{\footrulewidth}{0.4pt} % Thin footer rule

\renewcommand{\sectionmark}[1]{\markboth{#1}{}} % Removes the section number from the header when \leftmark is used

%\nouppercase\leftmark % Add this to one of the lines below if you want a section title in the header/footer

% Headers
\lhead{} % Left header
\chead{\textit{\thetitle}} % Center header - currently printing the article title
\rhead{} % Right header

% Footers
\lfoot{} % Left footer
\cfoot{} % Center footer
\rfoot{\footnotesize Page \thepage\ of \pageref{LastPage}} % Right footer, "Page 1 of 2"

\fancypagestyle{firstpage}{ % Page style for the first page with the title
	\fancyhf{}
	\renewcommand{\footrulewidth}{0pt} % Suppress footer rule
}

%----------------------------------------------------------------------------------------
%	TITLE SECTION
%----------------------------------------------------------------------------------------

\newcommand{\authorstyle}[1]{{\large\usefont{OT1}{phv}{b}{n}\color{DarkRed}#1}} % Authors style (Helvetica)

\newcommand{\institution}[1]{{\footnotesize\usefont{OT1}{phv}{m}{sl}\color{Black}#1}} % Institutions style (Helvetica)

\usepackage{titling} % Allows custom title configuration

\newcommand{\HorRule}{\color{DarkGoldenrod}\rule{\linewidth}{1pt}} % Defines the gold horizontal rule around the title

\pretitle{
	\vspace{-30pt} % Move the entire title section up
	\HorRule\vspace{10pt} % Horizontal rule before the title
	\fontsize{32}{36}\usefont{OT1}{phv}{b}{n}\selectfont % Helvetica
	\color{DarkRed} % Text colour for the title and author(s)
}

\posttitle{\par\vskip 15pt} % Whitespace under the title

\preauthor{} % Anything that will appear before \author is printed

\postauthor{ % Anything that will appear after \author is printed
	\vspace{10pt} % Space before the rule
	\par\HorRule % Horizontal rule after the title
	\vspace{20pt} % Space after the title section
}

%----------------------------------------------------------------------------------------
%	ABSTRACT
%----------------------------------------------------------------------------------------

\usepackage{lettrine} % Package to accentuate the first letter of the text (lettrine)
\usepackage{fix-cm}	% Fixes the height of the lettrine

\newcommand{\initial}[1]{ % Defines the command and style for the lettrine
	\lettrine[lines=3,findent=4pt,nindent=0pt]{% Lettrine takes up 3 lines, the text to the right of it is indented 4pt and further indenting of lines 2+ is stopped
		\color{DarkGoldenrod}% Lettrine colour
		{#1}% The letter
	}{}%
}

\usepackage{xstring} % Required for string manipulation

\newcommand{\lettrineabstract}[1]{
	\StrLeft{#1}{1}[\firstletter] % Capture the first letter of the abstract for the lettrine
	\initial{\firstletter}\textbf{\StrGobbleLeft{#1}{1}} % Print the abstract with the first letter as a lettrine and the rest in bold
}

%----------------------------------------------------------------------------------------
%	BIBLIOGRAPHY
%----------------------------------------------------------------------------------------

\usepackage[backend=biber,style=authoryear,natbib=true]{biblatex} % Use the bibtex backend with the authoryear citation style (which resembles APA)

\addbibresource{mybib.bib} % The filename of the bibliography

\usepackage[autostyle=true]{csquotes} % Required to generate language-dependent quotes in the bibliography


%----------------------------------------------------------------------------------------
%	Theorems and Definitions
%   https://tex.stackexchange.com/questions/45817/theorem-definition-lemma-problem-numbering
%----------------------------------------------------------------------------------------

\newtheoremstyle{break}%                % Name
  {}%                                     % Space above
  {}%                                     % Space below
  {}%                             % Body font
  {}%                                     % Indent amount
  {\itshape\bfseries}%                            % Theorem head font
  {}%                                    % Punctuation after theorem head
  {\newline}%                                    % Space after theorem head, ' ', or \newline
  {}%                                     % Theorem head spec (can be left empty, meaning `normal')

\newtheoremstyle{nte}%                % Name
  {}%                                     % Space above
  {}%                                     % Space below
  {}%                             % Body font
  {}%                                     % Indent amount
  {\itshape}%                            % Theorem head font
  {:}%                                    % Punctuation after theorem head
  { }%                                    % Space after theorem head, ' ', or \newline
  {}%                                     % Theorem head spec (can be left empty, meaning `normal')


\newtheoremstyle{trm}%                % Name
  {}%                                     % Space above
  {}%                                     % Space below
  {}%                             % Body font
  {}%                                     % Indent amount
  {\itshape}%                            % Theorem head font
  {:}%                                    % Punctuation after theorem head
  {\newline}%                                    % Space after theorem head, ' ', or \newline
  {}%                                     % Theorem head spec (can be left empty, meaning `normal')

\theoremstyle{break}

\renewenvironment{proof}{{\bf {Proof:} }}{\hfill $\Box$ \\} 
\newtheorem{theorem}{Theorem}[section] % reset theorem numbering for each chapter
\newtheorem{defn}[theorem]{Definition} % definition numbers are dependent on theorem numbers
\newtheorem{exmp}[theorem]{Example} % same for example numbers
\newtheorem{corollary}[theorem]{Corollary} % same for example numbers
%\newtheorem{proof}[theorem]{Proof} % 


\theoremstyle{nte}
\newtheorem*{note}{Note}
\newtheorem*{terminology}{Terminology}

\theoremstyle{trm}
%\newtheorem*{terminology}{Terminology}



%----------------------------------------------------------------------------------------
%	New definitions
%----------------------------------------------------------------------------------------
\usepackage{mathtools}
\usepackage{tensor}
\usepackage{tikz}
\usepackage{pgfplots}
\usepackage{hyperref}

\newcommand{\M}{\mathcal{M}}
\newcommand{\calL}{\mathcal{L}}
\newcommand{\calO}{\mathcal{O}}
\newcommand{\calA}{\mathcal{A}}
\newcommand{\Lie}{\mathcal{L}}

%\newcommand{\diff}{\mathrm{d}\!}
\DeclareMathOperator{\diff}{d\!}
\DeclareMathOperator{\difff}{d}
\DeclareMathOperator{\diag}{diag}
\newcommand{\eval}{\big|}
\newcommand{\linearto}{\xrightarrow[]{\sim}}
\DeclareMathOperator{\preim}{preim}
\DeclareDocumentCommand{\T}{o}
{%
    \IfNoValueTF{#1}{\operatorname{T}\!}{\operatorname{T}_{#1}\!}
}
\DeclareMathOperator{\Riem}{Riem}
%\DeclareMathOperator{\T}{T\!}
%\newcommand{\T}[1][]{\operatorename{T}_{#1} \!}

\makeatletter
\def\moverlay{\mathpalette\mov@rlay}
\def\mov@rlay#1#2{\leavevmode\vtop{%
   \baselineskip\z@skip \lineskiplimit-\maxdimen
   \ialign{\hfil$\m@th#1##$\hfil\cr#2\crcr}}}
\newcommand{\charfusion}[3][\mathord]{
    #1{\ifx#1\mathop\vphantom{#2}\fi
        \mathpalette\mov@rlay{#2\cr#3}
      }
    \ifx#1\mathop\expandafter\displaylimits\fi}
\makeatother

\newcommand{\cupdot}{\charfusion[\mathbin]{\cup}{\cdot}}
\newcommand{\bigcupdot}{\charfusion[\mathop]{\bigcup}{\cdot}}

%https://tex.stackexchange.com/a/229156
\newcommand{\subscript}[2]{$#1 _ #2$} 


\hypersetup{
    unicode=false,          % non-Latin characters in Acrobat’s bookmarks
    pdftoolbar=true,        % show Acrobat’s toolbar?
    pdfmenubar=true,        % show Acrobat’s menu?
    pdffitwindow=false,     % window fit to page when opened
    pdfstartview={FitH},    % fits the width of the page to the window
    pdftitle={},    % title
    pdfauthor={},     % author
    pdfsubject={},   % subject of the document
    pdfcreator={},   % creator of the document
    pdfproducer={}, % producer of the document
    pdfkeywords={}, % list of keywords
    pdfnewwindow=true,      % links in new PDF window
    colorlinks=false,       % false: boxed links; true: colored links
    linkcolor=red,          % color of internal links (change box color with linkbordercolor)
    citecolor=green,        % color of links to bibliography
    filecolor=magenta,      % color of file links
    urlcolor=cyan           % color of external links
}
%\usepackage{url}
%\pgfplotsset{compat=1.16}


%----------------------------------------------------------------------------------------
%	TIKZ and PGFPLOTS
%----------------------------------------------------------------------------------------
% https://tex.stackexchange.com/questions/382762/drawing-manifolds-in-tikz
\usepgfplotslibrary{patchplots}
\usetikzlibrary{patterns, positioning, arrows}
\usetikzlibrary{matrix}
\usetikzlibrary{backgrounds}
\pgfplotsset{compat=1.15}

\pgfplotsset{compat=1.7}
\pgfmathsetmacro\sprayRadius{.25pt}
\pgfmathsetmacro\sprayPeriod{.6cm}

\pgfdeclarepatternformonly{spray}{\pgfpoint{-\sprayRadius}{-\sprayRadius}}{\pgfpoint{1cm + \sprayRadius}{1cm + \sprayRadius}}{\pgfpoint{\sprayPeriod}{\sprayPeriod}}{
    \foreach \x/\y in {2/53,6/52,11/48,23/49,20/47,32/46,41/47,47/51,56/52,46/44,4/43,16/42,33/41,41/37,49/35,55/31,37/35,44/30,28/37,24/36,17/37,7/38,0/31,8/29,18/31,28/30,37/28,30/27,46/24,51/21,24/23,12/24,4/21,18/19,12/16,31/21,38/18,26/16,46/16,56/12,52/10,45/8,51/4,37/12,35/7,24/9,14/9,2/12,8/6,15/4,27/0,34/1,40/1} {
    \pgfpathcircle{\pgfpoint{(\x + random()) / 57 * \sprayPeriod}{\sprayPeriod - (\y + random()) / 55 * \sprayPeriod}}{\sprayRadius}
    }
\pgfusepath{fill}
}

\newcommand{\boxalign}[2][0.97\textwidth]{%
  \par\noindent\tikzstyle{mybox} = [draw=black,inner sep=6pt]
  \begin{center}\begin{tikzpicture}
   \node [mybox] (box){%
    \begin{minipage}{#1}{\vspace{-5mm}#2}\end{minipage}
   };
  \end{tikzpicture}\end{center}
}

%https://www.math.lsu.edu/~aperlis/publications/mathclap/perlis_mathclap_24Jun2003.pdf
\def\mathclap#1{\text{\hbox to 0pt{\hss$\mathsurround=0pt#1$\hss}}}
%----------------------------------------------------------------------------------------
%	SYMBOLS
%----------------------------------------------------------------------------------------
% 
%\usepackage{bbding}
%\usepackage{pifont}
%\usepackage{wasysym}
\usepackage{amsfonts}
\usepackage{cancel}
%\newcommand{\flower}{\SixFlowerRemovedOpenPetal}
\newcommand{\flower}{\star}


%----------------------------------------------------------------------------------------
%	EXAMPLE ENVIRONMENT
%   https://tex.stackexchange.com/questions/21227/example-environment/21241
%----------------------------------------------------------------------------------------
\usepackage[most]{tcolorbox}
\newcounter{examples}

\def\exampletext{Example} % If English

\NewDocumentEnvironment{example}{ O{} }
{
\colorlet{colexam}{red!55!black} % Global example color
\newtcolorbox[use counter=examples]{examplebox}{%
    % Example Frame Start
    empty,% Empty previously set parameters
    title={\exampletext: #1},% use \thetcbcounter to access the testexample counter text
    % Attaching a box requires an overlay
    attach boxed title to top left,
       % Ensures proper line breaking in longer titles
       minipage boxed title,
    % (boxed title style requires an overlay)
    boxed title style={empty,size=minimal,toprule=0pt,top=4pt,left=3mm,overlay={}},
    coltitle=colexam,fonttitle=\bfseries,
    before=\par\medskip\noindent,parbox=false,boxsep=0pt,left=3mm,right=0mm,top=2pt,breakable,pad at break=0mm,
       before upper=\csname @totalleftmargin\endcsname0pt, % Use instead of parbox=true. This ensures parskip is inherited by box.
    % Handles box when it exists on one page only
    overlay unbroken={\draw[colexam,line width=.5pt] ([xshift=-0pt]title.north west) -- ([xshift=-0pt]frame.south west); },
    % Handles multipage box: first page
    overlay first={\draw[colexam,line width=.5pt] ([xshift=-0pt]title.north west) -- ([xshift=-0pt]frame.south west); },
    % Handles multipage box: middle page
    overlay middle={\draw[colexam,line width=.5pt] ([xshift=-0pt]frame.north west) -- ([xshift=-0pt]frame.south west); },
    % Handles multipage box: last page
    overlay last={\draw[colexam,line width=.5pt] ([xshift=-0pt]frame.north west) -- ([xshift=-0pt]frame.south west); },%
    }
\begin{examplebox}}
{\end{examplebox}\endlist}
 % Specifies the document structure and loads requires packages

%----------------------------------------------------------------------------------------
%	ARTICLE INFORMATION
%----------------------------------------------------------------------------------------

%\title{To those physics students who asked why $q$ and $\dot{q}$ are independent in Lagrangian Mechanics}
\title{Mathematical Notes on Manifolds in Physics}

\author{%
	\authorstyle{Niklas Zorbach\textsuperscript{1} and Marco Knipfer\textsuperscript{1, 2}} % Authors
	\newline\newline % Space before institutions
	\textsuperscript{1}\institution{Institute for Theoretical Physics, Goethe-University Frankfurt, Germany}\\ 
	\textsuperscript{2}\institution{Institute for Physics and Astronomy, The University of Alabama, USA}\\ 
}

% Example of a one line author/institution relationship
%\author{\newauthor{John Marston} \newinstitution{Universidad Nacional Autónoma de México, Mexico City, Mexico}}

\date{\today} 

%----------------------------------------------------------------------------------------

\begin{document}

\maketitle % Print the title

\thispagestyle{firstpage} % Apply the page style for the first page (no headers and footers)

%----------------------------------------------------------------------------------------
%	ABSTRACT
%----------------------------------------------------------------------------------------

\lettrineabstract{Lorem ipsum dolor sit amet, consectetur adipiscing elit. Fusce maximus nisi ligula. Morbi laoreet ex ligula, vitae lobortis purus mattis vel. Vestibulum ante ipsum primis in faucibus orci luctus et ultrices posuere cubilia Curae; Donec ac metus ut turpis mollis placerat et nec enim. Duis tristique nibh maximus faucibus facilisis. Praesent in consequat leo. Maecenas condimentum ex rhoncus, elementum diam vel, malesuada ante.}

%----------------------------------------------------------------------------------------
%	ARTICLE CONTENTS
%----------------------------------------------------------------------------------------

\section{Manifolds}
\subsection{Topology}
This chapter is basically taken from \citep{Schuller15} with our remarks to it.

We start with a set $M$ which is supposed to be the space where physics happens.
The weakest structure we need in order to talk about continuity (of curves or fields)
is called a topology.

\begin{defn}[Power set $\mathcal{P}$]
    The set of all subsets of $M$.
\end{defn}

\begin{defn}[Topology]
    A Topology $\mathcal{O}$ is a subset $\mathcal{O} \subseteq \mathcal{P}(M)$ satisfying:
    \begin{enumerate}
        \item $\emptyset \in \mathcal{O}$, $M\in\mathcal{O}$,
        \item $U\in\mathcal{O},\quad V\in\mathcal{O}\Rightarrow U\cap V\in\mathcal{O}$
        \item $U_\alpha\in\mathcal{O},\quad \alpha\in A \Rightarrow
            \left( \bigcup\limits_{\alpha\in A} U_\alpha \right) \in \mathcal{O}$
    \end{enumerate}
\end{defn}
Every set has the \textit{chaotic topology}
\begin{equation}
    \mathcal{O}_\text{chaotic} := \left\{ \emptyset, M \right\}\,,
\end{equation}
and the \textit{discrete topology}
\begin{equation}
    \mathcal{O}_\text{discrete} := \mathcal{P}(M)\,,
\end{equation}
which are both useless.

The special case $M = \mathbb{R}^d = \mathbb{R} \times \cdots \times \mathbb{R}$ has a standard topology
for which we need the definition of a soft ball.
\begin{defn}[Soft Ball in $\mathbb{R}^d$]
   \begin{equation}
       B_r(p) := \left\{ (q_1,\cdots,q_d)| \sum_{i=1}^{d}(p_i-p_i) < r \right\}\,,
   \end{equation} 
   with $r\in \mathbb{R}^+$, $p\in\mathbb{R}^d$.
   Note: This does not need a norm or vector space structure on $\mathbb{R}^d$.
\end{defn}
\begin{defn}[$\mathcal{O}_\text{standard}$ on $\mathbb{R}^d$]
    \begin{equation}
        U \in \mathcal{O}_\text{standard} :\Leftrightarrow \forall p\in U:
        \exists r\in\mathbb{R}^+ : B_r(p) \subseteq U
    \end{equation}
\end{defn}

\begin{figure}[h]
\centering
\begin{tikzpicture}
    %\draw[smooth cycle,tension=3.0] plot coordinates{(1,0) (1,1) (2,2) (3,3) (6,0)} node [label=right:$M$];

    % Manifold
    \draw[smooth cycle, tension=0.4] plot coordinates{(2,4) (-1.5,0) (3.5,-2) (6,1)} node [label=$M$, anchor = west] {};
    \draw[smooth cycle, tension=1, dashed] plot coordinates{(1,1) (-1,0) (1,-1) (2.5,0.5)} node [label=$U$] {};
    \draw[smooth cycle, tension=0.4, dashed] plot coordinates{(1, 2) (1,3) (3,3) (3,2)} node [label=$V$] {};
    \draw[smooth cycle, tension=0.4] plot coordinates{(1, 2) (1,3) (3,3) (3,2)} node [label=$V$] {};
    \draw[dashed] (-0.5,0) circle (.4cm) node [label = $B_r$, anchor = north west] {};
    \draw[fill] (-0.5,0) circle (.05cm) ;
    \draw (1.5,1.95) circle (.2cm) node [label = $B_r$, anchor = north west] {};
    \draw[fill] (1.5,1.95) circle (.05cm) ;

    %\draw[help lines] (-3,-6) grid (8,6);

\end{tikzpicture}
\caption{The set $U$ is in the standard topology, $V$ not.}
\end{figure}

Some terminology: Let $M$ be a set with a topology $\mathcal{O} =:$ set of open sets.
We call $(M, \mathcal{O}$ a \textit{topological space} and:
    \begin{itemize}
        \item $U \in \mathcal{O} \Leftrightarrow: \text{call } U\subseteq M$ an \textit{open set}
        \item $M\backslash A \in \mathcal{O} \Leftrightarrow: \text{call } U\subseteq M$  a \textit{closed set}
    \end{itemize}
\begin{note}
    The empty set is open and closed. If a set is open we cannot directly follow that it is not closed or vise versa. For $M = \left\{ 1, 2 \right\}$ and $\mathcal{O}_M = \left\{ \emptyset, \left\{ 1 \right\}, \left\{ 2 \right\}, \left\{ 1,2 \right\} \right\}$ the set $\left\{ 2 \right\}$ is open and closed.
\end{note}

\subsection{Continuous Maps}
A map
\begin{equation}
    f: M \to N\,,
\end{equation}
takes every point from the domain M (a set) to the target N (a set).
If one point $p\in N$ is not reached, the map is not \textit{surjective}.
If a point is hit twice, the map is not \textit{injective}.
A map that is injective and surjective is called \textit{surjective}.

\begin{defn}[Preimage]
    \begin{align}
        f&: M \to N \supseteq V \nonumber \\
        \text{preim}_f(V) &:= \left\{ m\in M~|~f(f) \in V \right\}
    \end{align}
\end{defn}

\begin{defn}[Continuity]
    $(M, \mathcal{O}_M)$ and $(N, \mathcal{O}_N)$ topological spaces.
    Then a map $f: M \to N$ is called \textit{continuous with respect to $\mathcal{O}_M$ and
    $\mathcal{O}_N$} if
    \begin{equation}
    \forall V\in \mathcal{O}_N: \text{preim}_f(V) \in \mathcal{O}_M\,.
    \end{equation}
    \textit{``A map is open iff the preimages of all open sets are open sets.''}
\end{defn}

\begin{note}
    If a map is not surjective there are sets with preimage $\emptyset$, thus we need to have
    $\emptyset$ in $\mathcal{O}$, otherwise only surjective maps could be continuous.
\end{note}
\begin{note}
    The inverse of a continuous function does not need to be continuous.
\end{note}

\begin{defn}[Composition of maps]
    For $f$ and $g$
    \begin{equation*}
        f: M \to N, \quad g: N \to P\,,
    \end{equation*}
    we define the \textit{composition} as
    \begin{align}
        g \circ f&: M \to P \\
        m &\mapsto (g\circ f)(m) := g(f(m)) \nonumber
    \end{align}
\end{defn}

\begin{theorem}[Composition of continuos maps]
    For $f$, $g$ continuos also $g\circ f$ is continuous (if spaces match).
\end{theorem}


\begin{defn}[Subset topology, Inherited topology]
    A set $M$ with topology $\mathcal{O}_M$. Given any subset $S \subseteq M$ we
    can construct the inherited topology $\mathcal{O}\eval_S \subseteq \mathcal{P}(S)$
    \begin{equation}
        \mathcal{O}\eval_S := \left\{ U\cap S~|~U\in \mathcal{O}M \right\}\,.
    \end{equation}
\end{defn}

\begin{note}
    For $S \subseteq M$, if $f$ is continuous then $f\eval_{S}$ is also
    continuous if $\mathcal{O}\eval_S$ is chosen. This is for example important
    if you are on a trajectory $\gamma$ through $\mathbb{R}^n$ and measure the temperature
    $T\eval_\gamma$.
\end{note}

\begin{defn}[Topological manifold]
    A topological space $(\M, \mathcal{O})$ is called a \textit{d-dimensional topological manifold}
    if
    \begin{equation}
        \forall p \in \M: \exists U \in \mathcal{O},~p\in U: \exists x: U\to x(U)\subseteq \mathbb{R}^{d}
        \,,
    \end{equation}
    with the following properties (wrt. $\mathcal O_\text{std}$ on $\mathbb{R}^{d}$):
    \begin{enumerate}
        \item $x$ intervitble: $x^{-1} : x(U) \to U$,
        \item $x$ continuous,
        \item $x^{-1}$ continuous\,.
    \end{enumerate}
    ``Invertible, in both directions continuous map to $\mathbb{R}^{n}$.''
\end{defn}

\begin{note}
    Thus in the above definition $x(U)$ is also open (from the definition of continuity).
\end{note}

\begin{terminology}
    \begin{itemize}
        \item $(U,x)$ is a \textit{chart} of $\M, \mathcal{O}$,
        \item $\mathcal{A} = \left\{ (U_{(\alpha)}, x_{(\alpha)} | \alpha \in A \right\}$ is an \textit{atlas} of $(\M, \mathcal{O})$ if $\bigcup\limits_{\alpha\in A} U_{(\alpha)}$ covers the whole manifold $\M$,
        \item $x: U \to x(U) \subseteq \mathbb{R}^{d}$ is a \textit{chart map} 
            $x(p) = (x^1(p), \dots, x^d(p))$, where the \textit{component maps} $x^i: U\to\mathbb{R}$ are called \textit{coordinate maps},
        \item $p \in U$, then $x^1(p)$ is the first coordinate of the point p wrt.\ the chosen chart $(U, x)$.
    \end{itemize}
\end{terminology}
\begin{note}
    The choice of the chart (choice of coordinates) has nothing to do with the physics.
    Physics is chart independent. $\M$ is ``the real world''.
\end{note}

\subsection{Chart Transition Maps}

\begin{figure*}[tbh]
    \centering
    \begin{tikzpicture}
    % https://tex.stackexchange.com/questions/382762/drawing-manifolds-in-tikz
    % Functions i
        \path[->] (0.8, 0) edge [bend right] node[left, xshift=-2mm] {$x$} (-1, -2.9);
        \draw[white,fill=white] (0.06,-0.57) circle (.15cm);
        \path[->] (-0.7, -3.05) edge [bend right] node [right, yshift=-3mm] {$x^{-1}$} (1.093, -0.11);
        \draw[white, fill=white] (0.95,-1.2) circle (.15cm);

    % Functions j
        \path[->] (5.8, -2.8) edge [bend left] node[midway, xshift=-5mm, yshift=-3mm] {$y^{-1}$} (3.8, -0.35);
        \draw[white, fill=white] (4,-1.1) circle (.15cm);
        \path[->] (4.2, 0) edge [bend left] node[right, xshift=2mm] {$y$} (6.2, -2.8);
        \draw[white, fill=white] (4.54,-0.12) circle (.15cm);

    % Manifold
        \draw[smooth cycle, tension=0.4, fill=white, pattern color=brown, pattern=spray, opacity=0.7] plot coordinates{(2,2) (-0.5,0) (3,-2) (5,1)} node at (3,2.3) {$\M$};

    % Help lines
    %\draw[help lines] (-3,-6) grid (8,6);

    % Subsets
        \draw[smooth cycle, dashed, pattern color=orange, pattern=crosshatch dots, fill opacity = 0.4] 
        plot coordinates {(1,0) (1.5, 1.2) (2.5,1.3) (2.6, 0.4)} 
        node [label={[label distance=-0.3cm, xshift=-2cm, fill=white]:$U$}] {};
        \draw[smooth cycle, dashed, pattern color=blue, pattern=crosshatch dots, fill opacity = 0.4] 
        plot coordinates {(4, 0) (3.7, 0.8) (3.0, 1.2) (2.5, 1.2) (2.2, 0.8) (2.3, 0.5) (2.6, 0.3) (3.5, 0.0)} 
        node [label={[label distance=-0.8cm, xshift=.75cm, yshift=1cm, fill=white]:$V$}] {};

    % First Axis
        \draw[thick, ->] (-3,-5) -- (0, -5) node [label=above:$x(U)$] {};
        \draw[thick, ->] (-3,-5) -- (-3, -2) node [label=right:$\mathbb{R}^d$] {};

    % Arrow from i to j
        \draw[->] (0, -3.85) -- node[midway, above]{$y\circ x^{-1}$} (4.5, -3.85);

    % Second Axis
        \draw[thick, ->] (5, -5) -- (8, -5) node [label=above:$x(V)$] {};
        \draw[thick, ->] (5, -5) -- (5, -2) node [label=right:$\mathbb{R}^d$] {};

    % Sets in R^m
        \draw[white, dashed, pattern color=blue, pattern=crosshatch dots, fill opacity = 0.4] (-0.67, -3.06) -- +(180:0.8) arc (180:270:0.8);
        \fill[even odd rule, white] [smooth cycle] plot coordinates{(-2, -4.5) (-2, -3.2) (-0.8, -3.2) (-0.8, -4.5)} (-0.67, -3.06) -- +(180:0.8) arc (180:270:0.8);
        \draw[smooth cycle, dashed, pattern color = orange, pattern = crosshatch dots, fill opacity = 0.4] plot coordinates{(-2, -4.5) (-2, -3.2) (-0.8, -3.2) (-0.8, -4.5)};
        \draw[dashed] (-1.45, -3.06) arc (180:270:0.8);

        \draw[white, dashed, pattern color=orange, pattern=crosshatch dots, fill opacity = 0.4] (5.7, -3.06) -- +(-90:0.8) arc (-90:0:0.8);
        \fill[even odd rule, white] [smooth cycle] plot coordinates{(7, -4.5) (7, -3.2) (5.8, -3.2) (5.8, -4.5)} (5.7, -3.06) -- +(-90:0.8) arc (-90:0:0.8);
        \draw[smooth cycle, dashed, pattern color = blue, pattern = crosshatch dots, fill opacity = 0.4] plot coordinates{(7, -4.5) (7, -3.2) (5.8, -3.2) (5.8, -4.5)};
        \draw[dashed] (5.69, -3.85) arc (-90:0:0.8);

    \end{tikzpicture}

    \caption{Visualization of chart transition maps. ``How to glue together the charts of an atlas.'' Plot modified from~\citep{texstackexchange:manifolds}}
    \label{fig:transitionmaps}
\end{figure*}
Given $(U, x)$ and $(V, y)$ charts, on $U\cup V$ one can transition from
one chart to the other by (see figure~\ref{fig:transitionmaps})
\begin{equation}
    y\circ x^{-1}: \mathbb{R}^{d} \supseteq x(U\cap V) \to y(U\cap V) \subseteq \mathbb{R}^{d}\,,
\end{equation}
which is called the \textit{chart transition map}.
\begin{note}
    As a physicist one talks about a ``change in coordinates''.
\end{note}

\subsection{Manifold Philosophy}

The idea is to define properties of some object in the real world $\M$ by at a chart-representative
of it. For example the continuity of a curve $\gamma: [0,1]\to\M$ can be judged by looking at
$x\circ\gamma: [0,1]\to\mathbb{R}^{d}$, because $x$ is invertible and in both directions continuous
and the composition of two continuous maps is also continuous.
\begin{note}
    One needs to make sure that the property of the object on $\M$ does not depend on the
    map $x$ or $y$. For continuity this is the case, since 
    $y\circ\gamma = (y\circ x^{-1}) \circ x\circ \gamma$ and the chart transition map
    $y\circ x^{-1}$ is also continuous.
\end{note}

Other properties like ``differentiability'' are not even defined on $\M$ a priori,
so one can only talk about the chart representative. Here the definition that $\gamma$
is differentiable iff $x\circ\gamma: [0,1]\to\mathbb{R}^{d}$ is differentiable has the
problem that $x$ and $y$ only need to be continuous and so the chart transition map
$y\circ x^{-1}$ does not need to be differentiable unless one restricts oneself to only
differentiable charts.
%------------------------------------------------


%----------------------------------------------------------------------------------------
%	BIBLIOGRAPHY
%----------------------------------------------------------------------------------------

\printbibliography[title={Bibliography}] % Print the bibliography, section title in curly brackets

%----------------------------------------------------------------------------------------

\end{document}
