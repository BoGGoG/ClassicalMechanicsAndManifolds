% arara: pdflatex
%% arara: biber
%% arara: pdflatex
%%%%%%%%%%%%%%%%%%%%%%%%%%%%%%%%%%%%%%%%%
% Wenneker Article
% LaTeX Template
% Version 2.0 (28/2/17)
%
% This template was downloaded from:
% http://www.LaTeXTemplates.com
%
% Authors:
% Vel (vel@LaTeXTemplates.com)
% Frits Wenneker
%
% License:
% CC BY-NC-SA 3.0 (http://creativecommons.org/licenses/by-nc-sa/3.0/)
%
%%%%%%%%%%%%%%%%%%%%%%%%%%%%%%%%%%%%%%%%%

%----------------------------------------------------------------------------------------
%	PACKAGES AND OTHER DOCUMENT CONFIGURATIONS
%----------------------------------------------------------------------------------------

\documentclass[11pt, a4paper, twocolumn]{article} % 10pt font size (11 and 12 also possible), A4 paper (letterpaper for US letter) and two column layout (remove for one column)
%%%%%%%%%%%%%%%%%%%%%%%%%%%%%%%%%%%%%%%%%
% Wenneker Article
% Structure Specification File
% Version 1.0 (28/2/17)
%
% This file originates from:
% http://www.LaTeXTemplates.com
%
% Authors:
% Frits Wenneker
% Vel (vel@LaTeXTemplates.com)
%
% License:
% CC BY-NC-SA 3.0 (http://creativecommons.org/licenses/by-nc-sa/3.0/)
%
%%%%%%%%%%%%%%%%%%%%%%%%%%%%%%%%%%%%%%%%%

%----------------------------------------------------------------------------------------
%	PACKAGES AND OTHER DOCUMENT CONFIGURATIONS
%----------------------------------------------------------------------------------------

\usepackage[english]{babel} % English language hyphenation

\usepackage{microtype} % Better typography

\usepackage{amsmath,amsfonts,amsthm} % Math packages for equations

\usepackage[svgnames]{xcolor} % Enabling colors by their 'svgnames'

\usepackage[hang, small, labelfont=bf, up, textfont=it]{caption} % Custom captions under/above tables and figures

\usepackage{booktabs} % Horizontal rules in tables

\usepackage{lastpage} % Used to determine the number of pages in the document (for "Page X of Total")

\usepackage{graphicx} % Required for adding images
\usepackage{svg}
\usepackage{pstricks}
\def\svgwidth{0.5\textwidth}

\usepackage{enumitem} % Required for customising lists
\setlist{noitemsep} % Remove spacing between bullet/numbered list elements

\usepackage{xparse}
%\usepackage{todonotes}

\usepackage{sectsty} % Enables custom section titles
\allsectionsfont{\usefont{OT1}{phv}{b}{n}} % Change the font of all section commands (Helvetica)

%----------------------------------------------------------------------------------------
%	MARGINS AND SPACING
%----------------------------------------------------------------------------------------

\usepackage{geometry} % Required for adjusting page dimensions

\geometry{
	top=1cm, % Top margin
	bottom=1.5cm, % Bottom margin
	left=2cm, % Left margin
	right=2cm, % Right margin
	includehead, % Include space for a header
	includefoot, % Include space for a footer
	%showframe, % Uncomment to show how the type block is set on the page
}

\setlength{\columnsep}{7mm} % Column separation width

%----------------------------------------------------------------------------------------
%	FONTS
%----------------------------------------------------------------------------------------

\usepackage[T1]{fontenc} % Output font encoding for international characters
\usepackage[utf8]{inputenc} % Required for inputting international characters

\usepackage{XCharter} % Use the XCharter font

%----------------------------------------------------------------------------------------
%	HEADERS AND FOOTERS
%----------------------------------------------------------------------------------------

\setlength{\headheight}{52pt}% 
\usepackage{fancyhdr} % Needed to define custom headers/footers
\pagestyle{fancy} % Enables the custom headers/footers

\renewcommand{\headrulewidth}{0.0pt} % No header rule
\renewcommand{\footrulewidth}{0.4pt} % Thin footer rule

\renewcommand{\sectionmark}[1]{\markboth{#1}{}} % Removes the section number from the header when \leftmark is used

%\nouppercase\leftmark % Add this to one of the lines below if you want a section title in the header/footer

% Headers
\lhead{} % Left header
\chead{\textit{\thetitle}} % Center header - currently printing the article title
\rhead{} % Right header

% Footers
\lfoot{} % Left footer
\cfoot{} % Center footer
\rfoot{\footnotesize Page \thepage\ of \pageref{LastPage}} % Right footer, "Page 1 of 2"

\fancypagestyle{firstpage}{ % Page style for the first page with the title
	\fancyhf{}
	\renewcommand{\footrulewidth}{0pt} % Suppress footer rule
}

%----------------------------------------------------------------------------------------
%	TITLE SECTION
%----------------------------------------------------------------------------------------

\newcommand{\authorstyle}[1]{{\large\usefont{OT1}{phv}{b}{n}\color{DarkRed}#1}} % Authors style (Helvetica)

\newcommand{\institution}[1]{{\footnotesize\usefont{OT1}{phv}{m}{sl}\color{Black}#1}} % Institutions style (Helvetica)

\usepackage{titling} % Allows custom title configuration

\newcommand{\HorRule}{\color{DarkGoldenrod}\rule{\linewidth}{1pt}} % Defines the gold horizontal rule around the title

\pretitle{
	\vspace{-30pt} % Move the entire title section up
	\HorRule\vspace{10pt} % Horizontal rule before the title
	\fontsize{32}{36}\usefont{OT1}{phv}{b}{n}\selectfont % Helvetica
	\color{DarkRed} % Text colour for the title and author(s)
}

\posttitle{\par\vskip 15pt} % Whitespace under the title

\preauthor{} % Anything that will appear before \author is printed

\postauthor{ % Anything that will appear after \author is printed
	\vspace{10pt} % Space before the rule
	\par\HorRule % Horizontal rule after the title
	\vspace{20pt} % Space after the title section
}

%----------------------------------------------------------------------------------------
%	ABSTRACT
%----------------------------------------------------------------------------------------

\usepackage{lettrine} % Package to accentuate the first letter of the text (lettrine)
\usepackage{fix-cm}	% Fixes the height of the lettrine

\newcommand{\initial}[1]{ % Defines the command and style for the lettrine
	\lettrine[lines=3,findent=4pt,nindent=0pt]{% Lettrine takes up 3 lines, the text to the right of it is indented 4pt and further indenting of lines 2+ is stopped
		\color{DarkGoldenrod}% Lettrine colour
		{#1}% The letter
	}{}%
}

\usepackage{xstring} % Required for string manipulation

\newcommand{\lettrineabstract}[1]{
	\StrLeft{#1}{1}[\firstletter] % Capture the first letter of the abstract for the lettrine
	\initial{\firstletter}\textbf{\StrGobbleLeft{#1}{1}} % Print the abstract with the first letter as a lettrine and the rest in bold
}

%----------------------------------------------------------------------------------------
%	BIBLIOGRAPHY
%----------------------------------------------------------------------------------------

\usepackage[backend=biber,style=authoryear,natbib=true]{biblatex} % Use the bibtex backend with the authoryear citation style (which resembles APA)

\addbibresource{mybib.bib} % The filename of the bibliography

\usepackage[autostyle=true]{csquotes} % Required to generate language-dependent quotes in the bibliography


%----------------------------------------------------------------------------------------
%	Theorems and Definitions
%   https://tex.stackexchange.com/questions/45817/theorem-definition-lemma-problem-numbering
%----------------------------------------------------------------------------------------

\newtheoremstyle{break}%                % Name
  {}%                                     % Space above
  {}%                                     % Space below
  {}%                             % Body font
  {}%                                     % Indent amount
  {\itshape\bfseries}%                            % Theorem head font
  {}%                                    % Punctuation after theorem head
  {\newline}%                                    % Space after theorem head, ' ', or \newline
  {}%                                     % Theorem head spec (can be left empty, meaning `normal')

\newtheoremstyle{nte}%                % Name
  {}%                                     % Space above
  {}%                                     % Space below
  {}%                             % Body font
  {}%                                     % Indent amount
  {\itshape}%                            % Theorem head font
  {:}%                                    % Punctuation after theorem head
  { }%                                    % Space after theorem head, ' ', or \newline
  {}%                                     % Theorem head spec (can be left empty, meaning `normal')


\newtheoremstyle{trm}%                % Name
  {}%                                     % Space above
  {}%                                     % Space below
  {}%                             % Body font
  {}%                                     % Indent amount
  {\itshape}%                            % Theorem head font
  {:}%                                    % Punctuation after theorem head
  {\newline}%                                    % Space after theorem head, ' ', or \newline
  {}%                                     % Theorem head spec (can be left empty, meaning `normal')

\theoremstyle{break}

\renewenvironment{proof}{{\bf {Proof:} }}{\hfill $\Box$ \\} 
\newtheorem{theorem}{Theorem}[section] % reset theorem numbering for each chapter
\newtheorem{defn}[theorem]{Definition} % definition numbers are dependent on theorem numbers
\newtheorem{exmp}[theorem]{Example} % same for example numbers
\newtheorem{corollary}[theorem]{Corollary} % same for example numbers
%\newtheorem{proof}[theorem]{Proof} % 


\theoremstyle{nte}
\newtheorem*{note}{Note}
\newtheorem*{terminology}{Terminology}

\theoremstyle{trm}
%\newtheorem*{terminology}{Terminology}



%----------------------------------------------------------------------------------------
%	New definitions
%----------------------------------------------------------------------------------------
\usepackage{mathtools}
\usepackage{tensor}
\usepackage{tikz}
\usepackage{pgfplots}
\usepackage{hyperref}

\newcommand{\M}{\mathcal{M}}
\newcommand{\calL}{\mathcal{L}}
\newcommand{\calO}{\mathcal{O}}
\newcommand{\calA}{\mathcal{A}}
\newcommand{\Lie}{\mathcal{L}}

%\newcommand{\diff}{\mathrm{d}\!}
\DeclareMathOperator{\diff}{d\!}
\DeclareMathOperator{\difff}{d}
\DeclareMathOperator{\diag}{diag}
\newcommand{\eval}{\big|}
\newcommand{\linearto}{\xrightarrow[]{\sim}}
\DeclareMathOperator{\preim}{preim}
\DeclareDocumentCommand{\T}{o}
{%
    \IfNoValueTF{#1}{\operatorname{T}\!}{\operatorname{T}_{#1}\!}
}
\DeclareMathOperator{\Riem}{Riem}
%\DeclareMathOperator{\T}{T\!}
%\newcommand{\T}[1][]{\operatorename{T}_{#1} \!}

\makeatletter
\def\moverlay{\mathpalette\mov@rlay}
\def\mov@rlay#1#2{\leavevmode\vtop{%
   \baselineskip\z@skip \lineskiplimit-\maxdimen
   \ialign{\hfil$\m@th#1##$\hfil\cr#2\crcr}}}
\newcommand{\charfusion}[3][\mathord]{
    #1{\ifx#1\mathop\vphantom{#2}\fi
        \mathpalette\mov@rlay{#2\cr#3}
      }
    \ifx#1\mathop\expandafter\displaylimits\fi}
\makeatother

\newcommand{\cupdot}{\charfusion[\mathbin]{\cup}{\cdot}}
\newcommand{\bigcupdot}{\charfusion[\mathop]{\bigcup}{\cdot}}

%https://tex.stackexchange.com/a/229156
\newcommand{\subscript}[2]{$#1 _ #2$} 


\hypersetup{
    unicode=false,          % non-Latin characters in Acrobat’s bookmarks
    pdftoolbar=true,        % show Acrobat’s toolbar?
    pdfmenubar=true,        % show Acrobat’s menu?
    pdffitwindow=false,     % window fit to page when opened
    pdfstartview={FitH},    % fits the width of the page to the window
    pdftitle={},    % title
    pdfauthor={},     % author
    pdfsubject={},   % subject of the document
    pdfcreator={},   % creator of the document
    pdfproducer={}, % producer of the document
    pdfkeywords={}, % list of keywords
    pdfnewwindow=true,      % links in new PDF window
    colorlinks=false,       % false: boxed links; true: colored links
    linkcolor=red,          % color of internal links (change box color with linkbordercolor)
    citecolor=green,        % color of links to bibliography
    filecolor=magenta,      % color of file links
    urlcolor=cyan           % color of external links
}
%\usepackage{url}
%\pgfplotsset{compat=1.16}


%----------------------------------------------------------------------------------------
%	TIKZ and PGFPLOTS
%----------------------------------------------------------------------------------------
% https://tex.stackexchange.com/questions/382762/drawing-manifolds-in-tikz
\usepgfplotslibrary{patchplots}
\usetikzlibrary{patterns, positioning, arrows}
\usetikzlibrary{matrix}
\usetikzlibrary{backgrounds}
\pgfplotsset{compat=1.15}

\pgfplotsset{compat=1.7}
\pgfmathsetmacro\sprayRadius{.25pt}
\pgfmathsetmacro\sprayPeriod{.6cm}

\pgfdeclarepatternformonly{spray}{\pgfpoint{-\sprayRadius}{-\sprayRadius}}{\pgfpoint{1cm + \sprayRadius}{1cm + \sprayRadius}}{\pgfpoint{\sprayPeriod}{\sprayPeriod}}{
    \foreach \x/\y in {2/53,6/52,11/48,23/49,20/47,32/46,41/47,47/51,56/52,46/44,4/43,16/42,33/41,41/37,49/35,55/31,37/35,44/30,28/37,24/36,17/37,7/38,0/31,8/29,18/31,28/30,37/28,30/27,46/24,51/21,24/23,12/24,4/21,18/19,12/16,31/21,38/18,26/16,46/16,56/12,52/10,45/8,51/4,37/12,35/7,24/9,14/9,2/12,8/6,15/4,27/0,34/1,40/1} {
    \pgfpathcircle{\pgfpoint{(\x + random()) / 57 * \sprayPeriod}{\sprayPeriod - (\y + random()) / 55 * \sprayPeriod}}{\sprayRadius}
    }
\pgfusepath{fill}
}

\newcommand{\boxalign}[2][0.97\textwidth]{%
  \par\noindent\tikzstyle{mybox} = [draw=black,inner sep=6pt]
  \begin{center}\begin{tikzpicture}
   \node [mybox] (box){%
    \begin{minipage}{#1}{\vspace{-5mm}#2}\end{minipage}
   };
  \end{tikzpicture}\end{center}
}

%https://www.math.lsu.edu/~aperlis/publications/mathclap/perlis_mathclap_24Jun2003.pdf
\def\mathclap#1{\text{\hbox to 0pt{\hss$\mathsurround=0pt#1$\hss}}}
%----------------------------------------------------------------------------------------
%	SYMBOLS
%----------------------------------------------------------------------------------------
% 
%\usepackage{bbding}
%\usepackage{pifont}
%\usepackage{wasysym}
\usepackage{amsfonts}
\usepackage{cancel}
%\newcommand{\flower}{\SixFlowerRemovedOpenPetal}
\newcommand{\flower}{\star}


%----------------------------------------------------------------------------------------
%	EXAMPLE ENVIRONMENT
%   https://tex.stackexchange.com/questions/21227/example-environment/21241
%----------------------------------------------------------------------------------------
\usepackage[most]{tcolorbox}
\newcounter{examples}

\def\exampletext{Example} % If English

\NewDocumentEnvironment{example}{ O{} }
{
\colorlet{colexam}{red!55!black} % Global example color
\newtcolorbox[use counter=examples]{examplebox}{%
    % Example Frame Start
    empty,% Empty previously set parameters
    title={\exampletext: #1},% use \thetcbcounter to access the testexample counter text
    % Attaching a box requires an overlay
    attach boxed title to top left,
       % Ensures proper line breaking in longer titles
       minipage boxed title,
    % (boxed title style requires an overlay)
    boxed title style={empty,size=minimal,toprule=0pt,top=4pt,left=3mm,overlay={}},
    coltitle=colexam,fonttitle=\bfseries,
    before=\par\medskip\noindent,parbox=false,boxsep=0pt,left=3mm,right=0mm,top=2pt,breakable,pad at break=0mm,
       before upper=\csname @totalleftmargin\endcsname0pt, % Use instead of parbox=true. This ensures parskip is inherited by box.
    % Handles box when it exists on one page only
    overlay unbroken={\draw[colexam,line width=.5pt] ([xshift=-0pt]title.north west) -- ([xshift=-0pt]frame.south west); },
    % Handles multipage box: first page
    overlay first={\draw[colexam,line width=.5pt] ([xshift=-0pt]title.north west) -- ([xshift=-0pt]frame.south west); },
    % Handles multipage box: middle page
    overlay middle={\draw[colexam,line width=.5pt] ([xshift=-0pt]frame.north west) -- ([xshift=-0pt]frame.south west); },
    % Handles multipage box: last page
    overlay last={\draw[colexam,line width=.5pt] ([xshift=-0pt]frame.north west) -- ([xshift=-0pt]frame.south west); },%
    }
\begin{examplebox}}
{\end{examplebox}\endlist}
 % Specifies the document structure and loads requires packages

%----------------------------------------------------------------------------------------
%	ARTICLE INFORMATION
%----------------------------------------------------------------------------------------

%\title{To those physics students who asked why $q$ and $\dot{q}$ are independent in Lagrangian Mechanics}
\title{Mathematical Notes on Manifolds in Physics}

\author{%
	\authorstyle{Niklas Zorbach\textsuperscript{1} and Marco Knipfer\textsuperscript{1, 2}} % Authors 
	\newline\newline % Space before institutions
	\textsuperscript{1}\institution{Institute for Theoretical Physics, Goethe-University Frankfurt, Germany}\\ 
	\textsuperscript{2}\institution{Institute for Physics and Astronomy, The University of Alabama, USA}\\ 
}

% Example of a one line author/institution relationship
%\author{\newauthor{John Marston} \newinstitution{Universidad Nacional Autónoma de México, Mexico City, Mexico}}

\date{Version: \today} 

%----------------------------------------------------------------------------------------

\begin{document}

\maketitle % Print the title

\thispagestyle{firstpage} % Apply the page style for the first page (no headers and footers)

%----------------------------------------------------------------------------------------
%	ABSTRACT
%----------------------------------------------------------------------------------------

%\lettrineabstract makes first letter in abstract huge and nice, but does not allow equations in the whole {}.
\lettrineabstract{We are writing these notes in order to learn Riemannian Geometry and better understand
Lagrangian Mechanics and General Relativity. As a physicist one usually learns all of this in a rather practical way without
understanding the basic mathematical concepts. For example a physicist usually does not learn that the
Lagrangian lives on the tangent bundle, because one implicitly always
identifies some spaces (here the space and the tangent space), which is possible on a flat manifold.}

%----------------------------------------------------------------------------------------
%	ARTICLE CONTENTS
%----------------------------------------------------------------------------------------

\section{Manifolds}
\subsection{Topology}
This chapter is basically taken from \citep{Schuller15} with our remarks to it.

We start with a set $M$ which is supposed to be the space where physics happens.
The weakest structure we need in order to talk about continuity (of curves or fields)
is called a topology.

\begin{defn}[Power set $\mathcal{P}$]
    The set of all subsets of $M$.
\end{defn}

\begin{defn}[Topology]
    A Topology $\mathcal{O}$ is a subset $\mathcal{O} \subseteq \mathcal{P}(M)$ satisfying:
    \begin{enumerate}
        \item $\emptyset \in \mathcal{O}$, $M\in\mathcal{O}$,
        \item $U\in\mathcal{O},\quad V\in\mathcal{O}\Rightarrow U\cap V\in\mathcal{O}$
        \item $U_\alpha\in\mathcal{O},\quad \alpha\in A \Rightarrow
            \left( \bigcup\limits_{\alpha\in A} U_\alpha \right) \in \mathcal{O}$
    \end{enumerate}
\end{defn}
Every set has the \textit{chaotic topology}
\begin{equation}
    \mathcal{O}_\text{chaotic} := \left\{ \emptyset, M \right\}\,,
\end{equation}
and the \textit{discrete topology}
\begin{equation}
    \mathcal{O}_\text{discrete} := \mathcal{P}(M)\,,
\end{equation}
which are both useless.

The special case $M = \mathbb{R}^d = \mathbb{R} \times \cdots \times \mathbb{R}$ has a standard topology
for which we need the definition of a soft ball.
\begin{defn}[Soft Ball in $\mathbb{R}^d$]
   \begin{equation}
       B_r(p) := \left\{ (q_1,\cdots,q_d)| \sum_{i=1}^{d}(p_i-p_i) < r \right\}\,,
   \end{equation} 
   with $r\in \mathbb{R}^+$, $p\in\mathbb{R}^d$.
   Note: This does not need a norm or vector space structure on $\mathbb{R}^d$.
\end{defn}
\begin{defn}[$\mathcal{O}_\text{standard}$ on $\mathbb{R}^d$]
    \begin{equation}
        U \in \mathcal{O}_\text{standard} :\Leftrightarrow \forall p\in U:
        \exists r\in\mathbb{R}^+ : B_r(p) \subseteq U
    \end{equation}
\end{defn}

\begin{figure}[h]
\centering
\begin{tikzpicture}
    %\draw[smooth cycle,tension=3.0] plot coordinates{(1,0) (1,1) (2,2) (3,3) (6,0)} node [label=right:$M$];

    % Manifold
    \draw[smooth cycle, tension=0.4] plot coordinates{(2,4) (-1.5,0) (3.5,-2) (6,1)} node [label=$M$, anchor = west] {};
    \draw[smooth cycle, tension=1, dashed] plot coordinates{(1,1) (-1,0) (1,-1) (2.5,0.5)} node [label=$U$] {};
    \draw[smooth cycle, tension=0.4, dashed] plot coordinates{(1, 2) (1,3) (3,3) (3,2)} node [label=$V$] {};
    \draw[smooth cycle, tension=0.4] plot coordinates{(1, 2) (1,3) (3,3) (3,2)} node [label=$V$] {};
    \draw[dashed] (-0.5,0) circle (.4cm) node [label = $B_r$, anchor = north west] {};
    \draw[fill] (-0.5,0) circle (.05cm) ;
    \draw (1.5,1.95) circle (.2cm) node [label = $B_r$, anchor = north west] {};
    \draw[fill] (1.5,1.95) circle (.05cm) ;

    %\draw[help lines] (-3,-6) grid (8,6);

\end{tikzpicture}
\caption{The set $U$ is in the standard topology, $V$ not.}
\end{figure}

Some terminology: Let $M$ be a set with a topology $\mathcal{O} =:$ set of open sets.
We call $(M, \mathcal{O}$ a \textit{topological space} and:
    \begin{itemize}
        \item $U \in \mathcal{O} \Leftrightarrow: \text{call } U\subseteq M$ an \textit{open set}
        \item $M\backslash A \in \mathcal{O} \Leftrightarrow: \text{call } U\subseteq M$  a \textit{closed set}
    \end{itemize}
\begin{note}
    The empty set is open and closed. If a set is open we cannot directly follow that it is not closed or vise versa. For $M = \left\{ 1, 2 \right\}$ and $\mathcal{O}_M = \left\{ \emptyset, \left\{ 1 \right\}, \left\{ 2 \right\}, \left\{ 1,2 \right\} \right\}$ the set $\left\{ 2 \right\}$ is open and closed.
\end{note}

\subsection{Continuous Maps}
A map
\begin{equation}
    f: M \to N\,,
\end{equation}
takes every point from the domain M (a set) to the target N (a set).
If one point $p\in N$ is not reached, the map is not \textit{surjective}.
If a point is hit twice, the map is not \textit{injective}.
A map that is injective and surjective is called \textit{surjective}.

\begin{defn}[Preimage]
    \begin{align}
        f&: M \to N \supseteq V \nonumber \\
        \text{preim}_f(V) &:= \left\{ m\in M~|~f(f) \in V \right\}
    \end{align}
\end{defn}

\begin{defn}[Continuity]
    $(M, \mathcal{O}_M)$ and $(N, \mathcal{O}_N)$ topological spaces.
    Then a map $f: M \to N$ is called \textit{continuous with respect to $\mathcal{O}_M$ and
    $\mathcal{O}_N$} if
    \begin{equation}
    \forall V\in \mathcal{O}_N: \text{preim}_f(V) \in \mathcal{O}_M\,.
    \end{equation}
    \textit{``A map is open iff the preimages of all open sets are open sets.''}
\end{defn}

\begin{note}
    If a map is not surjective there are sets with preimage $\emptyset$, thus we need to have
    $\emptyset$ in $\mathcal{O}$, otherwise only surjective maps could be continuous.
\end{note}
\begin{note}
    The inverse of a continuous function does not need to be continuous.
\end{note}

\begin{defn}[Composition of maps]
    For $f$ and $g$
    \begin{equation*}
        f: M \to N, \quad g: N \to P\,,
    \end{equation*}
    we define the \textit{composition} as
    \begin{align}
        g \circ f&: M \to P \\
        m &\mapsto (g\circ f)(m) := g(f(m)) \nonumber
    \end{align}
\end{defn}

\begin{theorem}[Composition of continuos maps]
    For $f$, $g$ continuos also $g\circ f$ is continuous (if spaces match).
\end{theorem}


\begin{defn}[Subset topology, Inherited topology]
    A set $M$ with topology $\mathcal{O}_M$. Given any subset $S \subseteq M$ we
    can construct the inherited topology $\mathcal{O}\eval_S \subseteq \mathcal{P}(S)$
    \begin{equation}
        \mathcal{O}\eval_S := \left\{ U\cap S~|~U\in \mathcal{O}M \right\}\,.
    \end{equation}
\end{defn}

\begin{note}
    For $S \subseteq M$, if $f$ is continuous then $f\eval_{S}$ is also
    continuous if $\mathcal{O}\eval_S$ is chosen. This is for example important
    if you are on a trajectory $\gamma$ through $\mathbb{R}^n$ and measure the temperature
    $T\eval_\gamma$.
\end{note}

\begin{defn}[Topological manifold]
    A topological space $(\M, \mathcal{O})$ is called a \textit{d-dimensional topological manifold}
    if
    \begin{equation}
        \forall p \in \M: \exists U \in \mathcal{O},~p\in U: \exists x: U\to x(U)\subseteq \mathbb{R}^{d}
        \,,
    \end{equation}
    with the following properties (wrt. $\mathcal O_\text{std}$ on $\mathbb{R}^{d}$):
    \begin{enumerate}
        \item $x$ intervitble: $x^{-1} : x(U) \to U$,
        \item $x$ continuous,
        \item $x^{-1}$ continuous\,.
    \end{enumerate}
    ``Invertible, in both directions continuous map to $\mathbb{R}^{n}$.''
\end{defn}

\begin{note}
    Thus in the above definition $x(U)$ is also open (from the definition of continuity).
\end{note}

\begin{terminology}
    \begin{itemize}
        \item $(U,x)$ is a \textit{chart} of $\M, \mathcal{O}$,
        \item $\mathcal{A} = \left\{ (U_{(\alpha)}, x_{(\alpha)} | \alpha \in A \right\}$ is an \textit{atlas} of $(\M, \mathcal{O})$ if $\bigcup\limits_{\alpha\in A} U_{(\alpha)}$ covers the whole manifold $\M$,
        \item $x: U \to x(U) \subseteq \mathbb{R}^{d}$ is a \textit{chart map} 
            $x(p) = (x^1(p), \dots, x^d(p))$, where the \textit{component maps} $x^i: U\to\mathbb{R}$ are called \textit{coordinate maps},
        \item $p \in U$, then $x^1(p)$ is the first coordinate of the point p wrt.\ the chosen chart $(U, x)$.
    \end{itemize}
\end{terminology}
\begin{note}
    The choice of the chart (choice of coordinates) has nothing to do with the physics.
    Physics is chart independent. $\M$ is ``the real world''.
\end{note}

\subsection{Chart Transition Maps}

\begin{figure*}[tbh]
    \centering
    \begin{tikzpicture}
    % https://tex.stackexchange.com/questions/382762/drawing-manifolds-in-tikz
    % Functions i
        \path[->] (0.8, 0) edge [bend right] node[left, xshift=-2mm] {$x$} (-1, -2.9);
        \draw[white,fill=white] (0.06,-0.57) circle (.15cm);
        \path[->] (-0.7, -3.05) edge [bend right] node [right, yshift=-3mm] {$x^{-1}$} (1.093, -0.11);
        \draw[white, fill=white] (0.95,-1.2) circle (.15cm);

    % Functions j
        \path[->] (5.8, -2.8) edge [bend left] node[midway, xshift=-5mm, yshift=-3mm] {$y^{-1}$} (3.8, -0.35);
        \draw[white, fill=white] (4,-1.1) circle (.15cm);
        \path[->] (4.2, 0) edge [bend left] node[right, xshift=2mm] {$y$} (6.2, -2.8);
        \draw[white, fill=white] (4.54,-0.12) circle (.15cm);

    % Manifold
        \draw[smooth cycle, tension=0.4, fill=white, pattern color=brown, pattern=spray, opacity=0.7] plot coordinates{(2,2) (-0.5,0) (3,-2) (5,1)} node at (3,2.3) {$\M$};

    % Help lines
    %\draw[help lines] (-3,-6) grid (8,6);

    % Subsets
        \draw[smooth cycle, dashed, pattern color=orange, pattern=crosshatch dots, fill opacity = 0.4] 
        plot coordinates {(1,0) (1.5, 1.2) (2.5,1.3) (2.6, 0.4)} 
        node [label={[label distance=-0.3cm, xshift=-2cm, fill=white]:$U$}] {};
        \draw[smooth cycle, dashed, pattern color=blue, pattern=crosshatch dots, fill opacity = 0.4] 
        plot coordinates {(4, 0) (3.7, 0.8) (3.0, 1.2) (2.5, 1.2) (2.2, 0.8) (2.3, 0.5) (2.6, 0.3) (3.5, 0.0)} 
        node [label={[label distance=-0.8cm, xshift=.75cm, yshift=1cm, fill=white]:$V$}] {};

    % First Axis
        \draw[thick, ->] (-3,-5) -- (0, -5) node [label=above:$x(U)$] {};
        \draw[thick, ->] (-3,-5) -- (-3, -2) node [label=right:$\mathbb{R}^d$] {};

    % Arrow from i to j
        \draw[->] (0, -3.85) -- node[midway, above]{$y\circ x^{-1}$} (4.5, -3.85);

    % Second Axis
        \draw[thick, ->] (5, -5) -- (8, -5) node [label=above:$x(V)$] {};
        \draw[thick, ->] (5, -5) -- (5, -2) node [label=right:$\mathbb{R}^d$] {};

    % Sets in R^m
        \draw[white, dashed, pattern color=blue, pattern=crosshatch dots, fill opacity = 0.4] (-0.67, -3.06) -- +(180:0.8) arc (180:270:0.8);
        \fill[even odd rule, white] [smooth cycle] plot coordinates{(-2, -4.5) (-2, -3.2) (-0.8, -3.2) (-0.8, -4.5)} (-0.67, -3.06) -- +(180:0.8) arc (180:270:0.8);
        \draw[smooth cycle, dashed, pattern color = orange, pattern = crosshatch dots, fill opacity = 0.4] plot coordinates{(-2, -4.5) (-2, -3.2) (-0.8, -3.2) (-0.8, -4.5)};
        \draw[dashed] (-1.45, -3.06) arc (180:270:0.8);

        \draw[white, dashed, pattern color=orange, pattern=crosshatch dots, fill opacity = 0.4] (5.7, -3.06) -- +(-90:0.8) arc (-90:0:0.8);
        \fill[even odd rule, white] [smooth cycle] plot coordinates{(7, -4.5) (7, -3.2) (5.8, -3.2) (5.8, -4.5)} (5.7, -3.06) -- +(-90:0.8) arc (-90:0:0.8);
        \draw[smooth cycle, dashed, pattern color = blue, pattern = crosshatch dots, fill opacity = 0.4] plot coordinates{(7, -4.5) (7, -3.2) (5.8, -3.2) (5.8, -4.5)};
        \draw[dashed] (5.69, -3.85) arc (-90:0:0.8);

    \end{tikzpicture}

    \caption{Visualization of chart transition maps. ``How to glue together the charts of an atlas.'' Plot modified from~\citep{texstackexchange:manifolds}}
    \label{fig:transitionmaps}
\end{figure*}
Given $(U, x)$ and $(V, y)$ charts, on $U\cup V$ one can transition from
one chart to the other by (see figure~\ref{fig:transitionmaps})
\begin{equation}
    y\circ x^{-1}: \mathbb{R}^{d} \supseteq x(U\cap V) \to y(U\cap V) \subseteq \mathbb{R}^{d}\,,
\end{equation}
which is called the \textit{chart transition map}.
\begin{note}
    As a physicist one talks about a ``change in coordinates''.
\end{note}

\subsection{Manifold Philosophy}

The idea is to define properties of some object in the real world $\M$ by at a chart-representative
of it. For example the continuity of a curve $\gamma: [0,1]\to\M$ can be judged by looking at
$x\circ\gamma: [0,1]\to\mathbb{R}^{d}$, because $x$ is invertible and in both directions continuous
and the composition of two continuous maps is also continuous.
\begin{note}
    One needs to make sure that the property of the object on $\M$ does not depend on the
    map $x$ or $y$. For continuity this is the case, since 
    $y\circ\gamma = (y\circ x^{-1}) \circ x\circ \gamma$ and the chart transition map
    $y\circ x^{-1}$ is also continuous.
\end{note}

Other properties like ``differentiability'' are not even defined on $\M$ a priori,
so one can only talk about the chart representative. Here the definition that $\gamma$
is differentiable iff $x\circ\gamma: [0,1]\to\mathbb{R}^{d}$ is differentiable has the
problem that $x$ and $y$ only need to be continuous and so the chart transition map
$y\circ x^{-1}$ does not need to be differentiable unless one restricts oneself to only
differentiable charts.

\section{Vector Spaces}
\subsection{Vectors and Linear Maps}
In order to understand the tangent space we need to understand vector spaces.
\begin{defn}[Vector space $(V,+,\cdot)$]
    \label{def:vectorspace}
    A vector space $(V,+,\cdot)$ is a set $V$ with
    \begin{itemize}
        \item an ``addition'' $+: V\times V \ to V$,
        \item an ``S-multiplication'' $\cdot:\mathbb{R}\times V \ to V$
    \end{itemize}
    and the properties CANI ADDU:\\ $\forall v, w, u \in V, \lambda, \mu \in \mathbb{R}$
    \begin{description}
        \item[C$^+$:] $v + w = w + v$\,,
        \item[A$^+$:] $(u + v) + w = u + (v + w)$\,,
        \item[N$^+$:] $\exists 0 \in V: \forall v \in V: v + 0 = v$\,,
        \item[I$^+$:] $\forall v \in V: \exists(-v)\in V: v + (-v) = 0$\,,
        \item[A:] $\lambda \cdot ( \mu \cdot v) = (\lambda \cdot \mu) \cdot v$\,,
        \item[D:] $(\lambda + \mu)\cdot v = \lambda \cdot v + \mu \cdot v$\,,
        \item[D:] $\lambda \cdot v + \lambda \cdot w = \lambda \cdot (v + w)$\,,
        \item[U:] $ 1 \cdot v = v$\,.
    \end{description}
    An element of a vector space is called a \textit{vector}.
\end{defn}
\begin{note}
The addition $+$ in definition~\ref{def:vectorspace} sometimes is between vectors and sometimes between
scalars. It is important to know the difference.
\end{note}

\begin{defn}[Linear maps]
    (Structure respecting maps between vector spaces)\newline
    $(V, +_V, \cdot_V)$ and $(W, +_W, \cdot_W)$ vector spaces.
    A map
    \begin{equation}
        \phi: V \to W
    \end{equation}
    is called \textit{linear} if $\forall v, \tilde{v} \in V, \forall \lambda \in \mathbb{R}$
    \begin{enumerate}
        \item $\phi(v +_V + \tilde{v}) = \phi(v) +_W \phi(\tilde{v})$
        \item $\phi(\lambda\cdot_V v) = \lambda \cdot_W \phi(v)$
    \end{enumerate}
    We write:
    \begin{equation}
        \phi: V \to W\, \text{linear}\, \Leftrightarrow: \phi: V \linearto W\,.
    \end{equation}
\end{defn}

\begin{theorem}[Transitivity of linearity of maps]
    $V, W, U$ vector spaces, $\psi: V\linearto W$, $\phi: W\linearto U$
    then $\phi \circ \psi$ is also linear: $\phi\circ\psi: V \linearto U$.
\end{theorem}

\begin{defn}[Homomorphisms Hom$(V,W)$]
    \begin{equation}
        \text{Hom}(V,W):= \left\{ \phi: V\linearto W \right\}\,.
    \end{equation}
\end{defn}
\begin{note}
    Hom$(V,W)$ can be made into a vector space by defining an addition and a multiplication
    \begin{itemize}
        \item $(\phi + \psi)(v) := \phi(v) + \psi(v)$\,,
        \item $(\lambda \psi)(v) := \lambda (\psi(v))$\,.
    \end{itemize}
\end{note}

\begin{defn}[Dual vector space $V^\star$]
    \begin{equation}
        V^\star := \left\{ \phi: V \linearto \mathbb{R} \right\} = \text{Hom}(V, \mathbb{R})\,.
    \end{equation}
    The vector space $(V^\star, +, \cdot)$ is the \textit{dual vector space} to $V$.
    $\phi \in V^\star$ is informally called a \textit{covector}.
\end{defn}

\begin{defn}[$(r, s)$ - Tensors]
    $(V, +, \cdot)$ vector space, $r, s\in \mathbb{N}_0$.
    An $(r, s)$-tensor $T$ over $V$ is a multi-linar map
    \begin{equation}
        T: \overbrace{V^\star \times \cdots \times V^\star}^r \times
        \overbrace{V \times \cdots \times V}^s \xrightarrow{\begin{smallmatrix} \sim\\ \vdots \\ \sim \end{smallmatrix}} \mathbb{R}
    \end{equation}
\end{defn}
\begin{theorem}[Covector is (0,1)-tensor]
    \begin{equation}
        \phi\in V^\star \Leftrightarrow \phi: V\linearto \mathbb{R} \Leftrightarrow \phi\,(0,1)\,\text{tensor}\,.
    \end{equation}
\end{theorem}
\begin{theorem}[Vector is (1,0)-tensor]
    If $\text{dim}V < \infty$
    \begin{equation}
        v \in V = (V^\star)^\star \Leftrightarrow 
        v: V^\star \linearto \mathbb{R} \Leftrightarrow
        v\,\text{is}\, (1,0)-\text{tensor}\,.
    \end{equation}
\end{theorem}

\subsection{Bases}
\begin{defn}[Hamel-basis]
    $(V, +, \cdot))$ vector space. A subset $B \subset V$ is called a Hamel-basis if
    \begin{equation}
        \forall v\in V\, \exists!\,\text{finite}\,F=\left\{ f_1,\ldots,f_n \right\}\subset B: \exists! \underbrace{v^1, \cdots, v^n}_{\in \mathbb{R}}\,,
    \end{equation}
    such that
    \begin{equation}
        v = v^1 f_1 + \cdots + v^n f_n\,.
    \end{equation}
    (and all $f_i$ linearly independent).
\end{defn}
\begin{defn}[Dimension of a vector space]
    If a basis $B$ with $d<\infty$ many elements, then we call $d =: \text{dim}V$.
\end{defn}

If we have chosen a basis $\left\{ e_1, \ldots, e_n \right\}$ of $(V, +, \cdot)$ then
$(v^1, \ldots, v^n)$ are called the \textit{components of $V$} w.r.t.\ the chosen basis
if
\begin{equation}
    v = v^1 e_1 + \cdots v^n e_n\,.
\end{equation}

\begin{defn}[Dual basis]
    Choose basis $\left\{ e_1, \cdots, e_n \right\}$ for $V$. 
    The basis $\left\{ \epsilon^1, \cdots, \epsilon^n \right\}$
    for $V^\star$ can be chosen that
    \begin{equation}
        \epsilon^a(e_b) = \delta^a_b\quad \forall a,b = 1,\ldots, n\,.
    \end{equation}
    $\left\{ \epsilon^1,\dots,\epsilon^n \right\}$ is then called \textit{the dual basis} of the dual
    space.
\end{defn}

\subsection{Components of a tensor}
$T$ an $(r,s)$-tensor. Then the real numbers
\begin{equation}
    T\indices{^{i_1\ldots i_r} _{j_1\ldots j_s}} = T\left( \epsilon^{i_1},\ldots, \epsilon^{i_r}, e_{j_1}, \ldots e_{j_s}\right)
\end{equation}
are the components of $T$ with respect to the chosen basis.
From the components and the basis one can reconstruct the entire tensor:
Example for (1, 1)-tensor:
\begin{equation}
    T(\varphi, v) = T\indices{^i _j} \varphi_i v^j\, \label{eq:tensorcomponentsexample}
\end{equation}
where $\varphi_i$ are the components of $\varphi\in V^\star$ and $v^j$ the components of $v\in V$ with respect to
the chosen basis. In equation~(\ref{eq:tensorcomponentsexample}) the \textit{Einstein summation convention} is used, \textit{i.e.}\ an index that appears up and down in an expression is summed over.

\begin{note}
    The Einstein summation convention is only useful because we are working with \textit{linear maps},
    otherwise the expression
    \begin{equation}
        \varphi\left(\sum_i v^i e_i\right) = \sum_i \varphi(v^i e_i)\,, 
    \end{equation}
    would not hold and with the summation index we would not know where the sum sign goes.
\end{note}

\section{Differentiable Manifolds}
So far we only had topological manifolds. We also want to be able to talk about the velocity of
curves. The problem is that the notion of a topological manifold is not enough to define
differentiability of curves. In this section we will find out what additional structure we need
to be able to talk about the differentiability of
\begin{itemize}
    \item curves: $\mathbb{R}\to \M$
    \item functions: $\M\to \mathbb{R}$
    \item maps: $\M\to\mathcal{N}$
\end{itemize}
Strategy: Choose a chart $(U, x)$ and consider portion of the curve in the domain of the chart:
$\gamma: \mathbb{R}\to U$ (see figure~\ref{fig:curvechart}). Since $x\circ\gamma: \mathbb{R}\to\mathbb{R}^d$
we can try to ``lift'' the notion of differentiability of a curve on $\mathbb{R}^d$ to that of
a curve on $\M$. The problem is to make this independent of the chart.
\begin{figure}
    \centering
    \begin{tikzpicture}
    \matrix (m) [matrix of math nodes,row sep=3em,column sep=4em,minimum width=2em]
    {
        ~                 & y(U\cup V)\subseteq \mathbb{R}^d\\
        \gamma:\mathbb{R} & U\cup V \neq \emptyset \\
        ~                 & x(U\cup V)\subseteq \mathbb{R}^d \\
    };
    \path[-stealth]
        (m-2-1) edge node [above] {$\gamma$} (m-2-2)
        (m-2-1) edge node [below, rotate = -25] {$x\circ\gamma$} (m-3-2)
        (m-2-1) edge node [above, rotate = +25] {$y\circ\gamma$} (m-1-2)
        (m-3-2) edge[bend right = 60] node [right, rotate = 90, xshift = -0.5cm, yshift = -0.2cm] {$y\circ x^{-1}$} (m-1-2)
        (m-2-2) edge node [right] {$x$} (m-3-2)
        (m-2-2) edge node [right] {$y$} (m-1-2);
    \end{tikzpicture}
    \caption{Curve $\gamma$ in chart.}
    \label{fig:curvechart}
\end{figure}
\begin{equation}
    y\circ\gamma = \underbrace{(y\circ x^{-1})}_{\mathbb{R}^d\to\mathbb{R}^d}\circ \underbrace{(x\circ\gamma)}_{\substack{\mathbb{R}\to\mathbb{R}^d\\ \text{differentiable}}}\,,
\end{equation}
but we only know that the \textit{chart transition map} $y\circ x^{-1}$ is continuous (because of the definition of a top. Manifold). Thus it is not guaranteed that
$y\circ \gamma$ is continuons, not differentiable. Reminder: The composition of continuous maps is continuous,
same for differentiable. 
The above definition of differentiability of $\gamma$ by
checking the differentiability of $x\circ\gamma$ with some chart $x$ is not independent of the chart.

\begin{defn}[$\flower$ - compatibility of charts]
    Two charts $(U, x)$ and $(V, y)$ of a topological manifold are called $\flower$-compatible if
    either
    \begin{enumerate}
        \item $U\cup V = \emptyset$ or
        \item $U\cup V \neq \emptyset$ and the chart transition maps
            \begin{align*}
                y\circ x^{-1}&:\mathbb{R}^d\supseteq x(U\cup V) \to y(U\cup V) \subseteq \mathbb{R}^d \\
                x\circ y^{-1}&: \mathbb{R}^d\supseteq y(U\cup V) \to x(U\cup V) \subseteq \mathbb{R}^d
            \end{align*}
            have the $\flower$-property in the $\mathbb{R}^d$-sense.
    \end{enumerate}
\end{defn}

\begin{defn}[$\flower$-compatible atlas]
    An atlas $\mathcal{A}_\flower$ is a $\flower$-compatible atlas if any two charts in
    $\mathcal{A}_\flower$ are $\flower$-compatible.
\end{defn}

\begin{defn}[$\flower$-manifold]
    A $\flower$-manifold is a triple $(\underbrace{\M, \mathcal{O}}_{\text{top.\ manif.}}, 
    \underbrace{\mathcal{A}_\flower}_{\in \mathcal{A}_\text{max}}$).
\end{defn}

\begin{tabular}{ll}
    $\flower$ & $\flower$ property in $\mathbb{R}^d$-sense \\
    \hline\hline
    $C^0$     & $C^0(\mathbb{R}^d \to \mathbb{R}^d)$ continuous maps\\
    $C^1$     & $C^1(\mathbb{R}^d \to \mathbb{R}^d)$ differentiable and result is cont.\\
    $C^k$     & $C^k(\mathbb{R}^d \to \mathbb{R}^d)$ k times diffble and result is cont.\\
    $D^k$     & $D^k(\mathbb{R}^d \to \mathbb{R}^d)$ k times differentiable\\
    $C^\infty$     & $C^\infty(\mathbb{R}^d \to \mathbb{R}^d)$ smooth functions\\
    $C^\omega$  & $\exists$ multidim. Taylor expansion, $C^\omega \subset C^\infty$
\end{tabular}
\begin{note}
    The more fancy properties one wants for the objects on the manifold, the more restrictive one
    has to be for the atlas.
\end{note}
\begin{theorem}[$C^1\to C^\infty$]
    Any $C^{k \leq 1}$-manifold atlas $\mathcal{A}_{C^{k\leq 1}}$ of a topological manifold contains
    a $C^\infty$-atlas.
\end{theorem}
Thus we may without loss of generality always consider $C^\infty$-manifolds. 
``smooth'' manifolds, unless we wish to define Taylor expandibility or complex differentiability, \ldots.
\begin{defn}[Smooth manifold]
    $(\M, \mathcal{O}, \mathcal{A})$, where $(\mathcal{M, O})$ is a topological manifold and $\mathcal{A}$ is a $C^\infty$-atlas.
\end{defn}

\section{Diffeomorphisms}
\begin{equation*}
    M \xrightarrow{\phantom{asdf}\phi\phantom{asdf}} N
\end{equation*}
$M$, $N$ naked sets, then the structure-preserving maps are bijections (invertionable maps).
\begin{defn}[Set-theoretically isomorphic]
    Two sets $M$, $N$ are said to be \textit{set-thoretically isomorphic} $M\cong_\text{st} N$ if
    $\exists$ a bijection $\phi: M\to N$ between them.
\end{defn}
\begin{note}
    Then they are ``of the same size''. Examples: $\mathbb{N}\cong_\text{st}\mathbb{Z}$, $\mathbb{N}\cong_\text{st} \mathbb{Q}$,
    $\mathbb{N}\not\cong_\text{st} \mathbb{R}$
\end{note}
Now $(\M, \mathcal{O}_\M)$ and $(\mathcal{N}, \mathcal{O_N})$.
\begin{equation*}
    \M \xrightarrow{\phantom{asdf}\phi\phantom{asdf}} \mathcal{N}
\end{equation*}
\begin{defn}[Topologically isomorphic (homeomorphic)]
    $(\M, \mathcal{O}_\M) \cong_\text{top} (\mathcal{N}, \mathcal{O_N})$ topologically isomorphic = ``homeomorphic'' if $\exists \phi: \M\to\mathcal{N}$ and $\phi, \phi^{-1}$ are \textit{continuous}.
\end{defn}
\begin{note}
    Continuity is the important property here.
    This is a stronger notion. If two spaces are hoemomorphic then they are also set-theoretically isomorphic.
\end{note}
\begin{defn}[Isomorphic vector spaces]
    $(V, +_V, \cdot_V) \cong_\text{vec} (W, +_W, \cdot_W)$ if $\exists$ a bijection $\phi:V\to W$ that is
    linear in both directions.
\end{defn}
\begin{defn}[diffeomorphic]
    Two $C^\infty$ manifolds $(\M, \mathcal{O_M}, \mathcal{A_M})$ and $(\mathcal{N, O_N, A_N})$
    are said to be \textit{diffeomorphic} if $\exists$ a bijeciton $\phi: \M \to \mathcal{N}$ such that
    $\phi, \phi^{-1}$ are both $C^\infty$-maps, where by $C^\infty$ we mean that $y\circ \phi \circ x^{-1}$
    is in $C^\infty$ in the $\mathbb{R}^d$-sense, see figure~\ref{fig:diffeo-commdiag}.
    \label{def:diffeomorphic}
\end{defn}

\begin{figure}[tbh]
    \centering
    \begin{tikzpicture}
    \matrix (m) [matrix of math nodes,row sep=3em,column sep=4em,minimum width=2em]
    {
        \mathbb{R}^d & \mathbb{R}^l \\
        \M \supseteq U & V \subseteq N \\
        \mathbb{R}^d & \mathbb{R}^l \\
    };
    \path[-stealth]
        (m-1-1) edge node [below] {$\tilde{y} \circ \phi \circ \tilde{x}^{-1}$} (m-1-2)
        (m-2-1) edge node [left] {$\tilde{x}$} (m-1-1)
        (m-2-2) edge node [right] {$\tilde{y}$} (m-1-2)
        (m-2-1) edge node [above] {$\phi$} (m-2-2)
        (m-2-1) edge node [left] {$x$} (m-3-1)
        (m-2-2) edge node [right] {$y$} (m-3-2)
        (m-3-1) edge node [above] {$y \circ \phi \circ x^{-1}$} (m-3-2)
        (m-3-1) edge[bend left = 60] node [left, rotate = 90, xshift = 0.5cm, yshift = +0.2cm] {$C^\infty$} (m-1-1)
        (m-3-2) edge[bend right = 60] node [right, rotate = 90, xshift = -0.4cm, yshift = -0.25cm] {$C^\infty$} (m-1-2);
    \end{tikzpicture}
    \caption{In the definition of diffeomorphic~\ref{def:diffeomorphic} $\phi, \phi^{-1}$ have to be
    $C^\infty$, which is defined such that $y\circ\phi\circ x^{-1}$ has to be $C^\infty$ in the $\mathbb{R}^d$-sense, which is chart-independent here.}
    \label{fig:diffeo-commdiag}
\end{figure}
\begin{note}
    Since we started with $C^\infty$-manifolds, the chart transition maps are $C^\infty$ and thus
    the notion of differentiability in the definition~\ref{def:diffeomorphic} is independent of the
    choice of charts, \textit{i.e.} $\tilde{y}\circ\phi\tilde{x}^{-1}$ is also $C^\infty$, see figure~\ref{fig:diffeo-commdiag}.
\end{note}
\begin{theorem}
    \# = number of $C^\infty$-manifolds one can make of a given $C^\infty$-manifold (if any) --- up to diffeomorphisms ---.\newline
    \begin{center}
    \begin{tabular}{ll}
        $\text{dim}M$  & \# \\
        \hline
        \begin{tabular}{l} 1 \\ 2 \\ 3 \end{tabular} & $\left.\begin{tabular}{l} 1 \\ 2 \\ 3 \end{tabular}\right\}$ Moise-Radon theorem  \\
        \begin{tabular}{l} 4 \end{tabular} & \begin{tabular}{l} uncountable infinitely many \end{tabular} \\
        \begin{tabular}{l} 5 \\ 6 \\ \vdots \end{tabular} & $\left.\begin{tabular}{l} finite \\ finite \\ finite \end{tabular}\right\}$ surgery theory
    \end{tabular}
    \end{center}
\end{theorem}

\section{Tangent Spaces}
``What is the velocity of a curve $\gamma$ at a point $p$?''
\subsection{Velocities, Tangent Spaces}
\begin{defn}[Velocity]
    $(\mathcal{M,O,A})$ smooth manifold, curve $\gamma: \mathbb{R}\to \M$ at least $C^1$.
    Suppose $\gamma(\lambda_0) = p$.
    The velocity $v$ of $\gamma$ at $p$ is the linear map
    \begin{align}
        v_{\gamma, p}: C^\infty(\M) \linearto \mathbb{R}\,,
    \end{align}
    with
    \begin{equation}
        f \mapsto v_{\gamma,p}(f) := (f \circ \gamma)'(\underbrace{\gamma^{-1}(p)}_{\lambda_0})\,,
    \end{equation}
    \textit{i.e.}, the directional derivative of $f$ along $\gamma$ at the point $p$
\end{defn}
\begin{note}
    Remember 
    \begin{align}
        C^\infty(\M) &= \left\{ f: \M\to\mathbb{R} | f \text{ smooth, }\right. \\
            &(f+g)(p):= f(p) + g(p)\\
        &\left.(\lambda \cdot g)(p) := \lambda \cdot g(p), \lambda \in \mathbb{R}\right\}
    \end{align}
    is a vector space.
\end{note}
Since $f\circ\gamma: \mathbb{R}\to\mathbb{R}$ we can simply take the normal derivative.
\begin{center}
\begin{tikzpicture}
\matrix (m) [matrix of math nodes,row sep=3em,column sep=4em,minimum width=2em]
{
    \mathbb{R} & M & \mathbb{R} \\
};
\path[-stealth]
    (m-1-1) edge node [above] {$\gamma$} (m-1-2)
    (m-1-2) edge node [above] {$f$} (m-1-3)
    (m-1-1) edge[bend left = -60] node [above] {$f\circ\gamma$} (m-1-3);
    %(m-1-1) edge node [below] {$\tilde{y} \circ \phi \circ \tilde{x}^{-1}$} (m-1-2)
    %(m-2-1) edge node [left] {$\tilde{x}$} (m-1-1)
    %(m-2-2) edge node [right] {$\tilde{y}$} (m-1-2)
    %(m-2-1) edge node [above] {$\phi$} (m-2-2)
    %(m-2-1) edge node [left] {$x$} (m-3-1)
    %(m-2-2) edge node [right] {$y$} (m-3-2)
    %(m-3-1) edge node [above] {$y \circ \phi \circ x^{-1}$} (m-3-2)
    %(m-3-1) edge[bend left = 60] node [left, rotate = 90, xshift = 0.5cm, yshift = +0.2cm] {$C^\infty$} (m-1-1)
    %(m-3-2) edge[bend right = 60] node [right, rotate = 90, xshift = -0.4cm, yshift = -0.25cm] {$C^\infty$} (m-1-2);
\end{tikzpicture}
\end{center}
\begin{note}
    In differential calculus one had the directional derivative as $v^i(\partial_i f)$.
    The shift in philosophy is now to see $v^i\partial_i$, \textit{i.e.}, the operator
    that acts on $f$, as the vector.
\end{note}
\begin{defn}[Tangent space]
    $\forall p \in \M$ the ``tangent space to $\M$ at $p$'' consists of all the velocities
    of curves at that point :
    \begin{equation}
        T_p\M:=\left\{ v_{\gamma,p} | \gamma\text{ smooth curves} \right\}\,.
    \end{equation}
    \label{def:tangentspace}
\end{defn}
\begin{note}
    There is no reference to any external space or embedding in definition~\ref{def:tangentspace},
    see also figure~\ref{fig:tangentspace}!
\end{note}
$T_p\M$ can be made into a vector space. The proof for this is so important and contains so many
important things, that one should go through it in detail.

\begin{figure}[tbh]
    \centering
    \begin{tikzpicture}[x={(170:1cm)},y={(55:.7cm)},z={(90:1cm)}]
        \draw (2.5,-2.5,0)  -- (2.5,2.5,0) node[below right] {$T_p\M$} -- (-2.5,2.5,0) -- (-2.5,-2.5,0) -- cycle;
        \draw[looseness=.6] (2.5,-2.5,-1) node[above right] {$\M$}
        to[bend left] (2.5,2.5,-1)
        to[bend left] coordinate (mp) (-2.5,2.5,-1)
        to[bend right] (-2.5,-2.5,-1)
        to[bend right] coordinate (mm) (2.5,-2.5,-1)
        -- cycle;
        \draw[dashed,looseness=.2] (mm) to[bend left] (0,0,0) to[bend left] (mp);
        \draw[dashed,looseness=.8] (2,-0.9,0) to[bend left] (-2.8,-1,0);% to[bend left] (mp);
        \path[looseness=.2] (mm) to[bend left]
        node[pos=.2,pin={[pin distance=1cm,pin edge={solid,<-}]below right:$\gamma$}] {} (0,0,0);
        \path[looseness=.2] (2,-0.3,0) to[bend right]
        node[pos=.2,pin={[pin distance=0.5cm,pin edge={solid,<-}]below right:$\delta$}] {} (-2.8,0,0);

        \draw[->] (0,0,0) -- (0,1.5,0) node[above right] {$\dot{\gamma} = v_{\gamma, p}$};
        \node at (-0.3,-0.4,0) {$p$};
    \end{tikzpicture}
    \caption{Picture for immagining the tangent space. Keep in mind that there is no embedding needed like
    in this picture. Also in one can only think of $v_{\gamma,p}$ as an arrow in the chart, but not at
    manifold level. Also it is usefull to think of the arrow as the directional derivative $\partial_v$ in
    the direciton of this arrow.}
    \label{fig:tangentspace}
\end{figure}
\begin{defn}[Addition and multiplication for tangent space]
    For $p\in\M$, $\gamma$ smooth curve on $\M$, $\alpha \in \mathbb{R}$:
    \begin{align}
        +:~ &T_p\M\times T_p\M \to \mathrm{Hom}(C^\infty(\M), \mathbb{R}) \nonumber\\
        &(v_{\gamma,p} + v_{\delta, p})(f) := v_{\gamma, p}(f) + v_{\delta, p}(f)\\
        & f \in C^\infty(M)\nonumber\\
        \cdot:~& \mathbb{R} \times T_p\M \to \mathrm{Hom}(C^\infty(\M), \mathbb{R}) \nonumber\\
        & (\alpha \cdot v_{\gamma, p})(f) = \alpha \cdot v_{\gamma, p}(f)
    \end{align}
\end{defn}
But do they close, \textit{i.e.}, is the tangent space a vector space?
It remains to be shown that
\begin{enumerate}
    \item $\exists\, \text{curve }\sigma: v_{\gamma, p} + v_{\delta, p} = v_{\sigma, p}$
        \label{item:addofcurves}
    \item $\exists\, \text{curve }\tau: \alpha\cdot v_{\gamma,p} = v_{\tau, p}$
\end{enumerate}
The problem for~\ref{item:addofcurves} is that one cannot define $v_{\gamma,p} + v_{\delta, p}$
as just adding the points of the curves, since there is no such thing as adding two points on 
a manifold (what would be Paris + Berlin?).\\
\begin{proof} (Tangent space is a vector space)\\
    \begin{itemize}
        \item[2.]
            Construct $\tau: \mathbb{R} \to \M$:
            \begin{equation}
                \tau(\lambda) := \gamma(\alpha\cdot\lambda + \lambda_0) = (\gamma\circ\mu_\alpha)(\lambda)\,
            \end{equation}
            with $\mu_\alpha: \mathbb{R}\to\mathbb{R}, r\mapsto \alpha r + \lambda_0$.
            Then $\tau(0) = \gamma(\lambda_0) = p$.
            \begin{align}
                v_{\tau, p} &:= (f\circ\tau)'(0) = (f\circ\gamma\circ\mu_\alpha)'(0)\nonumber\\
                &= (f\circ\gamma)(\lambda_0)\alpha = \alpha v_{\gamma,p}\,.
            \end{align}
        \item [1.] Two curves $\gamma(\lambda), \delta(\lambda)$ with $\gamma(\lambda_0) = p$ and
            $\delta(\lambda_1) = p$. 
            Make a choice of chart $(U, x)$ with $p\in U$, later show independence of chart.
            Define
            \begin{align}
                \sigma_x&:\,\mathbb{R}\to\M\,,\nonumber \\
            \sigma_x(\lambda)&:= x^{-1} \left[ (x\circ\gamma)(\lambda_0 + \lambda)\right. +\\
            &+ \left.(x\circ\delta)(\lambda_1+\lambda) - (x\circ\gamma)(\lambda_0)\right]\,.\nonumber
            \end{align}
            \begin{align}
                \sigma_x(0) &= \delta(\lambda_1) = p\,,\\
                v_{\sigma_x, p}(f) &= (f\circ\sigma_x)'(0)\label{eq:vsigma}\\
                &= \left[ \underbrace{(f\circ x^{-1})}_{\mathbb{R}^d\to\mathbb{R}}\circ\underbrace{(x\circ\sigma_x)}_{\mathbb{R}^d\to\mathbb{R}} \right]'(0)\nonumber\,,
            \end{align}
            where now we use the multidimensional chain rule that a physicist would rather know as
            $\frac{\diff}{\diff \lambda}f(\vec{y}(\lambda))
            = (\vec{\nabla}_{y} f)\cdot \frac{\diff \vec{y}}{\diff\lambda}$.
            \begin{align*}
                v_{\sigma_x,p}(f) &= (x^i\circ\sigma_x)'(0)\left[\partial_i(f\circ x^{-1})\right]
                (\underbrace{x(\sigma_x(0))}_{x(p)})\\
                (x^i\circ\sigma_x)'(0) &=[(x^i\circ\gamma(\lambda_0+\lambda) + (x^i\circ\delta)(\lambda_1+\lambda)\\ 
                &~- (x^i\circ\gamma)(\lambda_0)]'\\
                &= (x^i\circ\gamma)'(\lambda_0) + (x^i\circ\delta)'(\lambda_1) = v^i_{\gamma,p} + v^i_{\delta,p}\,.
            \end{align*}
            Plugging this into equation~(\ref{eq:vsigma}) and doing the same step backwards we get
            \begin{equation}
                v_{\sigma_x, p}(f) = v_{\gamma,p}(f) + v_{\delta, p}(f)\quad\forall f\in C^\infty(\M)\,,
            \end{equation}
            independent of the chart.
    \end{itemize}
    \begin{note}
        It turns out that the sum of curves like this depends on the map $(U, x)$, but the derivative
        at the point $p$ does not.
    \end{note}
\end{proof}

%------------------------------------------------


%----------------------------------------------------------------------------------------
%	BIBLIOGRAPHY
%----------------------------------------------------------------------------------------

\printbibliography[title={Bibliography}] % Print the bibliography, section title in curly brackets

%----------------------------------------------------------------------------------------
\end{document}

