\section{Einstein Gravity}
Recall that in Newtonian spacetime we were able to reformulate the Poisson law
$\Delta \phi = 4 \pi G_\text{N} \rho$
in terms of the \textit{Newtonian spacetime} curvature as
\begin{equation*}
    R_{00} = 4 \pi G_\text{N} \rho\,,
\end{equation*}
This prompted Einstein to postulate that the relativistic field 
equations for the Lorentzian metric $g$ of relativistic spacetime
is 
\begin{equation*}
    R_{ab} = 8 \pi G_\text{N} T_{ab}\,.
    \label{eq:ProposedEFEs}
\end{equation*}
However, this equations suffers from a problem:
The divergence of the energy momentum tensor should be zero,
\begin{equation*}
    (\nabla_a T)^{ab} = 0\,.
\end{equation*}
\begin{note}
    Schuller states that this is true because $T$ comes from an action and
    could have been formulated without the choice of coordinates,
    which gives this kind of conservation law because of Noether's theorem.
    I don't understand the details.
\end{note}
But generally the left hand side of eq.~(\ref{eq:ProposedEFEs}) does not have
zero divergence,
\begin{equation*}
    (\nabla_a R)^{ab} \neq 0\,,
\end{equation*}
\subsection{Hilbert}
Hilbert was a specialist for variational principles.
To find the appropriate left hand side of the gravitational field equations,
Hilbert suggested to start from the (simplest) action
\begin{equation}
    S_\text{Hilbert}[g] = \int\limits_\M \sqrt{-g} R_{ab}g^{ab}\,.
\end{equation}
Variation of $S_\text{Hilbert}$ with respect to the metric yields $G^{ab}$.
\begin{note}[Simplest function of the metric]
    $R_{ab}$ is actually the simplest function one can build from the
    metric. From the zeroth derivative of the metric one cannot build anything
    interesting, from the first one can't build any tensor and from the second
    one can build $R_{ab}$.
\end{note}


\paragraph{Variation of the Hilbert action}
\begin{align*}
    &0 \stackrel{!}{=} \delta_g S_\text{Hilbert}[g] = \delta_g \int\limits_\M \sqrt{-g} R_{ab}g^{ab}\\
    &= \int\limits_\M \left( (\delta_g\sqrt{-g}) g^{ab} R_{ab} + \sqrt{-g} (\delta_g g^{ab}) R_{ab}
    + \sqrt{-g} g^{ab} (\delta_g R_{ab})\right)
\end{align*}
We look at the different terms separately:
\begin{itemize}
    \item 
        \begin{equation}
            \delta_g\sqrt{-g} = \frac{- (\det g) g^{mn}\delta g_{mn}}{2 \sqrt{-g}}\,,
        \end{equation}
        where we have used that $\delta_g \det(g) = \det(g) g^{mn}\delta g_{mn}$,
        which can be derived from $\det(g) = \exp \mathrm{tr} \ln g$ or how I remember it:
        ``ln det = trace log''.
    \item 
        From $g^{ab}g_{bc} = \delta^a_c$ follows
        \begin{equation}
            \delta g^{ab} = - g^{am} g^{bn} \delta g_{mn}\,.
        \end{equation}
    \item Variation of the Ricci tensor:
        We go to normal coordinates (at a point) where $\Gamma = 0$,
        \begin{align*}
            \delta R_{ab} &= \delta \partial_b \Gamma^m{}_{am} -
            \delta \partial_m \Gamma^m{}_{ab} + \delta(\cancel{\Gamma\Gamma} - \cancel{\Gamma\Gamma})\\
            &= \partial_b \delta \Gamma^{m}{}_{am} - \partial_m \delta \Gamma^m{}_{ab}\,.
        \end{align*}
        Now $\delta \Gamma^m_{ab}$ is the difference between two $\Gamma$s.
        Remember that $\Gamma - \tilde{\Gamma}$ is a tensor, because under coordinate transformation
        $\Gamma$ picks up a wrong term ($\Gamma$ is no tensor),
        but $\tilde\Gamma$ pics up the same term and so those cancel and the difference between
        two connection coefficients is a tensor.

        Because we are in normal coordinates and $\delta \Gamma$ is a tensor, we can now just write
        \begin{equation*}
            = \nabla_b \delta \Gamma^{m}{}_{am} - \nabla_m \delta \Gamma^m{}_{ab}\,,
        \end{equation*}
        since the difference between $\partial$ and $\nabla$ is a $\Gamma = 0$ in
        normal coordinates.
        Introducing new notation
        \begin{align}
            (\nabla_b A)^i{}_j &=: A^i{}_{j;b}\,,\\
            \partial_b A^i{}_j &=: A^i{}_{j,b}\,.
        \end{align}
        we have, using $\nabla g = 0$,
        \begin{align*}
            \sqrt{-g}& g^{ab} \delta R_{ab} = \sqrt{-g} \left( g^{ab} \delta \Gamma^m{}_{am} \right)_{;b}\\
             &~ \quad\quad\quad - \sqrt{-g} \left( g^{ab} \delta \Gamma^m{}_{ab} \right)_{;m}\\
             &= \sqrt{-g} \left( g^{ab} \delta \Gamma^m{}_{am} \right)_{;b} - 
              \sqrt{-g} \left( g^{ab} \delta \Gamma^m{}_{ab} \right)_{;m}\\
              &= \left(\sqrt{-g} A^a\right)_{;a} - \left( \sqrt{-g} B^b \right)_{;b}\\
              &= \left(\sqrt{-g} A^a\right)_{,a} - \left( \sqrt{-g} B^b \right)_{,b}\,.
        \end{align*}
        Collecting the terms, one obtains:
        \begin{align*}
            0 = \delta S_\text{Hilbert} &= \int\limits_\M
            \left[ \frac{1}{2} \sqrt{-g} g^{mn} \delta g_{mn} g^{ab}R_{ab}\right.\\
        &~\left.- \sqrt{-g} g^{am}g^{bn}\delta g_{nm}R_{ab} \right. \\
        &~\left.+ \left( \sqrt{-g} A^a \right)_{,a} - \left( \sqrt{-g} B^b \right)_{,b} \right]\,.
        \end{align*}
        The last two terms are surface terms and to get the equations of motion we can ignore them.
        We end up with
        \begin{equation}
            \delta S_\text{Hilbert} = \int\limits_\M \sqrt{-g}
            \underbrace{\left[ \frac{1}{2} g^{mn} R - R^{mn} \right]}_{- G^{mn}} \delta g_{mn}\,,
        \end{equation}
        with $\delta g_{mn}$ an arbitrary variation.
        We have defined the \textit{Einstein Tensor}
        \begin{equation}
            G^{mn} = R^{mn} - \frac{1}{2} g^{mn} R \,.
        \end{equation}
        Hence, Hilbert from this ``mathematical argument'' concluded that
        one may take
        \begin{equation}
            \boxed{%
            G^{mn} = R^{mn} - \frac{1}{2} g^{mn} R = 8 \pi G_\text{N} T_{ab}
        }
        \label{eq:EFEs}
        \end{equation}
        and in fact Einstein by physical arguments arrived at the same result.
        Equation~(\ref{eq:EFEs}) is called \textit{Einstein Field Equations} and the action is
        also called \textit{Einstein-Hilbert action}
        \begin{equation}
            \boxed{%
                S_\text{EH}[g] = \int\limits_\M \sqrt{-g} R
            }\,.
        \end{equation}
\end{itemize}
\paragraph{Solution of the $(\nabla_a T)^{ab} = 0$ issue:}
One can show that the \textit{Einstein curvature tensor}
\begin{equation}
    G_{ab} = R_{ab} - \frac{1}{2} g_{ab} R\,,
\end{equation}
satisfies the so-called \emph{contracted differential Bianci identity}
\begin{equation}
    (\nabla_a G)^{ab} = 0\,,
\end{equation}
and so the Einstein equations~(\ref{eq:EFEs}) fulfill energy momentum conservation
$(\nabla_a T)^{ab} = 0$.
\begin{note}[Conservation of the Einstein Tensor]
    Indeed $(\nabla_a G)^{ab}$ has to hold for the same reason for which
    $(\nabla_a T)^{ab}$ has to hold, namely that they were derived from
    an action.
\end{note}

\subsection{Variants of the Field Equations}
\begin{enumerate}[label=(\alph*)]
    \item A simple rewriting (forgetting constants in front of $T$):
        \begin{align*}
            R_{ab} - \frac{1}{2} g_{ab} R &= T_{ab} \quad\quad \Big|\, g^{ab} \\
            R - 2R &= T_{ab}g^{ab} = T \\
            R &= -T \\
            R_{ab} &= T_{ab} - \frac{1}{2}T g_{ab} =: \hat{T}_{ab}
        \end{align*}
        This is just a modified energy-momentum, that now of course
        does no longer fulfill $(\nabla_a \hat{T})^{ab} \neq 0$.
    \item Cosmological constant:
        \begin{equation}
            S_\text{EH}[g] := \int_\M \sqrt{-g} \left( R + 2 \Lambda \right) \,,
        \end{equation}
        with the cosmological constant $\Lambda$, which does not destroy
        the conservation of the energy momentum tensor.
        This can be interpreted as a contribution to curvature or to matter.
        If we understand it as a contribution $-\frac{1}{2}\Lambda g_{ab}$ to matter,
        then it is like a constant extra energy density all over the universe,
        ``dark energy''!
        You cannot see it, but it contributes to the energy density of the universe.
        \begin{equation}
            R_{ab} - \frac{1}{2} g_{ab} R = 8\pi G_\text{N}T_{ab} - \frac{1}{2} \Lambda g_{ab}\,.
        \end{equation}
        History:
        \begin{itemize}
            \item[1915:] $\Lambda <0$: Einstein wanted a non-expanding universe.
            \item[>1915:] $\Lambda = 0$: Hubble.
            \item[today:] $\Lambda > 0$ accelerated expanding universe.
        \end{itemize}
        The measured $\Lambda$ is very small.
        A good idea is that it might come from the (quantum) vacuum fluctuation of the
        standard matter fields, but if one actually does the calculation one gets
        \begin{equation}
            \Lambda_\text{calc} = 10^{120} \cdot \Lambda_\text{obs}\,,
        \end{equation}
        which is the \textsc{worst prediction of physics}.
        
\end{enumerate}
