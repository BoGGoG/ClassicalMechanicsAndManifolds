\section{Einstein Gravity}
Recall that in Newtonian spacetime we were able to reformulate the Poisson law
$\Delta \phi = 4 \pi G_\text{N} \rho$
in terms of the \textit{Newtonian spacetime} curvature as
\begin{equation*}
    R_{00} = 4 \pi G_\text{N} \rho\,,
\end{equation*}
This prompted Einstein to postulate that the relativistic field 
equations for the Lorentzian metric $g$ of relativistic spacetime
is 
\begin{equation*}
    R_{ab} = 8 \pi G_\text{N} T_{ab}\,.
\end{equation*}
However, this equations suffers from a problem:
\begin{equation*}
    (\nabla_a T)^{ab} = 0\,,
\end{equation*}
\begin{note}
    Schuller states that this is true because $T$ comes from an action and
    could have been formulated without the choice of coordinated.
    I don't know why.
\end{note}
But generally
\begin{equation*}
    (\nabla_a R)^{ab} \neq 0\,,
\end{equation*}
\subsection{Hilbert}
Hilbert was a specialist for variational principles.
To find the appropriate left hand side of the gravitational field equations,
Hilbert suggested to start from the (simplest) action
\begin{equation}
    S_\text{Hilbert}[g] = \int\limits_\M \sqrt{-g} R_{ab}g^{ab}\,.
\end{equation}
Varying of $S_\text{Hilbert}$ yields $G^{ab}$.

\paragraph{Variation of the Hilbert action}
\begin{align*}
    &0 \stackrel{!}{=} \delta_g S_\text{Hilbert}[g] = \delta_g \int\limits_\M \sqrt{-g} R_{ab}g^{ab}\\
    &= \int\limits_\M \left( \delta_g\sqrt{-g} g^{ab} R_{ab} + \sqrt{-g} \delta_g g^{ab} R_{ab}
    + \sqrt{-g} g^{ab} \delta_g R_{ab}\right)
\end{align*}
We look at the different terms separately:
\begin{itemize}
    \item 
        \begin{equation}
            \delta_g\sqrt{-g} = \frac{- (\det g) g^{mn}\delta g_{mn}}{2 \sqrt{-g}}\,,
        \end{equation}
        where we have used that $\delta_g \det(g) = \det(g) g^{mn}\delta g_{mn}$,
        which can be derived from $\det(g) = \exp \mathrm{tr} \ln g$ or how I remember it:
        ``ln det = trace log''.
    \item 
        From $g^{ab}g_{bc} = \delta^a_c$ follows
        \begin{equation}
            \delta g^{ab} = - g^{am} g^{bn} \delta g_{mn}\,.
        \end{equation}
    \item Variation of the Ricci tensor:
        We go to normal coordinates where $\Gamma = 0$,
        \begin{align*}
            \delta R_{ab} &= \delta \partial_b \Gamma^m{}_{am} -
            \delta \partial_m \Gamma^m{}_{ab} + \delta(\cancel{\Gamma\Gamma} - \cancel{\Gamma\Gamma})\\
            &= \partial_b \delta \Gamma^{m}{}_{am} - \partial_m \delta \Gamma^m{}_{ab}\,.
        \end{align*}
        Now $\delta \Gamma^m{ab}$ is the difference between two $\Gamma$s.
        Remember that $\Gamma - \tilde{\Gamma}$ is a tensor, because under coordinate transformation
        $\Gamma$ picks up a wrong term ($\Gamma$ is no tensor),
        but $\tilde\Gamma$ pics up the same term and so those cancel and the difference between
        two connection coefficients is a tensor.

        Because we are in normal coordinates and $\delta \Gamma$ is a tensor, we can now just write
        \begin{equation*}
            = \nabla_b \delta \Gamma^{m}{}_{am} - \nabla_m \delta \Gamma^m{}_{ab}
        \end{equation*}
        Introducing new notation
        \begin{align}
            (\nabla_b A)^i{}_j =: A^i{}_{j;b}\,,
            \partial_b A^i{}_j =: A^i{}_{j,b}\,,
        \end{align}
        we have, using $\nabla g = 0$,
        \begin{align*}
            \sqrt{-g} g^{ab} \delta R_{ab} &= \sqrt{-g} \left( g^{ab} \delta \Gamma^m{}_{am} \right)_{;b}\\
             &~- \sqrt{-g} \left( g^{ab} \delta \Gamma^m{}_{ab} \right)_{;m}\\
        \end{align*}
        \begin{align*}
             &= \sqrt{-g} \left( g^{ab} \delta \Gamma^m{}_{am} \right)_{,b} - 
              \sqrt{-g} \left( g^{ab} \delta \Gamma^m{}_{ab} \right)_{,m}\\
              &= \left(\sqrt{-g} A^a\right)_{,a} - \left( \sqrt{-g} B^b \right)_{,b}\,.
        \end{align*}
        Collecting the terms
        \begin{align*}
            \delta S_\text{Hilbert} &= \int\limits_\M
            \left[ \frac{1}{2} \sqrt{-g} g^{mn} \delta g_{mn} g^{ab}R_{ab}\right.\\
        &~\left.- \sqrt{-g} g^{am}g^{bn}\delta g_{nm}R_{ab} \right. \\
        &~\left.+ \left( \sqrt{-g} A^a \right)_{,a} - \left( \sqrt{-g} B^b \right)_{,b} \right]\,.
        \end{align*}
        The last two terms are surface terms and to get the equations of motion we can ignore them.
        We end up with
        \begin{equation}
            \delta S_\text{Hilbert} = \int\limits_\M \sqrt{-g}
            \underbrace{\left[ \frac{1}{2} g^{mn} R - R^{mn} \right]}_{G^{mn}} \delta g_{mn}\,,
        \end{equation}
        with $\delta g_{mn}$ an arbitrary variation.
        We have defined the \textit{Einstein Tensor}
        \begin{equation}
            G^{mn} = \frac{1}{2} g^{mn} R - R^{mn}\,.
        \end{equation}
        Hence, Hilbert, from this ``mathematical argument'' concluded that
        one may take
        \begin{equation}
            \boxed{%
            G^{mn} = \frac{1}{2} g^{mn} R - R^{mn} = 8 \pi G_\text{N} T_{ab}
        }
        \label{eq:EFEs}
        \end{equation}
        and in fact Einstein by physical arguments arrived at the same result.
        Equation~(\ref{eq:EFEs}) is called \textit{Einstein Field Equations} and the action is
        also called \textit{Einstein-Hilbert action}
        \begin{equation}
            \boxed{%
                S_\text{EH}[g] = \int\limits_\M \sqrt{-g} R
            }\,.
        \end{equation}
\end{itemize}

