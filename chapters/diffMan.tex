\section{Differentiable Manifolds}
So far we only had topological manifolds. We also want to be able to talk about the velocity of
curves. The problem is that the notion of a topological manifold is not enough to define
differentiability of curves. In this section we will find out what additional structure we need
to be able to talk about the differentiability of
\begin{itemize}
    \item curves: $\mathbb{R}\to \M$
    \item functions: $\M\to \mathbb{R}$
    \item maps: $\M\to\mathcal{N}$
\end{itemize}
Strategy: Choose a chart $(U, x)$ and consider portion of the curve in the domain of the chart:
$\gamma: \mathbb{R}\to U$ (see figure~\ref{fig:curvechart}). Since $x\circ\gamma: \mathbb{R}\to\mathbb{R}^d$
we can try to ``lift'' the notion of differentiability of a curve on $\mathbb{R}^d$ to that of
a curve on $\M$. The problem is to make this independent of the chart.
\begin{figure}[tbh]
    \centering
    \begin{tikzpicture}
    \matrix (m) [matrix of math nodes,row sep=3em,column sep=4em,minimum width=2em]
    {
        ~                 & y(U\cup V)\subseteq \mathbb{R}^d\\
        \gamma:\mathbb{R} & U\cup V \neq \emptyset \\
        ~                 & x(U\cup V)\subseteq \mathbb{R}^d \\
    };
    \path[-stealth]
        (m-2-1) edge node [above] {$\gamma$} (m-2-2)
        (m-2-1) edge node [below, rotate = -25] {$x\circ\gamma$} (m-3-2)
        (m-2-1) edge node [above, rotate = +25] {$y\circ\gamma$} (m-1-2)
        (m-3-2) edge[bend right = 60] node [right, rotate = 90, xshift = -0.5cm, yshift = -0.2cm] {$y\circ x^{-1}$} (m-1-2)
        (m-2-2) edge node [right] {$x$} (m-3-2)
        (m-2-2) edge node [right] {$y$} (m-1-2);
    \end{tikzpicture}
    \caption{Curve $\gamma$ in chart.}
    \label{fig:curvechart}
\end{figure}
\begin{equation}
    y\circ\gamma = \underbrace{(y\circ x^{-1})}_{\mathbb{R}^d\to\mathbb{R}^d}\circ \underbrace{(x\circ\gamma)}_{\substack{\mathbb{R}\to\mathbb{R}^d\\ \text{differentiable}}}\,,
\end{equation}
but we only know that the \textit{chart transition map} $y\circ x^{-1}$ is continuous (because of the definition of a top. Manifold). Thus it is not guaranteed that
$y\circ \gamma$ is continuons, not differentiable. Reminder: The composition of continuous maps is continuous,
same for differentiable. 
The above definition of differentiability of $\gamma$ by
checking the differentiability of $x\circ\gamma$ with some chart $x$ is not independent of the chart.

\begin{defn}[$\flower$ - compatibility of charts]
    Two charts $(U, x)$ and $(V, y)$ of a topological manifold are called $\flower$-compatible if
    either
    \begin{enumerate}
        \item $U\cup V = \emptyset$ or
        \item $U\cup V \neq \emptyset$ and the chart transition maps
            \begin{align*}
                y\circ x^{-1}&:\mathbb{R}^d\supseteq x(U\cup V) \to y(U\cup V) \subseteq \mathbb{R}^d \\
                x\circ y^{-1}&: \mathbb{R}^d\supseteq y(U\cup V) \to x(U\cup V) \subseteq \mathbb{R}^d
            \end{align*}
            have the $\flower$-property in the $\mathbb{R}^d$-sense.
    \end{enumerate}
\end{defn}

\begin{defn}[$\flower$-compatible atlas]
    An atlas $\mathcal{A}_\flower$ is a $\flower$-compatible atlas if any two charts in
    $\mathcal{A}_\flower$ are $\flower$-compatible.
\end{defn}

\begin{defn}[$\flower$-manifold]
    A $\flower$-manifold is a triple $(\underbrace{\M, \mathcal{O}}_{\text{top.\ manif.}}, 
    \underbrace{\mathcal{A}_\flower}_{\in \mathcal{A}_\text{max}}$).
\end{defn}

\begin{footnotesize}
    \begin{centering}
        \begin{tabular}{ll}
            $\flower$ & $\flower$ property in $\mathbb{R}^d$-sense \\
            \hline\hline
            $C^0$     & $C^0(\mathbb{R}^d \to \mathbb{R}^d)$ continuous maps\\
            $C^1$     & $C^1(\mathbb{R}^d \to \mathbb{R}^d)$ differentiable and result is cont.\\
            $C^k$     & $C^k(\mathbb{R}^d \to \mathbb{R}^d)$ k times diffble and result is cont.\\
            $D^k$     & $D^k(\mathbb{R}^d \to \mathbb{R}^d)$ k times differentiable\\
            $C^\infty$     & $C^\infty(\mathbb{R}^d \to \mathbb{R}^d)$ smooth functions\\
            $C^\omega$  & $\exists$ multidim. Taylor expansion, $C^\omega \subset C^\infty$
        \end{tabular}
    \end{centering}
\end{footnotesize}

\begin{note}
    The more fancy properties one wants for the objects on the manifold, the more restrictive one
    has to be for the atlas.
\end{note}
\begin{theorem}[$C^1\to C^\infty$]
    Any $C^{k \leq 1}$-manifold atlas $\mathcal{A}_{C^{k\leq 1}}$ of a topological manifold contains
    a $C^\infty$-atlas.
\end{theorem}
Thus we may without loss of generality always consider $C^\infty$-manifolds. 
``smooth'' manifolds, unless we wish to define Taylor expandibility or complex differentiability, \ldots.
\begin{defn}[Smooth manifold]
    $(\M, \mathcal{O}, \mathcal{A})$, where $(\mathcal{M, O})$ is a topological manifold and $\mathcal{A}$ is a $C^\infty$-atlas.
\end{defn}

