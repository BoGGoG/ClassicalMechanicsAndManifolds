\section{Relativistic Spacetime}

On a Lorentzian metric, here we choose the sign $(+---)$ in every tangent space
there is a \textit{light cone} defined by the metric $g$.
In figure~\ref{fig:lightCone} such a light cone is drawn.
Inside the cone there is $g(X,X)>0$ (timelike, same sign as time in metric)
, on the (surface of) the cone there is
$g(X,X)=0$ (lightlike or null) and outside $g(X,X)<0$ (spacelike, same sign as space
in metric).
\begin{figure}[tbh]
    \centering\def\svgwidth{\columnwidth}
    \input{figures/lightCone.pdf_tex}
    \caption{Light cone. One should rather think of a light cone in the tangent space of every
    point.}
    \label{fig:lightCone}
\end{figure}

We have to choose a time orientation, \textit{i.e.}\ which side of the cone is
future and which is past.
We do this by introducing a vector $T$.
\begin{defn}[Time-Orientation]
    Let $(\M, \calO, \calA^\uparrow, g)$ be a Lorentzian manifold.
    Then a time-orientation is given by a \textit{smooth} vector field $T$ that
    \begin{enumerate}
        \item does \textit{not vanish} anywhere,
        \item $g(T,T)>0$.
    \end{enumerate}
\end{defn}
We made this definition of \textit{spacetime} in order to enable the
following physical postulates:
\begin{enumerate}[label=(\subscript{P}{{\arabic*}})]
    \item The worldline $\gamma$ of a \textit{massive} particle satisfies
        \begin{enumerate}
            \item $g_{\gamma(\lambda)} \left( v_{\gamma,\gamma(\lambda)}, v_{\gamma,\gamma(\lambda)} \right) > 0\,\forall \lambda\in I$\,,
            \item $g_{\gamma(\lambda)} \left(T, v_{\gamma,\gamma(\lambda)} \right) > 0\,\forall \lambda\in I$\,,
        \end{enumerate}
        \label{item:massive}
    \item The worldlines of \textit{massless} particles satisfy 
        \begin{enumerate}
            \item $g_{\gamma(\lambda)} \left( v_{\gamma,\gamma(\lambda)}, v_{\gamma,\gamma(\lambda)} \right) = 0\,\forall \lambda\in I$\,,
            \item $g_{\gamma(\lambda)} \left(T, v_{\gamma,\gamma(\lambda)} \right) > 0\,\forall \lambda\in I$\,,
        \end{enumerate}
        \label{item:massless}
\end{enumerate}
I am not quite sure about the index of $g$, but I think the thing is that $g$ is a tensor field,
so it takes two vector fields  $\T\M\time\T\M\to\mathbb{R}$,
whereas we plug in two vectors, so we have to take the tensor field at the point $p\in\M$ 
where the vectors are in the tangent space $\T{p}\M$

Now the picture is that at every point $p$ in spacetime $\M$ there is a (double) cone
like in figure~\ref{fig:lightCone}.
Through the time orientation $T$ we decide to only take one side of that double
cone where $T$ lies in.
A massive particle has its velocity always in the light cone and a massless particle has its
velocity always on the boundary of the cone, see figure~\ref{fig:spacetimeLightCones}
\textit{Claim:} 9/10 of a metric are determined by the cone(s).

\begin{figure}[tbh]
    \centering\def\svgwidth{\columnwidth}
    \input{figures/spacetimeLightCones.pdf_tex}
    \caption{Light cones in space time, massive and massless particle.}
    \label{fig:spacetimeLightCones}
\end{figure}
\begin{note}
    The difference to Newtonian physics is now, that in Newtonian physics every tangent space was
    divided in a future and a past. 
    Now in General Relativity every tangent space is divided in the future light cone
    and what is not the future light cone (past light cone and something that is not in the
    future or past light cone).
\end{note}

\subsection{Observers}
Spacetime $(\M, \calO, \calA^\uparrow, \nabla, g, T)$.
\begin{defn}[Observer]
    An \textit{observer} is a \textit{worldline} $\gamma$ with
    $g(v_\gamma, v_\gamma) = 1$, $g(T, v_\gamma)>0$
    together with a choice of basis
    \begin{equation}
        \left\{ e_0(\lambda) = v_{\gamma,\gamma(\lambda)}, e_1(\lambda), e_2(\lambda), e_3(\lambda) \right\}\,,
    \end{equation}
    of each $T_{\gamma(\lambda)}\M$ where the observer worldline passes,
    if
    \begin{equation}
        g\left( e_a(\lambda), e_b(\lambda) \right) = \eta_{ab} = \diag(1, -1, -1, -1)\,.
    \end{equation}
    (More precisely: Observer = \textit{smooth} curve in the frame bundle
    $\mathrm{L}\M$ over $\M$.)
\end{defn}

\begin{figure}[tbh]
    \centering\def\svgwidth{\columnwidth}
    \input{figures/observers.pdf_tex}
    \caption{Observers}
    \label{fig:observers}
\end{figure}

\begin{enumerate}[resume, label=(\subscript{P}{{\arabic*}})]
    \item A \textit{clock} carried by a specific observer $(\gamma, e)$ will measure a
        \textit{time} called \textit{proper time} or \textit{eigentime}
        \begin{equation}
            \tau := \int_{\lambda_0}^{\lambda_1} \diff \lambda \sqrt{g_{\gamma(\lambda)}(v_{\gamma, \gamma(\lambda)}, v_{\gamma, \gamma(\lambda)}}
            \end{equation}
            between the two \textit{events} (see figure~\ref{fig:observers})
            \begin{itemize}
                \item $\gamma(\lambda_0)$: ``start the clock'' and
                \item $\gamma(\lambda_1)$: ``stop the clock''.
            \end{itemize}
            \label{item:clock}
\end{enumerate}
\paragraph{Example: (Twin paradoxon)}
Let $\M = \mathbb{R}^4$, $\mathcal{O} = \mathcal{O}_\text{st}$, 
$\calA^\uparrow \ni (\mathbb{R}^4, \mathrm{id}_{\mathbb{R}^4})$, $T^i_{(x)} = (1,0,0,0)^i$,
$g_{(x)ij} = \eta_{ij} \Rightarrow \Gamma^i_{(x)jl}=0$ everywhere,
$\Rightarrow \mathrm{Riem}=0$ everywhere.
This spacetime is flat and this situation is called \textit{Special Relativity}.
\begin{note}
    Otherwise Special Relativity is not different from General Relativity.
\end{note}
Consider two observers
\begin{align}
    \gamma&: (0,1)\to \M\\
    \gamma_{(x)}^i &= (\lambda, 0,0,0)^i
\end{align}
and
\begin{align}
    \delta&:(0,1)\to\M\\
    \delta_{(x)}^i &=
    \begin{cases}
        (\lambda, \alpha \lambda, 0,0)^i\,,& \lambda \leq \frac{1}{2}\\
        (\lambda, (1-\lambda) \alpha , 0,0)^i\,,& \lambda > \frac{1}{2}\\
    \end{cases}\,,
\end{align}
for $\alpha\in(0,1)$.
Let's calculate
\begin{align}
    \tau_\gamma &:= \int_0^1 \diff \lambda \sqrt{g_{(x)ij}\dot{\gamma}^i_{(x)}\dot{\gamma}^j_{(x)}} = 1\,,\\
    \tau_\delta &= \int_0^{\frac{1}{2}}\diff \lambda \sqrt{1 - \alpha^2} + \int_{\frac{1}{2}}^1 \sqrt{1 - (-\alpha)^2}
    = \sqrt{1 - \alpha^2}\,.
\end{align}
That means if we push $\alpha$ close to 1 we can make the time that the observer
$\delta$ measures close to zero.

Taking the clock postulate~\ref{item:clock} seriously,
one should better come up with a realistic clock design that suports the postulate.
Idea: Light clock, see figure~\ref{fig:lightClock}.
\begin{figure}[tbh]
    \centering\def\svgwidth{0.6\columnwidth}
    \input{figures/lightClock.pdf_tex}
    \caption{A light clock. Left: 11 ticks, right: 6 ticks.}
    \label{fig:lightClock}
\end{figure}
\begin{enumerate}[resume, label=(\subscript{P}{{\arabic*}})]
    \item Let $(\gamma, e)$ be an observer and $\delta$ be a \textit{massive} particle worldline
        that is parametrized such that $g(v_\delta, v_\delta)=1$.
        Suppose the observer and the particle \textit{meet} somewhere in spacetime.
        \begin{equation}
            \delta(\tau_2) = p = \gamma(\tau_1)
        \end{equation}
        This observer measures the 3-velocity (spatial velocity) of this particle as 
        \begin{equation}
            v^{(3)}_{\delta, \delta(\tau_2)} := \epsilon^\alpha\left(v_{\delta, \delta(\tau_2)}\right)e_\alpha\,,
            \, \alpha = 1,2,3\,,
        \end{equation}
        where $\{\epsilon^0, \epsilon^1, \epsilon^2, \epsilon^3\}$ is the unique dual basis
        of the basis $\{e_0, e_1, e_2, e_3\}$ of the observer $\gamma$ and we only take the spatial vectors,
        see figure~\ref{fig:3velocity}
        \label{item:3velocity}
\end{enumerate}
\begin{note}
    What happens here is that we project the 4-velocity $v_\delta$ onto the ``spatial part'' of the
    frame given by the observer $\gamma$.
    Of course $v_\delta$ is an objective thing, but $v_\delta^{(3)}$ depends on the observer,
    which is the whole point of doing it.
    $v_\delta^{(3)}$ is actually a 4-vector, but it just happens to ``lie in the span of $e_1, e_2, e_3$''.
    We will see that $|v_\delta^{(3)}|\leq 1$.
\end{note}
\begin{figure}[tbh]
    \centering\def\svgwidth{\columnwidth}
    \input{figures/3vector.pdf_tex}
    \caption{Projection of the velocity $v_\delta$ to the 3-velocity $v_\delta^{(3)}$.}
    \label{fig:3velocity}
\end{figure}
\paragraph{Consequence:}
An observer $(\gamma, e)$ will extract quantities measurable in his laboratory frame
from objective quantities always like that.
paragraph{Example:}
The Faraday (0,2)-tensor $F$ of electromagnetism is the objective quantity and
\begin{equation}
    F(e_a, e_b) = F_{ab} = 
    \begin{pmatrix}
        0     & -E_x & -E_y & -E_z \\
        E_x &  0     & -B_z   &  B_y    \\
        E_y &  B_z   &  0     & -B_x   \\
        E_z & -B_y   &  B_x   &  0
    \end{pmatrix}
\end{equation}
is what the observer measures.
\begin{note}
    I am not quite sure how to understand this.
    Don't we have the tensor field $F$ that we can express as $F_{ab}$ for some
    chart?
    As I see it, an observer gives something like a chart on a worldline, but
    $F_{ab}$ can exist on the whole are where we defined a chart.
\end{note}

\subsection{Role of the Lorentz Transformations}
Lorentz transformations emerge as follows:
Let $(\gamma, e)$ and $(\tilde{\gamma}, \tilde{e})$ be observers with
$\gamma(0) = \tilde{\gamma}(0) = p$.
Now $\{e_0, \dots, e_3\}$ and $\{\tilde{e}_0, \dots, \tilde{e}_3\}$ at $\tau=0$
are both bases for the same $\T_{\gamma(0)}\M$.
Thus
\begin{equation}
    \tilde{e}_a = \Lambda^b{}_a e_b\,,\quad \Lambda \in GL(4)\,.
\end{equation}
Now, since both are observers,
\begin{align}
    \eta_{ab} &= g(\tilde{e}_a, \tilde{e}_b)\\
    &= g\left( \lambda^m{}_a e_m, \Lambda^m{}_b e_n \right) =
    \Lambda^m{}_a \Lambda^n{}_b g(e_m, e_n) \\
    &= \Lambda^m{}_a \Lambda^n{}_b \eta_{mn}\,,
\end{align}
\textit{i.e.} $\Lambda \in O(1,3)$, the transformation $\Lambda$ is an element
of the \textit{Lorentz transformations}.

\textit{Result:} Lorentz transformations relate the frames of any two observers
\textit{at the same point}.
They do not transform spacetime! They just act on one tangent space of spacetime
where the two observers meet!
\begin{note}
    $\tilde{x}^\mu = \Lambda^\mu{}_\nu x^\nu$ is utter nonsense.
    This is just abuse of flat space structure.
    In SRT as well as GR Lorentz transformations transform between tangent spaces
    between observers at the point where they meet.
\end{note}

