\section{Newtonian Spacetime is Curved!?}
Let's review Newton's axioms:
\begin{enumerate}
    \item A body on which \textit{no force} acts moves uniformly along a straight line.
    \item \textit{Deviation} of a body's motion from such uniform straight motion is effected by a
        force, reduced by a factor of the body's reciprocal mass.
\end{enumerate}
One might think that the first axiom is a special case of the second.
The problem with that is that the second axiom needs to know what a 
straight line is.
So it might be a better idea to interpret the first axiom as a measurement prescription
for the geometry of space that tells us what a straight line is.
The second problem is: Gravity acts universally on every particle, so how should
the first axiom ever be applicable if there are at least two particles in the
universe?

Maybe similar reasoning lead Laplace (1749-1827) to state the following question:
\begin{quote}
    Can gravity be encoded in a curvature of space, such that particles that are subject
    to no other force than gravity, move in straight lines in this curved space?
    In other words: Can we get rid of the gravitational ``force'' and put it into the
    geometry of space?\footnote{%
        Not sure if I wrote this correctly. The text in the lecture~\cite{Schuller15}
    is strangely formulated.}
\end{quote}
\paragraph{Answer}: No!
\paragraph{Proof}:
Newton's equation and Laplace's equation are
\begin{align}
    \label{eq:newton} \cancel{m} \ddot{x}^i(t) &= \cancel{m} f^{i}(x(t)) = F^i\,,\\
    -\partial_i f^i &= 4\pi G \rho\,,\\
\end{align}
where $F$ denotes the force, $m$ the mass and $\rho$ the density.
If we could encode this in curvature then we should be able to write equation~(\ref{eq:newton})
as
\begin{align}
    &\ddot{x}^i(t) - f^i(x(t)) = 0\\
    =~&\ddot{x}^i(t) + \Gamma^i_{ab}\dot{x}^a(t)\dot{x}^b(t) = 0
\end{align}
But that is just not possible, since $f^i$ does not depend on the velocity $\dot(x^a)$.

But lo and behold:
We have not used the word \textit{uniformly} in the first axiom.
That basically means that the equal distances are passed in equal times.
\begin{note}
    A curve is more than the set of its points!
    It's the set of its points and its parametrization! 
\end{note}
The trick is: When we have a curve $\gamma(t)$ we can plot it in a coordinate
system with one dimension more that we take as time (just like s-t diagrams in school).
Then we basically put the information of the parametrization in the form of the line
in this space. 

\begin{center}
    \begin{tikzpicture}
        \draw[line width=1pt,blue, xshift = -100](0,0)--(0.5,0.5);
        \draw[line width=1pt,blue, xshift = -100](0.55,0.55)--(1,1);
        \draw[line width=1pt,blue, xshift = -100](1.05,1.05)--(1.5,1.5);
        \draw[line width=1pt,blue, xshift = -100](1.55,1.55)--(2,2);
        \draw[line width=1pt,blue, xshift = -100](2.05,2.05)--(2.5,2.5);
        \draw[line width=1pt,blue, xshift = -100](2.55,2.55)--(3.1,3.1);

        \draw[line width=1pt,red](0,0)--(1,1);
        \draw[line width=1pt,red](1.05,1.05)--(1.8,1.8);
        \draw[line width=1pt,red](1.85,1.85)--(2.4,2.4);
        \draw[line width=1pt,red](2.45,2.45)--(2.8,2.8);
        \draw[line width=1pt,red](2.85,2.85)--(3,3);
    \end{tikzpicture}
    \begin{tikzpicture}
        \draw [<->,thick, xshift = -100] (0,3.2) node (taxis) [above] {$t$}
        |- (3.2,0) node (xaxis) [right] {$x$};

        \draw[line width=1pt,blue, xshift = -100](0,0)--(0.5,0.1);
        \draw[line width=1pt,blue, xshift = -100](0.55,0.12)--(1,0.2);
        \draw[line width=1pt,blue, xshift = -100](1.05,0.22)--(1.5,0.3);
        \draw[line width=1pt,blue, xshift = -100](1.55,0.32)--(2,0.4);
        \draw[line width=1pt,blue, xshift = -100](2.05,0.42)--(2.5,0.5);
        \draw[line width=1pt,blue, xshift = -100](2.55,0.52)--(3.1,0.6);

        \draw [<->,thick, xshift = 20] (0,3.2) node (taxis) [above] {$t$}
        |- (3.2,0) node (xaxis) [right] {$x$};
        \draw[line width=1pt, red, xshift = 20](0,0)--(0.5,0.1);
        \draw[line width=1pt, red, xshift = 20](0.52,0.12)--(1,0.3);
        \draw[line width=1pt, red, xshift = 20](1.02,0.32)--(1.5,0.6);
        \draw[line width=1pt, red, xshift = 20](1.52,0.62)--(2,0.9);
        \draw[line width=1pt, red, xshift = 20](2.02,0.92)--(2.5,1.3);
        \draw[line width=1pt, red, xshift = 20](2.52,1.32)--(3.1,1.8);
        %\draw (0,0) coordinate (a_1) -- (2,1.8) coordinate (a_2);
        %\draw (0,1.5) coordinate (b_1) -- (2.5,0) coordinate (b_2);
        %\coordinate (c) at (intersection of a_1--a_2 and b_1--b_2);
        %\draw[dashed] (yaxis |- c) node[left] {$y'$}
        %-| (xaxis -| c) node[below] {$x'$};
        %\fill[red] (c) circle (2pt);
    \end{tikzpicture}
\end{center}

So now let us try not only in space, but in (Newtonian) spacetime:\\
\bigskip
\begin{minipage}{0.19\textwidth}
Let $x: \mathbb{R} \to \mathbb{R}^3$ be the particle's trajectory
in space fulfilling Newton's law $\ddot{x}^i = f^i(x(t))$.
\end{minipage}
\hfill
$\leftrightarrow$
\hfill
\begin{minipage}{0.25\textwidth}
    worldline (history) of the particle
    $X: \mathbb{R} \to \mathbb{R}^4$\\
    $t\mapsto (t, x^1(t), x^2(t), x^3(t))$
    $:= (X^0(t), X^1(t), X^2(t), X^3(t))$
\end{minipage}
Trivial rewritings:
\begin{equation}
    \dot{X^0} = 1
\end{equation}
and
\begin{align}
    \ddot{X}^0 &= 0\,,\\
    \ddot{X}^i - f^i(X(t))\dot{X}^0\dot{X}^0 &= 0\,,\quad i=1,2,3\,,
\end{align}
which is equivalent to
the autoparallel equation in spacetime
\begin{equation}
    \ddot{X}^\alpha + \Gamma^\alpha{}_{\beta \gamma} \dot{X}^\beta \dot{X}^\gamma = 0\,,\quad \alpha=0,1,2,3\,,
\end{equation}
with
\begin{equation}
    \Gamma^i{}_{00} = -f^i\,,
\end{equation}
and all other components of $\Gamma$ zero.

This is not a coordinate-choice artefact, since
\begin{equation}
    R^{a}{}_{0b0} = -\frac{\partial}{\partial x^b}f^a \neq 0\,\\
\end{equation}
and
\begin{equation}
    R_{00} = R^\mu{}_{0\mu 0} = -\partial f^a = 4\pi G \rho\,.
\end{equation}
If you already know the solution (General Relativity) you can cheat and write
$T_{00} = \rho/2$ to get
\begin{equation}
    R_{00} = 8 \pi G T_{00}\,.
    \label{eq:newtGR}
\end{equation}
Thus Newtonian spacetime is curved (only in time) even if we do not
have relativity and the curvature is prescribed by the distribution of
matter $\rho$.
Uniformly straight in space $\to$ straight in spacetime.

In fact Einstein proposed an equation similar to~(\ref{eq:newtGR}) in 1912, namely
\begin{equation}
    R_{\mu\nu} 8 \pi G T_{\mu\nu}\,,\quad \mu,\nu = 0,1,2,3\,,
\end{equation}
which is not entirely correct, but almost.

\subsection{Foundations of the Geometric Formulation of Newton's Axioms}
\begin{defn}[Newtonian Spacetime]
    A Netwonian spacetime (space+time) is a quintuple
    \begin{equation}
        (\M, \mathcal{O}, \mathcal{A}, \nabla, t)\,,
    \end{equation}
    where $(\M, \mathcal{O}, \mathcal{A})$ form a 4-dimensional smooth manifold and
    the \textit{absolute time}
    \begin{equation}
        t: \M\to\mathbb{R}\,,\quad\text{smooth function}\,,
    \end{equation}
    satisfies
    \begin{enumerate}
        \item There is an absolute space that follows from the existence of absolute time.
            \begin{equation}
                \left( \diff t \right)_p \neq 0\,,\quad\forall p\in\M\,,
            \end{equation}
        \item Absolute time flows uniformly,
            \begin{equation}
                \nabla \diff t = 0\,,\quad\text{everywhere}
            \end{equation}
        \item $\nabla$ is torsion free.
    \end{enumerate}
\end{defn}
\begin{defn}[Absolute space $S_\tau$ at time $\tau$]
    \begin{equation}
        S_\tau:=\left\{ p\in\M | t(p) = \tau \right\}\,,
    \end{equation}
    and thus because of $(\diff \tau)_p \neq 0$
    \begin{equation}
        \M = \bigcupdot S_\tau\,,
    \end{equation}
    where $\bigcupdot$ means the disjoint union.
\end{defn}
This means that $S_\tau$ \textit{foliate} spacetime.
\begin{note}
    By $\nabla \diff t$ we mean that the argument that $\nabla_\cdot$ takes is
    open, so what comes out is a (0,2)-tensor.
\end{note}
\begin{note}
    You can view Gravity as curvature of spacetime and already in Newtonian mechanics
    this is not just an alternative formulation,
    but if you look at the first axiom there is not really another possible choice.
    It's not relativity that forces us to use spacetime, it's gravity itself.
\end{note}
\begin{defn}[]
    A vector $X\in\T{p}\M$ is called
    \begin{enumerate}
        \item future-directed if
            \begin{equation}
                \diff t(X) > 0\,,
            \end{equation}
        \item spatial if
            \begin{equation}
                \diff(X) = 0\,,
            \end{equation}

        \item past-directed if
            \begin{equation}
                \diff{X}<0\,.
            \end{equation}
    \end{enumerate}
\end{defn}
\paragraph{Newton 1:} The worldline of a particle under the influence
of no force (gravity isn't a force now) is a \textit{future directed
autoparallel},\textit{i.e.}\ everywhere
\begin{align}
    \nabla_{v_X}v_X &= 0\,,
    \diff t(v_X)>0\,.
\end{align}
\paragraph{Newton 2:}
The acceleration of a worldline
\begin{equation}
    \underbrace{\nabla_{v_X}v_X}_a = \frac{F}{m}\,,
\end{equation}
where $F$ is a spatial vector field: $\diff t(F) =0$, $X$ is a future directed vector
and $a$ is the acceleration.
\paragraph{Convention:}
Restrict attention to \textit{stratified atlases} $\mathcal{A}_\text{stratified}$
whose charts $(U,x)$ have the property
\begin{equation}
    x^0 = t\eval_U
\end{equation}
In a stratified atlas the first axiom becomes
\begin{align}
    0 = \left( \nabla_\frac{\partial}{\partial x^a} \diff x^0 \right)_b = -\Gamma^0_{ba}\,.
\end{align}

\subsubsection{Geometric Relevance of Riem}
\begin{center}
    \textsc{Put picture here of $[X,Y] = 0$ and
    $[X,Y] \neq 0$}.
    Schuller did a really god job explaining this at the end of lecture
    8, but it's hard to write down.
\end{center}

Assuming a torsion-free connection, $T=0$, then one can imagine curvature as follows.
Parallel transporting a vector $Z$ along two different paths from $p$ to $q$
changes the vector.
Going infinitesimal and ``along'' $X$ or $Y$ (first along $X$ and then along $Y$ or the other
way round) one can find (for $[X,Y] = 0$)
\begin{equation}
    (\delta Z)^m = R^m{}_{nab}X^aY^bZ^n\,\delta s\delta t\,,
\end{equation}
plus higher order terms in the ``lengths of the curves'' $\delta s$, $\delta t$.
One contracts the first and the third index, because all others are either zero or equivalent.

Let's evaluate Newton 2 in a chart $(U,x)$ of a stratified atlas $\mathcal{A}_\text{stratified}$:

\fbox{Finish this section}

