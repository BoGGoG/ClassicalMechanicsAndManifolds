\section{Metric Manifolds}
\subsection[The metric g]{The metric $g$}
We establish a new structure called a \textit{metric} on a smooth manifold $\M$
that allows to assign a length to each vector $X$ in each tangent space $X\in
\T_p\M$ and an angle between vectors in the same tangent space.

Since the velocity $v_{\gamma,p}$ of a curve $\gamma$ at the point $p\in\M$
(defined by equation~(\ref{eq:defVelocity})) is a vector, we can then integrate
up the lengths of the velocities to get the length of a curve.

Then we require that the shortest curves are also the straightest curves,
$\nabla_{v_\gamma} v_\gamma=0$, which will result in the metric determining
the connection $\nabla$ if there is no torsion ($T=0$) and thus also the
curvature.

\begin{defn}[Metric]
    A metrig $g$ on a smooth manifold $(\M, \mathcal{O}, \mathcal{A})$ is a
    (0,2)-tensor field satisfying
    \begin{itemize}
        \item \textit{symmetry}:
            \begin{equation}
                g(X,Y) = g(Y,X)\,,\quad\forall X,Y\in\Gamma(\T\M)\,,
            \end{equation}
        \item \textit{non-degeneracy:} There are no non-zero vectors $X\in\T{p}\M$
            with
            \begin{equation}
                g(X,Y) = 0\,\quad \forall Y\in\T_p\M\,.
            \end{equation}

    \end{itemize}
\end{defn}

\begin{defn}[Inverse Metric $g^{-1}$]
    The symmetric (2,0)-tensor field $g^{-1}$ with respect to a metric $g$ is
    \begin{align}
        g^{-1}: \Gamma(\T^*\M) &\times\Gamma(\T^*\M) \linearto C^\infty(\M)\,\\
        (\omega, \sigma) &\mapsto w\left( \flat^{-1}(\sigma) \right)\,.
    \end{align}
\end{defn}

\begin{defn}[Musical map $\flat$]
    The musical map (``flat'')
    \begin{align}
        \label{eq:musicalMap}
        \flat: \Gamma(\T\M) &\to \Gamma(\T^*\M)\,\\ 
        X &\mapsto \flat(X)\,,
    \end{align}
    where
    \begin{equation}
        \flat(X)(Y):= g(X,Y)\,,
    \end{equation}
    \textit{i.e.}\ the musical map is like a partial evaluation
    of the metric, $\flat(X) = g(X,\cdot)$ and can also be written
    with indices and the so called raising and lowering of indices
    \begin{align}
        X_a := g_{am}X^m := \left( \flat(X) \right)_a\,\\
        X^a := g^{am}X_m := \left( \flat^{-1}(X) \right)^a\,
    \end{align}
\end{defn}

\begin{note}
    In a chart $g_{ab} = g_{ba}$ and
    \begin{equation}
        \left( g^{-1} \right)^{am} g_{mb} = \delta^a_b\,,
        \label{eq:inverseMetricComponents}
    \end{equation}
    but the inverse metric is not really an inverse of the metric.
    We see this by looking at the spaces:
    \begin{align}
        g&: \Gamma(\T\M)\times \Gamma(\T\M) \to C^\infty(\M)\,,\\
        g^{-1}&: \Gamma(\T^*\M)\times \Gamma(\T^*\M) \to C^\infty(\M)\,,
    \end{align}
    so it is not really the inverse map, but the inverse matrix in the sense
    of equation~(\ref{eq:inverseMetricComponents}).
\end{note}
\begin{note}
    Pulling down or up indices is a dangerous business.
    It means we are suppressing the metric and hiding
    that the object depends on the metric.
    Actually it then is not clear if $X_a$ are the
    components of a genuine one form or if it is
    constructed by pulling down the index of the index
    of a vector and hiding the metric.
\end{note}

\begin{example}[(Sphere)]
 $(S^2, \mathcal{O}, \mathcal{A})$
with a chart $(U,x)$, $\phi\in(0,2\pi)$, $\theta\in (0,\pi)$.
\begin{equation}
    g_{ij}\left(x^{-1}(\theta, \phi)\right) = 
    \begin{pmatrix}
        R^2 & 0 \\
        0 & R^2 \sin^2 \theta
    \end{pmatrix}_{ij}\,,
\end{equation}
is the metric of the \textit{round sphere} of radius $R\in\mathbb{R}^+$.
\end{example}

\subsection{Signature}
Remember \textit{linear algebra}:
Eigenvalues and eigenvectors
\begin{equation}
    A^a{}_m v^m = \lambda v^a\,.
    \label{eq:eigenvector}
\end{equation}
How does this translate the notion of eigenvectors to our case of a metric?
$A^a{}_m$ is a (1,1)-tensor and eigenvectors are a good notion,
but for a (0,2)-tensor eq.~(\ref{eq:eigenvector}) does not work
\begin{equation}
    g_{am}v^m \neq \lambda v^a\,.
\end{equation}
\begin{note}
    A (1,1) tensor cannot be symmetric on its own, it can only be symmetric
    with respect to a metric, \textit{i.e.}\ one can pull down an index
    and then switch indices.
\end{note}
\begin{itemize}
    \item A (1,1)-tensor has eigenvalues and can be transformed to look like
        \begin{equation}
            \begin{pmatrix}
                \lambda_1 & & \\
                & \ddots & \\
                & & \lambda_n
            \end{pmatrix} = \diag(\lambda_1, \ldots, \lambda_n)\,,
        \end{equation}
        with eigenvalues $\lambda_i$.
    \item A (0,2)-tensor like the metric has a \textit{signature} $(p,q)$ and can be transformed to
        \begin{equation}
            \diag(\underbrace{1,\dots, 1}_{p}, \underbrace{-1, \ldots, -1}_{q}, 
            \underbrace{0, \ldots, 0}_{\dim V - p - q})\,,
        \end{equation}
        which we can agree has way less information than the eigenvalues.
\end{itemize}
\begin{note}
    The condition that the musical isomorphism $\flat$, eq.~(\ref{eq:musicalMap})
    is invertible means that there are no zeros in the signature. 
    Basically a zero would mean that a whole subspace is mapped to zero and
    this is not invertible.
\end{note}

\begin{defn}[Riemannianian and Lorentzian Metric]
    \begin{itemize}
        \item A metric is called \textit{Riemannian} if its signature is
            $(+, \ldots, +)$ (or $(-, \ldots, -)$ is equivalent).
        \item A metric is called \textit{Lorentzian} if its signature is
            $(+,-, \ldots, -)$ (or $(-,+, \ldots, +)$ is equivalent).
            We will chose $(+,-, \ldots, -)$ here.
            This is what we need for General Relativity.
        \item All other signature including Lorentzian metrics are called
            \textit{pseudo Riemannian}.
    \end{itemize}
\end{defn}
\begin{note}
    One might call a non-Riemannian metric a \textit{pseudo metric}, since
    there are nonzero vectors that have zero length under such a metric.
    In a Lorentzian manifold one says they lie on the light cone.
\end{note}
\begin{note}
    Generally the metric itself will change from point to point in spacetime $\M$,
    but the signature stays.
\end{note}
\subsection{Length of a Curve}
Let $\gamma$ be a smooth curve.
Then we know its velocity $v_{\gamma,\gamma(\lambda)}(f) := (f \circ \gamma)'(\lambda)$
at each $\gamma(\lambda)\in \M$ from definition~\ref{defn:velocity}.

\begin{defn}[Speed of a curve]
    On a \textit{Riemannian metric manifold} $(\M, \mathcal{O}, \mathcal{A}, G)$ the
    \textit{speed} of a curve $\gamma$ at $\gamma(\lambda)$ is the number
    \begin{equation}
        s(\lambda) = \left( \sqrt{g(v_\gamma, v_\gamma)} \right)_{\gamma(\lambda)}\,.
    \end{equation}
    Basically its just the magnitude of the velocity.
\end{defn}
\begin{note}
    I feel the need, the need for \cancel{speed} a metric to define speed.
\end{note}
\begin{note}
    The physical dimensions are
    \begin{align*}
        [v^a] &= \frac{1}{T}\,,\\
        [g_{ab}] &= L^2\,,\\
        [\sqrt{g_{ab}v^av^b}] &= \frac{L}{T}\,.
    \end{align*}
    The idea that coordinate distance has anything to do with real distance is just wrong.
    Going double as far in coordinates has nothing to do with going double as far
    in ``reality'' (the manifold $\M$).
\end{note}

\begin{defn}[Length of a curve]
    Let $\gamma: (0,1)\to \M$ be a smooth curve.
    Then the \textit{length of} $\gamma$ is the number
    \begin{align}
        L[\gamma] &:= \int_0^1 \diff \lambda\, s(\lambda) \\
        &= \int_0^1\diff \lambda\, \sqrt{\left( g(v_\gamma,v_\gamma \right)_{\gamma(\lambda)}}\,.\nonumber
            \label{eq:lengthCurve}
    \end{align}
    It is a functional, \textit{i.e.}\ a function is mapped to a number.
\end{defn}
\begin{note}
    It's exactly the other way than one usually thinks.
    Velocity is more fundamental than speed and speed is more fundamental than
    length.
\end{note}

\begin{example}[(Round sphere of radius $R$)]
Its equator is a curve
\begin{align}
    \theta(\lambda) &:= (x^1\circ \gamma)(\lambda) = \frac{\pi}{2}\,,\\
    \phi(\lambda) &:= (x^2\circ\gamma)(\lambda) = 2\pi \lambda^3\,,
\end{align}
where we have randomly chosen any parametrization that has
$\phi(0) = 0$, $\phi(1) = 2\pi$.
Then the components of the velocity are (eq.~(\ref{eq:vectorcomponents2}))
\begin{align*}
    v^i &= \dot{\gamma}_x^i (0) := (x^i\circ\gamma)'(0)\,,\\
    v^1 &= \left( \frac{\pi}{2} \right)' = 0\,,\\
    v^2 &= \left( 2\pi \lambda^3 \right)' = 6\pi \lambda^2\,.
\end{align*}
and $g_{ij} = \diag(R^2, R^2 \sin^2\theta)$ the length of the curve around the equator
($\theta = \pi/2$, $\sin(\theta) = 1$)
\begin{align*}
    L[\gamma] &= \int_0^1\diff\lambda\,\sqrt{R^2\cdot 0 + R^2 \sin^2(\theta(\lambda))(6\pi\lambda^2)^2}\\
    &= \int_0^1\diff\lambda\, R 6 \pi \lambda^2 = 2\pi R\,,
\end{align*}
or just to have it written down in a rigorous way
\begin{align*}
    &L[\gamma] =\\
    &\int_0^1\diff\lambda\,\sqrt{g_{ij}(x^{-1}(\theta(\lambda),\phi(\lambda))(x^i\circ\gamma)'(\lambda)
        (x^j\circ\gamma)'(\lambda)}
\end{align*}
\end{example}

\begin{theorem}[Reparametrization invariance of the lengh of a curve]
    Let $\gamma: (0,1)\to\M$ be a curve and
    $\sigma: (0,1)\to(0,1)$ a smooth bijection and increasing (don't drive back
    on the curve), then the reparametrized curve has the same length,
    \begin{equation}
        L[\lambda] = L[\gamma\circ\sigma]\,.
    \end{equation}
\end{theorem}

\subsection{Geodesics}
\begin{defn}[Geodesic]
    A curve $\gamma: (0,1) \to \M$ is called a \textit{geodesic} on a
    Riemannian manifold $(\M, \mathcal{O}, \mathcal{A}, g)$ is a \textit{stationary}
    curve with respect to the length functional $L[\gamma]$.
\end{defn}

\begin{theorem}
    A curve $\gamma$ is a \textit{geodesic} iff it satisfies the Euler-Lagrange equations for
    the Lagrangian
    \begin{align}
        \calL: \T \M &\to \mathbb{R}\,,\\
        X &\mapsto \sqrt{g(X,X)}\,.
    \end{align}
    In a chart the Euler-Lagrange equations take the form (chart dependent)
    \begin{equation}
        \frac{\diff}{\diff \lambda}\left( \frac{\partial \calL}{\partial \dot{\gamma}^m} \right) -
        \frac{\partial \calL}{\partial \gamma^m} = 0\,,
        \label{eq:EulerLagrange}
    \end{equation}
    where
    \begin{equation}
        \calL(\gamma^i, \dot{\gamma}^i) = \sqrt{g_{ij}(\gamma (\lambda)) \dot{\gamma}^i(\lambda)
            \dot{\gamma}^j(\lambda)}\,.
            \label{eq:lagrangian}
    \end{equation}
\end{theorem}
Plugging the Lagrangian~(\ref{eq:lagrangian}) into the Euler-Lagrange equations~(\ref{eq:EulerLagrange})
and using the parametrization of $\gamma$ such that $g(\dot\gamma, \dot\gamma)=1$
(always driving at unit speed) we get after raising the index with $(g^{-1})^{qm}$
\begin{align}
    \ddot{\gamma}^q + 
    \overbrace{\frac{1}{2}\left( g^{-1} \right)^{qm} \left( \partial_i g_{mj} + \partial_j g_{mi}
    - \partial_m g_{ij} \right)}^{:= ^\text{L.C.}\Gamma^q{}_{ij}(\gamma(\lambda))} \dot{\gamma}^i \dot{\gamma}^j = 0\,.
    \label{eq:geodesic}
\end{align}
Equation~(\ref{eq:geodesic}) is the \textit{geodesic equation} for $\gamma$.
We call $^\text{L.C.}\Gamma^q{}_{ij}$ the \textit{Levi-Civita connection coefficient functions}.
\begin{defn}[Levi-Civita connection]
    The \textit{Levi-Civita connection} coefficient functions $^\text{L.C.}\Gamma^q{}_{ij}(\gamma(\lambda))$
    (also called \textit{Christoffel symbols} or ``Christ awful symbols'' because of the labour needed 
    to calculate them)
    of the so called \textit{Levi-Civita connection} $^\text{L.C.}\nabla$ and they are
    \begin{equation}
        \boxed{%
        ^\text{L.C.}\Gamma^q{}_{ij} = 
        \frac{1}{2}\left( g^{-1} \right)^{qm} \left( \partial_i g_{mj} + \partial_j g_{mi}
        - \partial_m g_{ij} \right)
    }
    \end{equation}
\end{defn}
\begin{note}
    If we use the Levi-Civita connection as the connection on our manifold,
    then the geodesic equation
    \begin{equation}
        \ddot{\gamma}^q + \Gamma^q{}_{ij}\dot{\gamma}^i \dot{\gamma}^j = 0\,,
    \end{equation}
    is exactly the equation~(\ref{eq:autoparallelEquation}) for an autoparallelly transported curve,
    \textit{i.e.}\ for a curve that is as straight as possible.
    Choice of the connection as Christoffel conection thus means we identify autoparallelly transported
    curves (straight as possible) with the shortest curves.
\end{note}
\begin{note}
    Thus a metric manifold $(\M, \calO, \calA, g)$ implies a manifold with the Christoffel connection
    \begin{equation}
        (\M, \calO, \calA, g) \to (\M, \calO, \calA, g,~^\text{L.C.}\nabla)\,,
    \end{equation}
    and we usually make this choice of connection.
\end{note}
\begin{note}
    If for a metric manifold $(\M, \calO, \calA, g)$ one imposes
    \begin{enumerate}
        \item \textit{Metric compatibility:} $\nabla g = 0$,
        \item \textit{Absence of torsion:} $T=0$,
    \end{enumerate}
    then the connection is already fixed to be the Levi Civita connection
    $\nabla = {}^\text{L.C.}\nabla$.
    This is the way many General Relativity textbooks go.
    They impose metric compatibility and write the equation $\nabla_i g_{ab}$ in
    three permutations, add them in some way and find an expression for $\nabla$.
    Sadly they usually don't talk about implicitly identifying autoparallelly transported
    curves with the shortest curves by doing this.
\end{note}
\begin{defn}[Riemann-Christoffel Curvature]
    The \textit{Riemann-Christoffel Curvature} $R_{abcd}$ of a manifold
    $(\M, \calO, \calA, g)$ is defined by (in coordinates)
    \begin{equation}
        R_{abcd} := g_{am} R^{m}{}_{bcd}\,,
    \end{equation}
where the connection used to calculate the Riemann tensor $R^m{}_{bcd}$ is the Levi-Civita connection.
\end{defn}
\begin{note}
    In contrast to the Riemann curvature~(\ref{eq:RiemannCurvature}), 
    which only needs a connection, the Riemann-Christoffel cuvature also needs a metric.
\end{note}
\begin{defn}[Ricci Tensor]
    The \textit{Ricci tensor} $R_{ab}$ is defined by (in coordinates)
    \begin{equation}
        R_{ab} := R^m{}_{amb}\,,
    \end{equation}
    where again for the connection to calculate $R^c{}_{amb}$ the Levi-Civita connection is used.
\end{defn}
\begin{defn}[Ricci Scalar Curvature]
    The \textit{Ricci scalar curvature} is
    \begin{equation}
        R := g^{ab}R_{ab}\,,
    \end{equation}
    where we have introduced the \textit{convention}
    \begin{equation}
        g^{ab} := (g^{-1})^{ab} \,.
    \end{equation}
\end{defn}
\begin{defn}[Einstein Curvature]
    The \textit{Einstein Curvature} $G_{ab}$ is defined by
    \begin{equation}
        G_{ab} := R_{ab} - \frac{1}{2} g_{ab} R\,.
    \end{equation}
\end{defn}

