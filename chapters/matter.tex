\section{Matter}
\subsection{Point Matter}
Our postulates~\ref{item:massive} and~\ref{item:massless} already constrain the possible particle worldlines.
But what is the precise law of motion, possibly in the presence of ``forces''?

\paragraph{Without external forces:}
\begin{equation}
    S_\text{massive}[\gamma] := m\int\diff\lambda\sqrt{g_{\gamma(\lambda)} ( v_{\gamma, \gamma(\lambda)} v_{\gamma, \gamma(\lambda)})}\,,
\end{equation}
with particles going forward in time, \textit{i.e.}
\begin{equation}
    g_{\gamma(\lambda)} \left( T_{\gamma(\lambda)}, v_{\gamma, \gamma(\lambda)} \right) > 0\,,
\end{equation}
The dynamical law then follows from the Euler-Lagrange equations of $S_\text{matter}$ or equivalently
from the variation of $S_\text{matter}$ with respect to the curve $\gamma$.
Similarly for massless particles
\begin{equation}
    S_\text{massive}[\gamma] := \int\diff\lambda\, \mu g_{\gamma(\lambda)} ( v_{\gamma, \gamma(\lambda)} v_{\gamma, \gamma(\lambda)})\,,
    \label{eq:massiveAction}
\end{equation}
where $\mu$ is a Lagrange multiplier and its Euler Lagrange equation enforces
\begin{equation}
    g\left( v_{\gamma, \gamma(\lambda)}, v_{\gamma, \gamma(\lambda)} \right) = 0\,,
    \label{eq:masslessAction}
\end{equation}
so it has to be a massless particle, postulate~\ref{item:massless}.
\begin{note}
    Note that there is no square root for the massless particle~\ref{eq:masslessAction}.
    For a massless particle the world line cannot parametrized with respect to time,
    because time does not run along it.
    Actually this is a meaningless statement, because time was only defined for
    \textit{observers} and an observer was defined for timelike curves, not null curves.
    (timelike curve means the velocity vector is timelike everywhere, in our convention
    the norm must be positive. A null/lightlike vector has norm zero.)
\end{note}

\begin{note}
    The reason for describing equations of motions by actions is that compositve systems have
    an action that is the sum of the actions of the parts of that system plus possibly
    \textit{interaction terms.}
    Example:
    \begin{equation}
        S[\gamma] + S[\delta] + S_\text{int}[\gamma,\delta]\,.
    \end{equation}
\end{note}

\paragraph{With external forces:}
or rather: Presence of \textit{fields} to which a particle couples, \textit{i.e.}\ there
is an interaction therm with the field in the action.

\begin{example}[(Coupling to Electromagnetic Field)]
    \begin{align*}
        S[\gamma;A] &= \int\diff\lambda \left(m\sqrt{g_{\gamma(\lambda)} ( v_{\gamma, \gamma(\lambda)} v_{\gamma, \gamma(\lambda)})}\right.\\
        &~+\underbrace{q A(v_{\gamma, \gamma(\lambda)})}_{L_\text{int}}\left.\vphantom{\sqrt{g_{\gamma(\lambda)}}}\right)\,,
    \end{align*}
    where $A$ is a \textit{covector field} on $\M$ (electromagnetic field).
    In $S[\gamma;A]$ the semicolon means that $A$ is fixed and the action should not be varied with respect to it.

    Now we can use the Euler-Lagrange equations to get the equations of motion.
    It is easier to first write out the Lagrangian in a chart.
    \begin{equation*}
        L = m \sqrt{g_{ab}\dot{\gamma}^a\dot{\gamma}^b} + A_c(\gamma(\lambda)) \dot{\gamma}(\lambda)^c\,.
    \end{equation*}
    We also know that the E.L.\ equations for the first part give $m\nabla_{v_{\gamma, \gamma(\lambda)}}
    v_{\gamma, \gamma(\lambda)}$ or in a chart $m \dot{\gamma}^a\nabla_a \dot{\gamma}_m$.
    Then,
    \begin{align*}
        \frac{\diff}{\diff \lambda}\frac{\partial L}{\partial \dot{\gamma}^m} &= 
        m \dot{\gamma}^a\nabla_a \dot{\gamma}_m + \frac{\partial A_m}{\partial \gamma^n}\dot{\gamma}^n\,,\\
        \frac{\partial L}{\partial \gamma^m} &= \frac{\partial A_n}{\partial \gamma^m}\dot{\gamma}^n\,,
    \end{align*}
    and so
    \begin{align*}
        0 &= \frac{\diff}{\diff \lambda}\frac{\partial L}{\partial \dot{\gamma}^m} - 
        \frac{\partial L}{\partial \gamma^m}\\
        &= 
        m \dot{\gamma}^a\nabla_a \dot{\gamma}_m + \frac{\partial A_m}{\partial \gamma^n}\dot{\gamma}^n
        - \frac{\partial A_n}{\partial \gamma^m}\dot{\gamma}^n\,.
    \end{align*}
    In general, I think $\partial/\partial\gamma^i$ should be $\partial/\partial x^i$, because in the
    E.L.\ equations it should also be this way and in the end overall $x^i \to \gamma^i$,
    \textit{i.e.}\ in the end it's evaluated on the curve.
    Finally with $F_{ab} = \partial_a A_b - \partial_b A_a$ we arrive at
    \begin{equation}
        \boxed{%
        m \left( \nabla_{v_\gamma}v_\gamma \right)^a = -q F^a{}_m \dot{\gamma}^m
    }\,,
    \end{equation}
    or coordinate free
    \begin{equation}
        \boxed{%
        m \left( \nabla_{v_\gamma}v_\gamma \right)^a =
        -q \flat^{-1}\left( F(\cdot, v_\gamma) \right)
    }\,,
    \end{equation}
    the \textit{Lorentz force} on a charged particle.

    The coupling to the electromagnetic field like in the action above exactly yields the
    Lorentz force law.
    This is the reason why it is the right coupling.
    Note that The action is still reparametrization invariant, because
    $A(v_\gamma)\diff\lambda$ also is.
\end{example}

\subsection{Field Matter}
\begin{defn}[Classical Field Matter]
    \textit{Classical field matter} is any tensor field on spacetime whose equations
    of motion derive from an action.
    Classical here refers to non-quantum.
\end{defn}
\begin{example}[(Maxwell Theory)]
    For simplicity assume the whole manifold $\M$ is covered by one chart.
    \begin{equation}
        S_\text{Maxwell}[A;g] = \frac{1}{4} \int_{\M} \diff x\,\underbrace{\sqrt{-g} F_{ab}F_{cd}g^{ac}g^{bd}}_\mathcal{L}\,.
    \end{equation}
    The minus sign by the determinant is because of the metric signature $(+---)$.
    The spacetime metric $g$ is fixed for now.
    Again, $F_{ab} = \partial a A_b - \partial_b A_a = 2 \partial_{[a}A_{b]} = 2 (\nabla_{[a}A)_{b]}$.
    The Euler-Lagrange equations are
    \begin{equation}
        0 = \frac{\partial \mathcal{L}}{A_m} 
        - \frac{\partial}{\partial x^s}\left( \frac{\partial \mathcal{L}}{\partial(\partial_s A_m)} \right)\,.
    \end{equation}
    After some calculation one arrives at
    \begin{equation}
        \boxed{%
        \left( \nabla_{\frac{\partial}{\partial x^m}} F \right)^{ma} = j^a
    }\,,
    \end{equation}
    the \textit{inhomogeneous} Maxwell equations.
    The current $j$ is zero for our case, but if we had a coupling $\sqrt{-g} A(j)$ in the Lagrangian
    it would be nonzero. For \textit{e.g.}\ a point particle it would be $j = q v_\gamma$.
    The \textit{homogeneous} Maxwell equations are
    \begin{equation}
        \partial_{[a}F_{bc} = \left( \nabla_{[a}F \right)_{bc]} = 0\,.
    \end{equation}
    In components the equations would be
    \begin{align*}
        \nabla_m F^{ma} &= j^a\,,\\
        \epsilon^{abcd}\nabla_a F_{bc} &= 0\,.
    \end{align*}
\end{example}

Another example that is well-liked by textbooks is the \textit{Klein-Gordon} field.
\begin{example}[(Klein-Gordon Field)]
    \begin{equation*}
        S_\text{K.G.}[\phi] := \int_m\difff^4\!x\sqrt{-g} \left[ 
        g^{ab}(\partial_a \phi) (\partial_b \phi) - m^2 \phi^2 \right]\,,
    \end{equation*}
    where $\phi$ is a (0,0)-tensor field.
\end{example}
