\section{Matter}
\subsection{Point Matter}
Our postulates~\ref{item:massive} and~\ref{item:massless} already constrain the possible particle worldlines.
But what is the precise law of motion, possibly in the presence of ``forces''?

\paragraph{Without external forces:}
\begin{equation}
    S_\text{massive}[\gamma] := m\int\diff\lambda\sqrt{g_{\gamma(\lambda)} ( v_{\gamma, \gamma(\lambda)} v_{\gamma, \gamma(\lambda)})}\,,
\end{equation}
with particles going forward in time, \textit{i.e.}
\begin{equation}
    g_{\gamma(\lambda)} \left( T_{\gamma(\lambda)}, v_{\gamma, \gamma(\lambda)} \right) > 0\,,
\end{equation}
The dynamical law then follows from the Euler-Lagrange equations of $S_\text{matter}$ or equivalently
from the variation of $S_\text{matter}$ with respect to the curve $\gamma$.
Similarly for massless particles
\begin{equation}
    S_\text{massive}[\gamma] := \int\diff\lambda\, \mu g_{\gamma(\lambda)} ( v_{\gamma, \gamma(\lambda)} v_{\gamma, \gamma(\lambda)})\,,
    \label{eq:massiveAction}
\end{equation}
where $\mu$ is a Lagrange multiplier and its Euler Lagrange equation enforces
\begin{equation}
    g\left( v_{\gamma, \gamma(\lambda)}, v_{\gamma, \gamma(\lambda)} \right) = 0\,,
    \label{eq:masslessAction}
\end{equation}
so it has to be a massless particle, postulate~\ref{item:massless}.
\begin{note}
    Note that there is no square root for the massless particle~\ref{eq:masslessAction}.
    For a massless particle the world line cannot parametrized with respect to time,
    because time does not run along it.
    Actually this is a meaningless statement, because time was only defined for
    \textit{observers} and an observer was defined for timelike curves, not null curves.
    (timelike curve means the velocity vector is timelike everywhere, in our convention
    the norm must be positive. A null/lightlike vector has norm zero.)
\end{note}

\begin{note}
    The reason for describing equations of motions by actions is that compositve systems have
    an action that is the sum of the actions of the parts of that system plus possibly
    \textit{interaction terms.}
    Example:
    \begin{equation}
        S[\gamma] + S[\delta] + S_\text{int}[\gamma,\delta]\,.
    \end{equation}
\end{note}

\paragraph{With external forces:}
or rather: Presence of \textit{fields} to which a particle couples, \textit{i.e.}\ there
is an interaction therm with the field in the action.

\begin{example}
    \begin{align*}
        S[\gamma;A] &= \int\diff\lambda \left(m\sqrt{g_{\gamma(\lambda)} ( v_{\gamma, \gamma(\lambda)} v_{\gamma, \gamma(\lambda)})}\right.\\
        &~+\underbrace{q A(v_{\gamma, \gamma(\lambda)})}_{L_\text{int}}\left.\vphantom{\sqrt{g_{\gamma(\lambda)}}}\right)\,,
    \end{align*}
    where $A$ is a \textit{covector field} on $\M$ (electromagnetic field).
    In $S[\gamma;A]$ the semicolon means that $A$ is fixed and the action should not be varied with respect to it.
\end{example}

