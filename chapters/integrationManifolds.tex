\section{Integration on Manifolds}
Now we do the completion of our ``lift'' of analysis on the charts to the manifold level.
We want to define
\begin{equation}
    \int_\M f\,,
\end{equation}
and this requires a mild new structure called the \textit{volume form}
and we need to restrict the atlas a little bit (``orientation'').

\subsection[Review of Integration]{Review of Integration on $\mathbb{R}^d$}
\begin{enumerate}
    \item A function $F: \mathbb{R} \to \mathbb{R}$. Assume a notion of integration is known
        (Riemann, Lebesque):
        \begin{equation}
            \int_{(a,b)}F := \int_a^b \diff x\, F(x)\,.
        \end{equation}
    \item $F: \mathbb{R}^d \to \mathbb{R}$.
        On a box-shaped domain
        $I = (a,b)\times(c,d)\times\cdots\times (u,v) \subseteq \mathbb{R}^d$,
        \begin{equation}
            \int_{I} \difff^d\! x\, F(x) := \int_{(a,b)}\diff x^1 \cdots \int_{(u,v)}\diff x^d\,.
        \end{equation}
    \item Other domains $G\subseteq \mathbb{R}^d$:
        \begin{defn}[Indicator function]
            \begin{align}
                \nonumber \mu_G: \mathbb{R}^d &\to \mathbb{R}\,,\\
                x&\mapsto
                \begin{cases}
                    1\,, & \text{if } x\in G\\
                    0\,, & \text{if } x\not\in G
                \end{cases}
            \end{align}
        \end{defn}
        and then define
        \begin{equation}
            \int_G \difff^d\! x F(x):= \int_{-\infty}^\infty \diff x^1 \cdots \int_{-\infty}^\infty \diff x^d \mu_G(x) F(x)\,,
        \end{equation}
        if it exists.
\end{enumerate}
\paragraph{Change of variables}
\begin{center}
    \begin{tikzpicture}
        \matrix (m) [matrix of math nodes,row sep=3em,column sep=4em,minimum width=2em]
        {%
            \mathbb{R}^d \supseteq \mathrm{preim}_\phi(G) & G \subseteq \mathbb{R}^d \\
            ~ & \mathbb{R}\\
        };
        \path[-stealth]
        (m-1-1) edge node [above] {$\phi$} (m-1-2)
        (m-1-2) edge node [right] {$F$} (m-2-2)
        (m-1-1) edge node [below] {$F\circ \phi$} (m-2-2);
            %(m-1-1) edge node [above] {$\phi_*$} node [below] {push-forward} (m-1-2)
            %(m-2-2) edge node [above] {$\phi^*$} node [below] {pull-back} (m-2-1);
    \end{tikzpicture}
\end{center}
\begin{equation}
    \int_G \difff^d\! x\, F(x) = \int\limits_{\mathclap{\mathrm{preim}_\phi(G)}}\difff^d y
    \left|\det(\partial_\cdot \phi^\cdot)(y) \right| (F\circ \phi)(y)\,.
\end{equation}

\subsection{Integration on a Chart}
Let $(\M, \calO, \calA)$ be a smooth manifold and $f: M\to\mathbb{R}$.
Choose two charts $(U,x)\in \calA \ni (U,y)$.
\begin{center}
    \begin{tikzpicture}
        \matrix (m) [matrix of math nodes,row sep=3em,column sep=4em,minimum width=2em]
        {%
            y(u)\subseteq \mathbb{R}^d & \\
            U & \mathbb{R} \\
            x(U) & \\
        };
        \path[-stealth]
        (m-1-1) edge node [above, rotate=-32, scale=0.7] {$f_{(y)}:=f\circ y^{-1}$} (m-2-2)
        (m-2-1) edge node [right] {$y$} (m-1-1)
        (m-2-1) edge node [right] {$x$} (m-3-1)
        (m-2-1) edge node [above] {$f$} (m-2-2)
        (m-3-1) edge node [below, rotate=32, scale=0.7] {$f_{(x)}:=f\circ x^{-1}$} (m-2-2)
        (m-3-1) edge[bend left = 60] node [sloped, above, rotate=180, pos=0.5] {$\phi=y\circ x^{-1}$} (m-1-1);
            %(m-1-1) edge node [above] {$\phi_*$} node [below] {push-forward} (m-1-2)
            %(m-2-2) edge node [above] {$\phi^*$} node [below] {pull-back} (m-2-1);
    \end{tikzpicture}
\end{center}
The naive way of integration, namely just integrating $f_{(y)}$ over $y(U)$, \textit{i.e.}\
just integrating on a chart, \textit{does not work}, because it depends on the chart.
We can see it by transforming from chart $y$ to chart $x$.
\begin{align}
    \nonumber&\int\limits_{\mathclap{y(U)}} \difff^d\! \beta f_{(y)} (\beta)= \\
    \nonumber&=\int\limits_{\mathclap{x(U)}} \difff^d\! \alpha \left| \det(\partial_a (y^b\circ x^{-1})(\alpha) \right| \left( f_{(y)} \circ (y\circ x^{-1} \right)(\alpha)\\
            \nonumber&=\int\limits_{\mathclap{x(U)}} \difff^d\! \alpha \left| \det\left( \frac{\partial y^b}{\partial x^a} \right)_{x^{-1}(\alpha)} \right| f_{(x)}(\alpha)\\
            &\neq \int\limits_{\mathclap{x(U)}} \difff^d\! \alpha f_{(x)} (\alpha)\,.
            \label{eq:naiveIntegral}
        \end{align}
        So we need to put something into the integral that transforms with the inverse of the factor that is too much.

        \subsection{Volume Forms} 
        \begin{defn}[Volume Form]
            On a smooth manifold $(\M, \calO, \calA)$ a $(0, \dim \M)$-tensor field
            $\Omega$ is called a \textit{volume form} if
            \begin{enumerate}
                \item $\Omega$ vanishes nowhere,
                \item $\Omega$ is totally antisymmetric:
                    \begin{equation}
                        \Omega(\ldots, X, \ldots, Y, \ldots) = - \Omega(\ldots, Y, \ldots, X, \ldots)\,,
                    \end{equation}
                    for any vectors $X, Y$ in any of the positions of $\Omega$.
                    In a chart we write
                    \begin{equation}
                        \Omega_{i_1\cdots i_d} = \Omega_{[i_1\cdots i_d]}\,.
                    \end{equation}
            \end{enumerate}
        \end{defn}

        On a metric manifold $(\M, \calO, \calA^\uparrow, g)$ one can construct a volume for $\Omega$ from
        the metric $g$.
        In any chart:
        \begin{equation}
            \Omega_{i_1\cdots i_d} := \sqrt{\det(g_{(x)ij}} \epsilon_{i_1\cdots i_d}\,,
            \end{equation}
            where $\epsilon_{i_1\cdots i_d}$ is the \textit{Levi-Civita symbol} and defined as
            \begin{align}
                \epsilon_{123\cdots d} &= 1\,,\\
                \epsilon_{i_1\cdots i_d} &= \epsilon_{[i_1\cdots i_d]}\,.
            \end{align}
            It is not a tensor, just see it as a symbol!

            For $\Omega$ to be well defined we actually need to assume the atlas to be an \textit{oriented}
            atlas $\calA^\uparrow$.
            The reason is the following:
            Transforming $\Omega$ that comes form $g$ should give a factor that cancels the problematic factor
            we had in eq.~(\ref{eq:naiveIntegral}), but
            \begin{align}
                \label{eq:omegaTranform}&\Omega_{(y)i_1\cdots i_d} = \sqrt{\det(g_{(y)ij})}\, \epsilon_{i_1\ldots i_d}\\
                \nonumber&= \sqrt{\det\left( g_{(x)mn}\frac{\partial x^m}{\partial y^i}\frac{\partial x^n}{\partial y^j} \right)}
                \epsilon_{m_1\cdots m_d}
            \end{align}
            where now one has to be careful with the transformations.
            The transformation in the metric has $\partial x / \partial y$, which is just the way the metric transforms.
            On the other hand for the Levi-Civita symbol we employ a trick now:
            This is for example described in~\cite{Carroll}:
            The determinant $\det(A)$ of a matrix $A^m_n$ is
            \begin{equation}
                \det(A^m_n) \epsilon_{i_1\cdots i_d} = \epsilon_{m_1\cdots m_d}A^{m_1}_{i_1}\cdots A^{m_d}_{i_d}\,,
            \end{equation}
            Then setting $A^m_n = \frac{\partial x^m}{\partial y^n}$ we get
            \begin{equation}
                \epsilon_{i_1\cdots i_d} = \det\left( \frac{\partial y^n}{\partial x^m} \right)
                \epsilon_{m_1\cdots m_d} \frac{\partial x^{m_1}}{\partial y^{i_1}}\cdots \frac{\partial x^{m_d}}{\partial y^{i_d}}\,,
            \end{equation}
            which is of course just an identity and not really a transformation, since the Levi-Civita symbol is the same in
            every chart.
            It however looks like a transformation.
            Plugging this into equation~(\ref{eq:omegaTranform})
            \begin{align}
                \nonumber &\sqrt{\det\left( g_{(x)mn}\frac{\partial x^m}{\partial y^i}\frac{\partial x^n}{\partial y^j} \right)}
                \frac{\partial y^{m_1}}{\partial x^{i_1}}\cdots \frac{\partial y^{m_d}}{\partial x^{i_d}}
                \det\left( \frac{\partial y^a}{\partial x^b} \right) \epsilon_{m_1\cdots m_d} \MoveEqLeft[2]\\
                \nonumber &= 
                \sqrt{\det g_{(x)}} \left| \det \left( \frac{\partial x^c}{\partial y^d} \right)\right|
                \det\left( \frac{\partial y^a}{\partial x^b} \right)
                \frac{\partial x^{m_1}}{\partial y^{i_1}}\cdots \frac{\partial x^{m_d}}{\partial y^{i_d}}
                \epsilon_{m_1\cdots m_d}\\
                \nonumber&=  \sqrt{\det g_{(x)}} 
                \mathrm{sgn}\left( \det \frac{\partial x^c}{\partial y^d} \right) 
                \det \left( \frac{\partial x^a}{\partial y^b}\right)
                \epsilon_{i_1\cdots i_d}\\
                &= 
                \det \left( \frac{\partial x^a}{\partial y^b}\right)
                \mathrm{sgn}\left( \det \frac{\partial x^c}{\partial y^d} \right) 
                \Omega_{(x)i_1\cdots i_d}\,,
                \label{eq:OmegaTransform}
            \end{align}
            where we now see that the first term is exactly the inverse of the term that was too much in
            the naive integral eq.~(\ref{eq:naiveIntegral}),
            but we have the sign of the determinant extra.\footnote{I am not sure about what I did here, but I think
            Schuller made a mistake in his lecture and the way I did it seems correct to me.}
            I actually don't see the problem, because in~(\ref{eq:naiveIntegral}) we also have an
            absolute value sign which should make it okay again?
            Nevertheless, Schuller states that we want the sign to be 1,
            restrict ourselves to an \textit{oriented atlas}, which we denote by
            $\calA^\uparrow$.

            \begin{defn}[Oriented Atlas $\calA^\uparrow$]
                An atlas $\calA$ such that any two charts $(U,x)$, $(V,y)$ have
                a chart transition map $y\circ x^{-1}$ with
                \begin{equation}
                    \det\left( \frac{\partial y^a}{\partial x^b} \right) > 0\,,
                \end{equation}
                is called an oriented atlas and denoted with $\calA^\uparrow$.
            \end{defn}

            We introduce the \textit{Levi-Civita symbol} with upper indices $\epsilon^{i_1\cdots i_d}$ with
            exactly the same values as lower indices (no pulling up the indices with metric)
            and then define the scalar density
            \begin{defn}[$\omega_{(y)}$]
                Let $\Omega$ be a volume form on $(\M,\calO, \calA^\uparrow)$
                and a chart $(U,x)$, then
                \begin{equation}
                    \omega_{(x)}(p) := \Omega_{i_1\cdots i_d}(p)\epsilon^{i_1\cdots i_n}\,.
                \end{equation}
            \end{defn}
            Looking at the transformation of $\Omega$ in equation~(\ref{eq:OmegaTransform})
            we directly see that (for an oriented atlas, which is already needed for the definition
            of $\Omega)$
            \begin{equation}
                \boxed{%
                    \omega_{(y)} = \det\left( \frac{\partial x^a}{\partial y^b} \right) \omega_{(x)}\,.
                }
            \end{equation}
            An object with such a transformation behaviour is called a \textit{scalar density}.

            \subsection{Integration on One Chart Domain}

            \begin{defn}[Integration on one chart domain $U$]
                On a chart $(U,x)$
                \begin{equation}
                    \int_U f := \int\limits_{x(U)} \difff^d\!\alpha \omega_{(x)}\left( x^{-1}(\alpha) \right) f_{(x)}(\alpha)\,.
                \end{equation}
            \end{defn}
            From the transformation behaviour of $\omega$ and the integral (Jacobian) we see that this definition is
            independent of the chart.
            On an oriented metric manifold $(\M, \calO, \calA^\uparrow, g)$
            \begin{equation}
                \int_U f:= \int\limits_{x(U)} \difff^d\!\alpha \underbrace{\sqrt{\det(g_{(x)ij}(x^{-1}(\alpha))}}_{\sqrt{g}} f_{(x)}(\alpha)\,.
                \end{equation}

                \subsection{Integration on the Entire Manifold}

                We need to require that the manifold admits a so-called \textit{partition of unity}.
                Roughly: For any finite subatlas $\calA'\subseteq \calA^\uparrow$
                $\exists$ continuous functions
                \begin{equation}
                    \rho_i: U_i \to \mathbb{R}\,,
                \end{equation}
                such that
                \begin{equation}
                    \sum\limits_{\mathclap{p\in U_i}} \rho_i(p) = 1\,,\quad\forall p\in \M\,.
                \end{equation}
                This is called a \textit{partition of unity}.
                The point is that the subatlas needs to be finite in order to not have problems
                with convergence.
                Then we define
                \begin{equation}
                    \int\limits_{\M} f = \sum\limits_{i=1}^{\text{finite}}\int\limits_{U_i}\rho_i f\,.
                \end{equation}

                An example of a partition of unity can be seen in figure~\ref{fig:partitionOfUnity}

                \begin{figure}[tbh]
                    \centering\def\svgwidth{\columnwidth}
                    %LaTeX with PSTricks extensions
%%Creator: Inkscape 1.0beta1 (32d4812, 2019-09-19)
%%Please note this file requires PSTricks extensions
\psset{xunit=.5pt,yunit=.5pt,runit=.5pt}
\begin{pspicture}(793.7007874,1122.51968504)
{
\newrgbcolor{curcolor}{0 0 0}
\pscustom[linewidth=0.99999871,linecolor=curcolor]
{
\newpath
\moveto(62.85714142,1032.51968504)
\lineto(748.57141417,1032.51968504)
}
}
{
\newrgbcolor{curcolor}{0 0 0}
\pscustom[linewidth=0.99999871,linecolor=curcolor]
{
\newpath
\moveto(62.85714142,1002.5196926)
\lineto(467.14284094,1005.37683402)
}
}
{
\newrgbcolor{curcolor}{0 0 0}
\pscustom[linewidth=0.99999871,linecolor=curcolor]
{
\newpath
\moveto(297.1428548,1061.09111433)
\lineto(747.14286614,1061.09111433)
}
}
{
\newrgbcolor{curcolor}{0 0 0}
\pscustom[linestyle=none,fillstyle=solid,fillcolor=curcolor,opacity=0]
{
\newpath
\moveto(389.99998198,879.65023491)
\curveto(389.99998198,860.71475835)(374.01015168,845.36452166)(354.28569644,845.36452166)
\curveto(334.56124119,845.36452166)(318.57141089,860.71475835)(318.57141089,879.65023491)
\curveto(318.57141089,898.58571147)(334.56124119,913.93594817)(354.28569644,913.93594817)
\curveto(374.01015168,913.93594817)(389.99998198,898.58571147)(389.99998198,879.65023491)
\closepath
}
}
{
\newrgbcolor{curcolor}{0 0 0}
\pscustom[linestyle=none,fillstyle=solid,fillcolor=curcolor,opacity=0]
{
\newpath
\moveto(467.1426071,1005.3967729)
\curveto(467.11188753,1008.05745175)(469.24242269,1010.24001534)(471.90306835,1010.27348631)
\curveto(474.56371401,1010.30695728)(476.74847963,1008.17868025)(476.78470195,1005.51807062)
\curveto(476.82092427,1002.85746099)(474.69490764,1000.6704957)(472.03433689,1000.63152206)
\curveto(469.37376615,1000.59254842)(467.18460351,1002.71630237)(467.14287859,1005.37683139)
}
}
{
\newrgbcolor{curcolor}{0 0 0}
\pscustom[linewidth=0.99999871,linecolor=curcolor]
{
\newpath
\moveto(467.1426071,1005.3967729)
\curveto(467.11188753,1008.05745175)(469.24242269,1010.24001534)(471.90306835,1010.27348631)
\curveto(474.56371401,1010.30695728)(476.74847963,1008.17868025)(476.78470195,1005.51807062)
\curveto(476.82092427,1002.85746099)(474.69490764,1000.6704957)(472.03433689,1000.63152206)
\curveto(469.37376615,1000.59254842)(467.18460351,1002.71630237)(467.14287859,1005.37683139)
}
}
{
\newrgbcolor{curcolor}{0 0 0}
\pscustom[linestyle=none,fillstyle=solid,fillcolor=curcolor,opacity=0]
{
\newpath
\moveto(287.11299304,1060.97098396)
\curveto(287.08227346,1063.63166281)(289.21280862,1065.81422641)(291.87345429,1065.84769737)
\curveto(294.53409995,1065.88116834)(296.71886556,1063.75289131)(296.75508788,1061.09228168)
\curveto(296.79131021,1058.43167205)(294.66529358,1056.24470676)(292.00472283,1056.20573312)
\curveto(289.34415208,1056.16675948)(287.15498944,1058.29051343)(287.11326453,1060.95104245)
}
}
{
\newrgbcolor{curcolor}{0 0 0}
\pscustom[linewidth=0.99999871,linecolor=curcolor]
{
\newpath
\moveto(287.11299304,1060.97098396)
\curveto(287.08227346,1063.63166281)(289.21280862,1065.81422641)(291.87345429,1065.84769737)
\curveto(294.53409995,1065.88116834)(296.71886556,1063.75289131)(296.75508788,1061.09228168)
\curveto(296.79131021,1058.43167205)(294.66529358,1056.24470676)(292.00472283,1056.20573312)
\curveto(289.34415208,1056.16675948)(287.15498944,1058.29051343)(287.11326453,1060.95104245)
}
}
{
\newrgbcolor{curcolor}{0 0 0}
\pscustom[linewidth=0.99999871,linecolor=curcolor]
{
\newpath
\moveto(62.85714142,798.26072693)
\lineto(467.14287874,798.26072693)
}
}
{
\newrgbcolor{curcolor}{0 0 0}
\pscustom[linewidth=1.26569954,linecolor=curcolor]
{
\newpath
\moveto(297.14284724,597.24037795)
\lineto(749.53319055,597.24037795)
\lineto(747.99242835,597.24037795)
}
}
{
\newrgbcolor{curcolor}{0 0 0}
\pscustom[linewidth=1.00157475,linecolor=curcolor]
{
\newpath
\moveto(62.85714142,798.26073449)
\lineto(62.85714142,978.06787276)
}
}
{
\newrgbcolor{curcolor}{0 0 0}
\pscustom[linestyle=none,fillstyle=solid,fillcolor=curcolor]
{
\newpath
\moveto(62.85714142,968.05212527)
\lineto(66.86344041,964.04582627)
\lineto(62.85714142,978.06787276)
\lineto(58.85084242,964.04582627)
\closepath
}
}
{
\newrgbcolor{curcolor}{0 0 0}
\pscustom[linewidth=1.06834643,linecolor=curcolor]
{
\newpath
\moveto(62.85714142,968.05212527)
\lineto(66.86344041,964.04582627)
\lineto(62.85714142,978.06787276)
\lineto(58.85084242,964.04582627)
\closepath
}
}
{
\newrgbcolor{curcolor}{0 0 0}
\pscustom[linewidth=1.00157103,linecolor=curcolor]
{
\newpath
\moveto(297.14284724,597.24037795)
\lineto(297.14284724,777.04751244)
}
}
{
\newrgbcolor{curcolor}{0 0 0}
\pscustom[linestyle=none,fillstyle=solid,fillcolor=curcolor]
{
\newpath
\moveto(297.14284724,767.03180212)
\lineto(301.14913137,763.02551799)
\lineto(297.14284724,777.04751244)
\lineto(293.13656312,763.02551799)
\closepath
}
}
{
\newrgbcolor{curcolor}{0 0 0}
\pscustom[linewidth=1.06834247,linecolor=curcolor]
{
\newpath
\moveto(297.14284724,767.03180212)
\lineto(301.14913137,763.02551799)
\lineto(297.14284724,777.04751244)
\lineto(293.13656312,763.02551799)
\closepath
}
}
{
\newrgbcolor{curcolor}{0 0 0}
\pscustom[linewidth=0.99999871,linecolor=curcolor]
{
\newpath
\moveto(62.85714142,896.24551937)
\lineto(297.14284724,896.24551937)
}
}
{
\newrgbcolor{curcolor}{0 0 0}
\pscustom[linewidth=0.99999871,linecolor=curcolor]
{
\newpath
\moveto(467.14287874,686.13380787)
\lineto(748.57141417,686.13380787)
}
}
{
\newrgbcolor{curcolor}{0 0 0}
\pscustom[linewidth=0.99999871,linecolor=curcolor]
{
\newpath
\moveto(297.14284724,896.24551937)
\lineto(467.14287874,798.09595843)
}
}
{
\newrgbcolor{curcolor}{0 0 0}
\pscustom[linewidth=0.99999871,linecolor=curcolor]
{
\newpath
\moveto(467.14287874,686.13380787)
\lineto(297.14284724,597.24037795)
}
}
{
\newrgbcolor{curcolor}{0 0 0}
\pscustom[linestyle=none,fillstyle=solid,fillcolor=curcolor]
{
\newpath
\moveto(49.4560909,889.17445544)
\lineto(36.74440552,889.17445544)
\lineto(36.74440552,891.4237844)
\lineto(41.63351528,891.4237844)
\lineto(41.63351528,906.19240481)
\lineto(36.74440552,906.19240481)
\lineto(36.74440552,908.2049623)
\curveto(37.40680103,908.2049623)(38.11651052,908.25428969)(38.87353396,908.35294447)
\curveto(39.63055741,908.46146473)(40.2035821,908.61437963)(40.59260804,908.81168919)
\curveto(41.07626191,909.05832614)(41.45477363,909.36908869)(41.72814321,909.74397685)
\curveto(42.012027,910.12873049)(42.17499733,910.64173534)(42.21705419,911.28299141)
\lineto(44.66160907,911.28299141)
\lineto(44.66160907,891.4237844)
\lineto(49.4560909,891.4237844)
\closepath
}
}
{
\newrgbcolor{curcolor}{0 0 0}
\pscustom[linestyle=none,fillstyle=solid,fillcolor=curcolor]
{
\newpath
\moveto(273.0360982,672.04909658)
\lineto(260.32442753,672.04909658)
\lineto(260.32442753,674.29842289)
\lineto(265.21353163,674.29842289)
\lineto(265.21353163,689.06702591)
\lineto(260.32442753,689.06702591)
\lineto(260.32442753,691.07958103)
\curveto(260.98682228,691.07958103)(261.69653094,691.12890836)(262.45355351,691.22756302)
\curveto(263.21057608,691.33608315)(263.78360011,691.48899788)(264.1726256,691.6863072)
\curveto(264.65627891,691.93294386)(265.03479019,692.24370605)(265.30815945,692.61859377)
\curveto(265.59204292,693.00334695)(265.75501306,693.5163512)(265.79706986,694.15760651)
\lineto(268.24162192,694.15760651)
\lineto(268.24162192,674.29842289)
\lineto(273.0360982,674.29842289)
\closepath
}
}
{
\newrgbcolor{curcolor}{0 0 0}
\pscustom[linewidth=0.99999871,linecolor=curcolor]
{
\newpath
\moveto(295.3571452,680.01970394)
\lineto(299.64285354,680.01970394)
}
}
\end{pspicture}

                    \caption{A partition of unity for $\M = \mathbb{R}$}
                    \label{fig:partitionOfUnity}
                \end{figure}

