\section{Manifolds}
\subsection{Topology}
This chapter is basically taken from \citep{Schuller15} with our remarks to it.

We start with a set $M$ which is supposed to be the space where physics happens.
The weakest structure one needs in order to talk about \textit{continuity} (of curves or fields)
is called a topology.

\begin{defn}[Power set $\mathcal{P}$]
    The set of all subsets of $M$.
\end{defn}

\begin{defn}[Topology]
    A Topology $\mathcal{O}$ is a subset $\mathcal{O} \subseteq \mathcal{P}(M)$ satisfying:
    \begin{enumerate}
        \item $\emptyset \in \mathcal{O}$, $M\in\mathcal{O}$,
        \item $U\in\mathcal{O},\quad V\in\mathcal{O}\Rightarrow U\cap V\in\mathcal{O}$
        \item $U_\alpha\in\mathcal{O},\quad \alpha\in A \Rightarrow
            \left( \bigcup\limits_{\alpha\in A} U_\alpha \right) \in \mathcal{O}$
    \end{enumerate}
\end{defn}

Every set has the \textit{chaotic topology}
\begin{equation}
    \mathcal{O}_\text{chaotic} := \left\{ \emptyset, M \right\}\,,
\end{equation}
and the \textit{discrete topology}
\begin{equation}
    \mathcal{O}_\text{discrete} := \mathcal{P}(M)\,,
\end{equation}
which are both useless.
The special case $M = \mathbb{R}^d = \mathbb{R} \times \cdots \times \mathbb{R}$ has a standard topology
for which we need the definition of a soft ball.
\begin{defn}[Soft Ball in $\mathbb{R}^d$]
   \begin{equation}
   B_r(p) := \left\{ (q_1,\cdots,q_d) \left| \sum_{i=1}^{d}(p_i-p_i)^2 < r^2\right. \right\}\,,
   \end{equation} 
   with $r\in \mathbb{R}^+$, $p\in\mathbb{R}^d$.
   Note: This does not need a norm or vector space structure on $\mathbb{R}^d$.
\end{defn}
\begin{defn}[$\mathcal{O}_\text{standard}$ on $\mathbb{R}^d$]
    \begin{equation}
        U \in \mathcal{O}_\text{standard} :\Leftrightarrow \forall p\in U:
        \exists r\in\mathbb{R}^+ : B_r(p) \subseteq U
    \end{equation}
\end{defn}

\begin{figure}[h]
\centering
\begin{tikzpicture}
    %\draw[smooth cycle,tension=3.0] plot coordinates{(1,0) (1,1) (2,2) (3,3) (6,0)} node [label=right:$M$];

    % Manifold
    \draw[smooth cycle, tension=0.4] plot coordinates{(2,4) (-1.5,0) (3.5,-2) (6,1)} node [label=$M$, anchor = east] {};
    \draw[smooth cycle, tension=1, dashed] plot coordinates{(1,1) (-1,0) (1,-1) (2.5,0.5)} node [label=$U$, anchor = south west] {};
    %\draw[smooth cycle, tension=0.4, dashed] plot coordinates{(1, 2) (1,3) (3,3) (3,2)} node [label=$V$] {};
    \draw[smooth cycle, tension=0.4] plot coordinates{(1, 2) (1,3) (3,3) (3,2)} node [label=$V$, anchor = east] {};
    \draw[dashed] (-0.5,0) circle (.4cm) node [label = $B_r$, anchor = north west] {};
    \draw[fill] (-0.5,0) circle (.05cm) ;
    \draw (1.5,1.95) circle (.2cm) node [label = $B_r$, anchor = north west] {};
    \draw[fill] (1.5,1.95) circle (.05cm) ;

    %\draw[help lines] (-3,-6) grid (8,6);

\end{tikzpicture}
\caption{The set $U$ is in the standard topology, $V$ not.}
\end{figure}

Some terminology: Let $M$ be a set with a topology $\mathcal{O} =:$ set of open sets.
We call $(M, \mathcal{O}$ a \textit{topological space} and:
    \begin{itemize}
        \item $U \in \mathcal{O} \Leftrightarrow: \text{call } U\subseteq M$ an \textit{open set}
        \item $M\backslash A \in \mathcal{O} \Leftrightarrow: \text{call } A\subseteq M$  a \textit{closed set}
    \end{itemize}
\begin{note}
    The empty set is open and closed. If a set is open we cannot directly follow that it is not closed or vise versa. For $M = \left\{ 1, 2 \right\}$ and $\mathcal{O}_M = \left\{ \emptyset, \left\{ 1 \right\}, \left\{ 2 \right\}, \left\{ 1,2 \right\} \right\}$ the set $\left\{ 2 \right\}$ is open and closed.
\end{note}

\subsection{Continuous Maps}
A map
\begin{equation}
    f: M \to N\,,
\end{equation}
takes every point from the domain M (a set) to the target N (a set).
If at least one point $p\in N$ is not reached, the map is not \textit{surjective}.
If at least one point is hit twice, the map is not \textit{injective}.
A map that is injective and surjective is called \textit{surjective}.

\begin{defn}[Preimage]
    \begin{align}
        f&: M \to N \supseteq V \nonumber \\
        \text{preim}_f(V) &:= \left\{ m\in M~|~f(m) \in V \right\}
    \end{align}
\end{defn}

\begin{defn}[Continuity]
    $(M, \mathcal{O}_M)$ and $(N, \mathcal{O}_N)$ topological spaces.
    Then a map $f: M \to N$ is called \textit{continuous with respect to $\mathcal{O}_M$ and
    $\mathcal{O}_N$} if
    \begin{equation}
        \boxed{
    \forall V\in \mathcal{O}_N: \text{preim}_f(V) \in \mathcal{O}_M
}\,.
    \end{equation}
    \textit{``A map is open iff the preimages of all open sets are open sets.''}
\end{defn}

\begin{note}
    If a map is not surjective there are sets with preimage $\emptyset$, thus we need to have
    $\emptyset$ in $\mathcal{O}$, otherwise only surjective maps could be continuous.
\end{note}
\begin{note}
    The inverse of a continuous function does not need to be continuous.
\end{note}

\begin{defn}[Composition of maps]
    For $f$ and $g$
    \begin{equation*}
        f: M \to N, \quad g: N \to P\,,
    \end{equation*}
    we define the \textit{composition} as
    \begin{align}
        g \circ f&: M \to P \\
        m &\mapsto (g\circ f)(m) := g(f(m)) \nonumber
    \end{align}
\end{defn}

\begin{theorem}[Composition of continuous maps]
    For $f$, $g$ continuous also $g\circ f$ is continuous (if space match, \textit{i.e.}\ $g\circ f$ is defined).
\end{theorem}


\begin{defn}[Subset topology, Inherited topology]
    A set $M$ with topology $\mathcal{O}_M$. Given any subset $S \subseteq M$ we
    can construct the inherited topology $\mathcal{O}\eval_S \subseteq \mathcal{P}(S)$
    \begin{equation}
        \mathcal{O}\eval_S := \left\{ U\cap S~|~U\in \mathcal{O}M \right\}\,.
    \end{equation}
\end{defn}

\begin{note}
    For $S \subseteq M$, if $f$ is continuous then $f\eval_{S}$ is also
    continuous if $\mathcal{O}\eval_S$ is chosen. This is for example important
    if you are on a trajectory $\gamma$ through $\mathbb{R}^n$ and measure the temperature
    $T\eval_\gamma$.
\end{note}

\begin{defn}[Topological manifold]
    A topological space $(\M, \mathcal{O})$ is called a \textit{d-dimensional topological manifold}
    if
    \begin{equation}
        \forall p \in \M: \exists U \in \mathcal{O},~p\in U: \exists x: U\to x(U)\subseteq \mathbb{R}^{d}
        \,,
    \end{equation}
    with the following properties (wrt. $\mathcal O_\text{std}$ on $\mathbb{R}^{d}$):
    \begin{enumerate}
        \item $x$ intervitble: $x^{-1} : x(U) \to U$,
        \item $x$ continuous,
        \item $x^{-1}$ continuous\,.
    \end{enumerate}
    ``Invertible, in both directions continuous map to $\mathbb{R}^{n}$.''
\end{defn}

\begin{note}
    Thus in the above definition $x(U)$ is also open (from the definition of continuity).
\end{note}

\begin{terminology}~\\
    \begin{itemize}
        \item $(U,x)$ is a \textit{chart} of $\M, \mathcal{O}$,
        \item $\mathcal{A} = \left\{ (U_{(\alpha)}, x_{(\alpha)} | \alpha \in A \right\}$ is an \textit{atlas} of $(\M, \mathcal{O})$ if $\bigcup\limits_{\alpha\in A} U_{(\alpha)}$ covers the whole manifold $\M$,
        \item $x: U \to x(U) \subseteq \mathbb{R}^{d}$ is a \textit{chart map} 
            $x(p) = (x^1(p), \dots, x^d(p))$, where the \textit{component maps} $x^i: U\to\mathbb{R}$ are called \textit{coordinate maps},
        \item $p \in U$, then $x^1(p)$ is the first coordinate of the point p wrt.\ the chosen chart $(U, x)$.
    \end{itemize}
\end{terminology}
\begin{note}
    The choice of the chart (choice of coordinates) has nothing to do with the physics.
    Physics is chart independent. $\M$ is ``the real world''.
\end{note}

\subsection{Chart Transition Maps}

\begin{figure*}[tbh]
    \centering
    \begin{tikzpicture}
    % https://tex.stackexchange.com/questions/382762/drawing-manifolds-in-tikz
    % Functions i
        \path[->] (0.8, 0) edge [bend right] node[left, xshift=-2mm] {$x$} (-1, -2.9);
        \draw[white,fill=white] (0.06,-0.57) circle (.15cm);
        \path[->] (-0.7, -3.05) edge [bend right] node [right, yshift=-3mm] {$x^{-1}$} (1.093, -0.11);
        \draw[white, fill=white] (0.95,-1.2) circle (.15cm);

    % Functions j
        \path[->] (5.8, -2.8) edge [bend left] node[midway, xshift=-5mm, yshift=-3mm] {$y^{-1}$} (3.8, -0.35);
        \draw[white, fill=white] (4,-1.1) circle (.15cm);
        \path[->] (4.2, 0) edge [bend left] node[right, xshift=2mm] {$y$} (6.2, -2.8);
        \draw[white, fill=white] (4.54,-0.12) circle (.15cm);

    % Manifold
        \draw[smooth cycle, tension=0.4, fill=white, pattern color=brown, pattern=spray, opacity=0.7] plot coordinates{(2,2) (-0.5,0) (3,-2) (5,1)} node at (3,2.3) {$\M$};

    % Help lines
    %\draw[help lines] (-3,-6) grid (8,6);

    % Subsets
        \draw[smooth cycle, dashed, pattern color=orange, pattern=crosshatch dots, fill opacity = 0.4] 
        plot coordinates {(1,0) (1.5, 1.2) (2.5,1.3) (2.6, 0.4)} 
        node [label={[label distance=-0.3cm, xshift=-2cm, fill=white]:$U$}] {};
        \draw[smooth cycle, dashed, pattern color=blue, pattern=crosshatch dots, fill opacity = 0.4] 
        plot coordinates {(4, 0) (3.7, 0.8) (3.0, 1.2) (2.5, 1.2) (2.2, 0.8) (2.3, 0.5) (2.6, 0.3) (3.5, 0.0)} 
        node [label={[label distance=-0.8cm, xshift=.75cm, yshift=1cm, fill=white]:$V$}] {};

    % First Axis
        \draw[thick, ->] (-3,-5) -- (0, -5) node [label=above:$x(U)$] {};
        \draw[thick, ->] (-3,-5) -- (-3, -2) node [label=right:$\mathbb{R}^d$] {};

    % Arrow from i to j
        \draw[->] (0, -3.85) -- node[midway, above]{$y\circ x^{-1}$} (4.5, -3.85);

    % Second Axis
        \draw[thick, ->] (5, -5) -- (8, -5) node [label=above:$x(V)$] {};
        \draw[thick, ->] (5, -5) -- (5, -2) node [label=right:$\mathbb{R}^d$] {};

    % Sets in R^m
        \draw[white, dashed, pattern color=blue, pattern=crosshatch dots, fill opacity = 0.4] (-0.67, -3.06) -- +(180:0.8) arc (180:270:0.8);
        \fill[even odd rule, white] [smooth cycle] plot coordinates{(-2, -4.5) (-2, -3.2) (-0.8, -3.2) (-0.8, -4.5)} (-0.67, -3.06) -- +(180:0.8) arc (180:270:0.8);
        \draw[smooth cycle, dashed, pattern color = orange, pattern = crosshatch dots, fill opacity = 0.4] plot coordinates{(-2, -4.5) (-2, -3.2) (-0.8, -3.2) (-0.8, -4.5)};
        \draw[dashed] (-1.45, -3.06) arc (180:270:0.8);

        \draw[white, dashed, pattern color=orange, pattern=crosshatch dots, fill opacity = 0.4] (5.7, -3.06) -- +(-90:0.8) arc (-90:0:0.8);
        \fill[even odd rule, white] [smooth cycle] plot coordinates{(7, -4.5) (7, -3.2) (5.8, -3.2) (5.8, -4.5)} (5.7, -3.06) -- +(-90:0.8) arc (-90:0:0.8);
        \draw[smooth cycle, dashed, pattern color = blue, pattern = crosshatch dots, fill opacity = 0.4] plot coordinates{(7, -4.5) (7, -3.2) (5.8, -3.2) (5.8, -4.5)};
        \draw[dashed] (5.69, -3.85) arc (-90:0:0.8);

    \end{tikzpicture}

    \caption{Visualization of chart transition maps. ``How to glue together the charts of an atlas.'' Plot modified from~\citep{texstackexchange:manifolds}}
    \label{fig:transitionmaps}
\end{figure*}
Given $(U, x)$ and $(V, y)$ charts, on $U\cup V$ one can transition from
one chart to the other by (see figure~\ref{fig:transitionmaps})
\begin{equation}
    y\circ x^{-1}: \mathbb{R}^{d} \supseteq x(U\cap V) \to y(U\cap V) \subseteq \mathbb{R}^{d}\,,
\end{equation}
which is called the \textit{chart transition map}.
\begin{note}
    As a physicist one talks about a ``change in coordinates''.
\end{note}

\subsection{Manifold Philosophy}

The idea is to define properties of some object in the real world $\M$ by at a chart-representative
of it. For example the continuity of a curve $\gamma: [0,1]\to\M$ can be judged by looking at
$x\circ\gamma: [0,1]\to\mathbb{R}^{d}$, because $x$ is invertible and in both directions continuous
and the composition of two continuous maps is also continuous.
\begin{note}
    One needs to make sure that the property of the object on $\M$ does not depend on the
    map $x$ or $y$. For continuity this is the case, since 
    $y\circ\gamma = (y\circ x^{-1}) \circ x\circ \gamma$ and the chart transition map
    $y\circ x^{-1}$ is also continuous.
\end{note}

Other properties like ``differentiability'' are not even defined on $\M$ a priori,
so one can only talk about the chart representative. Here the definition that $\gamma$
is differentiable iff $x\circ\gamma: [0,1]\to\mathbb{R}^{d}$ is differentiable has the
problem that $x$ and $y$ only need to be continuous and so the chart transition map
$y\circ x^{-1}$ does not need to be differentiable unless one restricts oneself to only
differentiable charts.

